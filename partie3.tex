% \documentclass[10pt,a4paper]{article}
% \usepackage[utf8]{inputenc}
% \usepackage{amsmath}
% \usepackage{amsfonts}
% \usepackage{amssymb}
% \usepackage{array}
% \begin{document}
%% 	\chapter{Rappel}
% 		\subsection{Conséquence logique}
% 			$p$ est conséquence logique de $q$ si et seulement si $p \Rightarrow q$ est une tautologie. En d'autres termes, si
% 			\begin{center}
% 			\begin{tabular}{ll}
% 			$p \models q$ & $q$ est valide dans tous les modèles de $p$ \\
% 			&\\
% 			alors & \\
% 			$\models (p \Rightarrow q)$ & $p \Rightarrow q$ est une tautologie.\\
% 			&\\
% 			On peut donc écrire & \\
% 			$p \Rrightarrow q$ & $p$ est conséquence logique de $q$.\\
% 			\end{tabular}
% 			\end{center}
% 			Cependant, la conséquence logique ($\Rrightarrow$) n'est pas une proposition logique (cf. syntaxe d'une proposition).
% 		
% 		\subsection{Équivalence logique}
% 			Par le raisonnement ci-dessus, on peut dire que $p$ est logiquement équivalent à $q$ si et seulement si
% 			\begin{center}
% 			\begin{tabular}{ll}
% 			$p \models q$ & $q$ est valide dans tous les modèles de $p$ \\
% 			$q \models p$ & $p$ est valide dans tous les modèles de $q$ \\
% 			&\\
% 			et donc & \\
% 			$\models (p \Rightarrow q)$ & $p \Rightarrow q$ est une tautologie et\\
% 			$\models (p \Rightarrow q)$ & $p \Rightarrow q$ est une tautologie.\\
% 			&\\
% 			On peut donc écrire & \\
% 			%impossible de trouver l'équivalence logique en symbole
% 			$p \Lleftarrow \Rrightarrow q$ & $p$ est conséquence logique de $q$.\\ 
% 			\end{tabular}
% 			\end{center}
% 			L'équivalence logique n'est pas non plus une proposition logique.\\
% 			
% 			Il ne faut pas non plus oublier la différence entre phrase propositionnelle ($p$, $q$, $s$,...) et propositions primaires ($P$, $Q$, $S$,...) (cf. syntaxe d'une proposition) :
% 			\begin{center}
% 			\begin{tabular}{ll}
% 				$p \Rightarrow q$ & n'est pas une proposition\\
% 				
% 				mais &\\
% 				$P \land Q \Rightarrow R \land \lnot S$ & en est bien une.\\
% 			\end{tabular}
% 			\end{center}
% 	
	\chapter{Preuves en logique propositionnelle}
		Une preuve est un raisonnement déductif qui démontre si une proposition
est vraie ou fausse.
On distingue des preuves informelles et des preuves formelles.
Une preuve informelle est un raisonnement en langage naturel, parfois augmenté avec des
notations mathématiques.
Une preuve formelle est un objet mathématique qui formalise le raisonnement déductif.
Un des buts de la logique mathématique est de prouver le plus possibles des résultats
mathématiques avec des preuves formelles.

Au 20ème siècle les mathématiciens sont arrivés à prouver la plupart des mathématiques
classiques (telles qu'utilisées par des ingénieurs) avec des preuves formelles.
Un des résultats les plus célèbres est la preuve formelle du théorème des quatre couleurs,
fait par Georges Gonthier et Benjamin Werner avec l'assistant de preuve Coq (un logiciel
qui automatise la plupart des manipulations formelles nécessaires).
Ce théorème dit que toute carte découpée en régions connexes peut être colorée avec seulement
quatre couleurs, de sorte que deux régions adjacentes ont toujours des couleurs distinctes.

Dans ce chapitre nous allons définir des preuves formelles pour la logique propositionnelle.
Nous présenterons trois approches:
		\begin{itemize}
			\item Table de vérité
			\item Preuve transformationnelle
			\item Preuve déductive (la plus générale)		
		\end{itemize}

		\section{Preuve avec table de vérité}
La preuve formelle la plus simple est une table de vérité.
			\newcolumntype{x}{>{\itshape\bfseries}c}
		Prouvons que $\lnot (P \land Q) \Leftrightarrow (\lnot P \lor \lnot Q)$ est vrai :
			\begin{center}
			\begin{tabular}{cc|ccxcx}
			$P$ & $Q$ & $\lnot P$ & $\lnot Q$ & $(\lnot P \lor \lnot Q)$ & $P \land Q$ & $ (\lnot P \lor \lnot Q)$\\
			\hline
			F&F&T&T&T&F&T\\
			T&F&F&T&T&F&T\\
			F&T&T&F&T&F&T\\
			T&T&F&F&F&T&F\\
			\end{tabular}
			\end{center}

On peut constater que le vecteur de vérité de $\lnot (P \land Q)$ est équivalent à celui de $(\lnot P \lor  \lnot Q)$. La preuve a donc vérifié la véracité de la proposition.
Notez qu'une table de vérité est un objet mathématique en métalangage
parce qu'elle n'est pas une proposition.

L'inconvénient de cette méthode de preuve est qu'elle devient rapidement très lourde quand le nombre de propositions premières augmente. Il faut en effet $2^n$ lignes dans la table pour $n$ propositions. 

\section{Preuve transformationnelle}

Une preuve transformationnelle est une séquence de transformations
$p_1 \Lleftarrow \Rrightarrow p_2 \Lleftarrow \Rrightarrow \cdots \Lleftarrow \Rrightarrow p_n$,
dans laquelle on a toujours $p_i \Lleftarrow \Rrightarrow p_{i+1}$
(équivalence logique entre éléments adjacents dans la séquence).
Une preuve transformationnelle est aussi un objet mathématique en métalangage.
Pour faciliter la création d'une preuve transformationnelle,
on utilise des "Lois", c'est-à-dire des équivalences connues.
			\begin{center}
			\begin{tabular}{|ll|}
			\hline
			$p \Lleftarrow \Rrightarrow p \lor p$ & Idempotence\\
			$p \lor q \Lleftarrow \Rrightarrow q \lor p$ & Commutativité\\
			$(p \lor q) \lor r \Lleftarrow \Rrightarrow p \lor (q \lor r)$ & Associativité\\
			$ \lnot \lnot p \Lleftarrow \Rrightarrow p$ & Double Négation\\
			$p \Rightarrow q \Lleftarrow \Rrightarrow \lnot p \lor q$ & Implication\\
			$\lnot (p \land q) \Lleftarrow \Rrightarrow \lnot p \lor \lnot q$ & $1^{ere}$ loi de De Morgan\\
			$p \Leftrightarrow q \Lleftarrow \Rrightarrow (p \Rightarrow q) \land (q \Rightarrow p)$ & Équivalence\\
			\hline
			\end{tabular}
			\end{center}
On ajoute deux règles supplémentaires : la transitivité et la substitution.
			\subsection*{Transitivité de l'équivalence}
			\indent Si $p \Lleftarrow \Rrightarrow q$ et $q \Lleftarrow \Rrightarrow r$, alors $p \Lleftarrow \Rrightarrow r$.
			\subsection*{Substitution}
			Il est autorisé de remplacer une formule par une formule équivalente à l’intérieur d’une autre formule. Autrement dit : \\
			\indent Soit p,q,r des formules propositionnelles.\\
			\indent Si $p \Leftrightarrow q$ et $r(p)$, alors $r(p) \Lleftarrow \Rrightarrow r(q)$.\\
			On peut remplacer $p$ par $q$ car elles sont équivalentes. 
			
			
			\subsection*{Exemple}
			On veut prouver : $p \land (q \land r) \Lleftarrow \Rrightarrow (p \land q) \land r$
			\begin{center}
			\begin{tabular}{ll}
			
			$p \land (q \land r)$ & $\Lleftarrow \Rrightarrow p \land \lnot \lnot (q \land r)$\\
			& $\Lleftarrow \Rrightarrow p \land \lnot (\lnot q \lor \lnot r)$\\
			& $\Lleftarrow \Rrightarrow \lnot \lnot (p \land \lnot (\lnot q \lor \lnot r))$\\
			& $\Lleftarrow \Rrightarrow \lnot (\lnot p \lor \lnot \lnot (\lnot q \lor \lnot r))$\\
			& $\Lleftarrow \Rrightarrow \lnot (\lnot p \lor (\lnot q \lor \lnot r))$\\
			& $\Lleftarrow \Rrightarrow \lnot ((\lnot p \lor \lnot q) \lor \lnot r)$\\
			&$\vdots$\\
			& effectuer les mêmes lois dans le sens contraire \\
			&$\vdots$\\
			& $\Lleftarrow \Rrightarrow (p \land q) \land r$\\
			\end{tabular}
			\end{center}
Le problème de cette méthode de preuve est qu'elle requiert de l'intuition, de la créativité. Elle n'est donc pas forcément plus efficace que les tables de vérité, surtout si "l'astuce" est difficile à trouver.
		
\section{Preuve déductive}

Une preuve déductive est un objet mathématique qui formalise une séquence
de pas de raisonnement simples.
Ces pas sont une formalisation de trois différentes techniques de raisonnements:
les équivalences logiques, les règles d'inférence et les schémas de preuve.
En utilisant ces techniques,
une preuve déductive est beaucoup plus proche à une preuve informelle faite par un être humain,
tout en restant formelle.

\subsection{Equivalences logiques}
			\begin{center}
			\begin{tabular}{ll}
			$p \Lleftarrow \Rrightarrow p \lor p$ & Idempotence de $\lor$\\
			$p \lor q \Lleftarrow \Rrightarrow q \lor p$ & Commutativité de $\lor$\\
			$(p \lor q) \lor r \Lleftarrow \Rrightarrow p \lor (q \lor r)$ & Associativité de $\lor$\\
			$ \lnot \lnot p \Lleftarrow \Rrightarrow p$ & Double Négation\\
			$p \Rightarrow q \Lleftarrow \Rrightarrow \lnot p \lor q$ & Implication\\
			$\lnot (p \land q) \Lleftarrow \Rrightarrow \lnot p \lor \lnot q$ & $1^{ere}$ loi de De Morgan\\
			$\lnot (p \lor q) \Lleftarrow \Rrightarrow \lnot p \land \lnot q$ & $2^{eme}$ loi de De Morgan\\
			$(p \land q) \lor r \Lleftarrow \Rrightarrow (p \lor r) \land (q \lor r)$ & Distributivité de $\lor$\\
			\end{tabular}
			\end{center}
À ces équivalence nous ajoutons aussi l'idempotence, la commutativité, l'associativité et
la distributivité de $\land$.\\
			
\subsection{Règles d'inférence}
		
À la différence de la preuve transformationnelle, les règles d'inférences ont une direction : elles commencent par les prémisses et se terminent par la conclusion.
Pour chaque règle, si les prémisses sont vraies, alors la conclusion est vraie.
Nous utilisons un raisonnement informel pour justifier chaque règle.
		\begin{center}
		\begin{tabular}{llp{3cm}l}
		
   			Conjonction : &
   			\begin{tabular}{cl}
      		p & prémisse\\
      		q & prémisse\\
      		\line(1,0){25}&\\
      		$p\land q$ & Conclusion
   			\end{tabular} 
   			&
   			
   			Simplification : &
   			\begin{tabular}{c}
      		$p \land q$ \\
      		\hline
      		$p$\\
   			\end{tabular} \\
   			&\\
   			
   			Addition : &
   			\begin{tabular}{c}
      		$p $ \\
      		\hline
      		$p \lor q$\\
   			\end{tabular}
   			&
   			
   			Contradiction : &
   			\begin{tabular}{c}
      		$p$ \\
      		$\lnot p$\\
      		\hline
      		$q$\\
   			\end{tabular}\\
   			&\\
   			
   			Double Négation : &
   			\begin{tabular}{c}
      		$\lnot \lnot p $ \\
      		\hline
      		$p$\\
   			\end{tabular}
   			&
   			
   			\raggedright Transitivité de l'équivalence : &
   			\begin{tabular}{c}
      		$p \Leftrightarrow q$ \\
      		$q \Leftrightarrow r$ \\
      		\hline
      		$p \Leftrightarrow r$\\
   			\end{tabular}\\
   			&\\
   			
   			Modus Ponens : &
   			\begin{tabular}{c}
      		$p \Rightarrow q$ \\
      		$p$ \\
      		\hline
      		$q$\\
   			\end{tabular}
   			&
   			
   			Modus Tollens : &
   			\begin{tabular}{c}
      		$p \Rightarrow q$ \\
      		$\lnot q$ \\
      		\hline
      		$\lnot p$\\
   			\end{tabular}\\
   			&\\
   			
   			Loi d'équivalence : &
   			\begin{tabular}{c}
      		$p \Leftrightarrow q$ \\
      		\hline
      		$q \Leftrightarrow p$\\
   			\end{tabular}\\		
		\end{tabular}
		\end{center}
		
\subsection{Schémas de preuve}

En plus des équivalences logiques et des règles d'inférence,
nous ajoutons deux schémas de preuve qui formalisent des
techniques de raisonnement plus abstraites:
le théorème de déduction et la preuve par contradiction.
Ces schémas donnent à une preuve déductive une grande expressivité,
beaucoup plus qu'une preuve transformationnelle.

\subsubsection*{Théorème de déduction}
Pour prouver la proposition $s\Rightarrow t$, on suppose $s$ vrai.
La proposition $s$ s'ajoute donc aux prémisses utilisées dans la preuve.  
Ensuite, on fait une preuve de $t$:
on peut construire une preuve (objet mathématique) de $t$ en commençant de $s$.
On note ce théorème $s \vdash t$.
On écrit ce schéma un peu comme une règle d'inférence:
			\begin{center}
			\begin{tabular}{c}
      		$p,...,r,s \vdash t$ \\
      		\hline
      		$p,...,r \vdash s\Rightarrow t$\\
   			\end{tabular}\\
			\end{center}
On déduit $t$ et donc on sait que l'hypothèse $s\Rightarrow t$ est vraie et on l'évacue.
			
			\textbf{Remarque :} Il ne faut pas confondre les deux notations $p\models t$ et $p\vdash t$.
			\begin{itemize}
			\item $p\models t$ est une notion de vérité (tout modèle de $p$ est un modèle de $t$), et donc de sémantique;
			\item $ p\vdash t$ est une notion syntaxique (en commençant de $p$ on peut construire une preuve de $t$),
car une preuve est une séquence de manipulations syntaxiques.
			\end{itemize}
			
\subsubsection*{Preuve par contradiction}
		C'est une preuve indirecte :\\
		Implicitement, on suppose que $p$ jusqu'à $q$ n'a pas de problème, c'est-à-dire qu'on ne peut pas prouver une contradiction pour ces formules propositionnelles.
		\begin{center}
			\begin{tabular}{c}
      		$p,...,q,r \vdash s$ \\
      		$p,...,q,r \vdash \lnot s$\\
      		\hline
      		$p,...,q \vdash \lnot r$\\
   			\end{tabular}\\
			\end{center}
			
			Ceci signifie qu'il y a une erreur dans les prémisses, et on suppose ici que c'est la formule propositionnelle $r$ qui est fautive. On justifie qu'il n'y a aucune contradiction dans $,p,...,q$ car on part du principe qu'il existe un modèle de $p,...q$.\\
			
			Tout ceci est une formalisation de choses qu'on connaît déjà, mais ça nous donne une notion précise du raisonnement à avoir.
			
\section{Exemple de preuve propositionnelle}

Voici les propositions premières :
\begin{enumerate}
\item[A =] "tu manges bien"
\item[B =] "ton système digestif est en bonne santé"
\item[C =] "tu pratiques une activité physique régulière"
\item[D =] "tu es en bonne forme physique"
\item[E =] "tu vis longtemps"
\end{enumerate}

On peut maintenant établir une théorie dont on espère qu'elle aura un modèle. 

\paragraph{Théorie} 
\begin{enumerate}
\item $A \implies B$
\item $C \implies D$
\item $B \lor D \implies E$
\item $\lnot E$
\end{enumerate}

\paragraph{A Prouver} $\lnot A \land \lnot C$

\paragraph{Preuve}
\begin{tabular}{|l|l|}
\hline
1. A$\Rightarrow$B & prémisse \\
2. C$\Rightarrow$D & prémisse \\
3. B$\lor$D $\Rightarrow$E & prémisse \\
4. $\lnot$E & prémisse \\ 
\indent 5. A & hypothèse \\
\indent 6. B & modus ponens (1) \\
\indent 7. B$\lor$D & addition (6) \\
\indent 8. E & modus ponens (7) \\
9. $\lnot$A & preuve indirecte \\
\indent 10. C & hypothèse \\
\indent 11. D & modus ponens (2) \\
\indent 12. D$\lor$B & addition (11) \\
\indent 13. B$\lor$D & commutativité (12)\\
\indent 14. E & modus ponens (9) \\
15. $\lnot$C & preuve indirecte \\
16. $\lnot$A $\land$ $\lnot$C & conjonction (9,15) \\
\hline
\end{tabular}\\

Grâce à la déduction, nous avons pu prouver que tu ne manges pas bien et que tu ne pratiques pas d'activité physique régulière. 
On aimerait maintenant pouvoir automatiser les preuves quand elles existent. Mais il faut savoir s'il peut tout résoudre ou pas. 
On va donc construire un algorithme nous permettant de résoudre automatiquement les preuves en logique propositionnelle.
% \end{document}
