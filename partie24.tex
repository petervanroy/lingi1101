\section{L’Analyse des liens}
\begin{itemize}
\item Concentrateurs ( Hubs )
\item Autorités ( Authorities )
\item Comment trouver la meilleure page ?
\end{itemize}

Voir figure ci-dessous \\	

\includegraphics[scale=0.5]{images/ref/fig-14-1.jpeg}
\subsection{Requête News Papers}

\paragraph{Pourquoi Facebook, Yahoo, Amazon, se retrouvent-ils dans la requette News Papers ?}
Car beaucoup d'utilisateurs ont des pages concentrés sur ces sites et comme dans cet exemple on utilise un algorithme qui n'est pas très sophistiqué, elles apparaissent. 

\paragraph{Comment pouvoir trouver la meilleure page?}
\begin{enumerate}
\item Liens entrants $ \rightarrow $ votes
\item  Liens sortants  calcul du$ \rightarrow$  poids
\item Mise à jour des liens entrants
\item Mise à jour des liens sortants 
\end{enumerate}

\subsubsection{Algorithme}
Pages autorité (entrants (liens)) auto (p) \\
Pages concentrateurs (sortants ) conc (p)\\
Mises à jour :auto'(p) =  $ \sum_ {p'\rightarrow p}conc(p') $ \\
avec $p'\rightarrow p $ les pages p' qui ont un lien vers P  \\
conc'(p) = $ \sum_ {p\rightarrow p'}auto(p') $ \\
avec  $p \rightarrow p' $ les pages p' qui sont référencées à partir de p  \\

\subsubsection*{Itération}}

$ \forall p(page) $ : \[
\left \{
\begin{array}{r c l}
auto(p)  & = & 1 \\
conc (p)   & = & 1 \\

\end{array}
\right .
\]
 \\
 Mise à jour , normalisation, auto'(p) = $\frac{auto (p)}{  \sum auto(p')  }$\\
 conc' (p) = $\frac{conc (p)}{  \sum conc(p)  }$\\
 Cet algorithme converge \\
 \\
 En appliquant maintenant cet algorithme,  la Requête News Papers nous donnerait les résultats suivants:\\
 \includegraphics[scale=0.5]{images/ref/fig-14-2.jpeg}
 
 \includegraphics[scale=0.5]{images/ref/fig-14-3.jpeg}
 
 \includegraphics[scale=0.5]{images/ref/fig-14-4.jpeg}
 
 \includegraphics[scale=0.5]{images/ref/fig-14-5.jpeg}
	
Comme nous pouvons voir, l'algorithme compte le nombre de liens entrants pour attribuer un poids à chaque place. Puis on normalise les valeurs trouvées, ce qui correspond au poids de la page.
Au plus grand est le poids d'une page au plus son autorité sera grande.\\
\section{PageRank}
\begin{itemize}
\item Consolider autorités et concentrateurs.
\item Une valeur par noeud $ \rightarrow $ son "PageRank" que nous allons calculer. 
\item Intuition: Un "fluide" qui circule dans le réseau. 
\end{itemize}
\textbf{ Algorithme PageRank :}
\begin{enumerate}
\item N noeuds (chaque noeud représentant une page) : Initialisation Pr(p) =  $\frac{1}{n}$
\item Choisir un nombre de pas k
\item K mises à jour:\\
Pr(p) = $ \sum_ {p'}\frac{Pr(p')}{n(p')} $ avec n(p') le nombre de liens sortant de p' et Pr(p') le poids (ou PageRank de p') à la $k^{ème}$ itération.
\end{enumerate}
\textbf{Exemple de PageRank :}\\
Voir graphe ci-dessous : \\ 
 \includegraphics[scale=0.8]{images/24_imagePr.pdf}
 
 	\begin{tabular}{|c| c |c |c |}
		\hline
		Noeuds/Itérations & 0 & 1 & 2 \\
		\hline
		A & 1/8 & 1/2 & 5/16 \\
		B & 1/8 & 1/16 &  1/4   \\
		C & 1/8 & 1/16 & 1/4    \\
		D & 1/8 & 1/16 & 1/32  \\
		E & 1/8 & 1/16 &  1/32   \\
		F & 1/8 & 1/16 & 1/32    \\
		G & 1/8 & 1/16 & 1/32    \\
		H & 1/8 & 1/8 &  1/16   \\
		\hline
		$\sum $ & 1 & 1 & 1 \\
		\hline
	\end{tabular}

 Si nous continuons à laisser travailler l'algorithme, pour un nombre n fini d'itérations telles que k=n, il y aura convergence de l'algorithme. Dans cet exemple-ci,  nous aurons :\\
 \\
 Pr(A) = 4/13 ; Pr(B) = 2/13 ; Pr(C) = 2/13 ; Pr(Autres) = 1/13 \\
 Et la condition $ \sum Pr(p) = 1 $ est toujours vérifiée .
 
 L'équilibre est vérifié si le graphe est connexe, s’il ne l'est pas un problème se pose: Le fluide peut arriver au mauvais noeud (analogie réseau d'eau )  :

\includegraphics[scale=0.8]{images/24_Nconnexe.pdf}

\subsubsection*{Solution du problème}
\begin{itemize}
\item Fermer la boucle
\item Créer des cycles
\item Analogie : Circulation d'eau dans l'atmosphère
\end{itemize}
 	Il faut un peu de fluide partout dans chaque itération.\\
 	Nouvelle règle de mise à jour : \\
 	
      Pr(p) = $ \sum_ {p'}\frac{Pr(p')}{n(p')} $ (ancien) \\
     \\
         Pr(p) = Sx(Pr(p) + (1-S)x$\frac{1}{n}$) \\
         S est un paramètre : $ 0 \le S \le 1 $ \\
         \\
        \textbf{ Autre manière de voir l'algorithme :} \\
        \\
        Marche aléatoire : 
        \\
        Probabilité de S : Suivre un lien \\
        (1-S) : Choisi un nœud au hasard \\
        $\rightarrow $ La même valeur pour Pr(p) \\
        PageRank : (début 1990) \\
        - Abandon partiel en 2003/2004 à cause de SEO : Search engine optimisation (Tricheurs )
