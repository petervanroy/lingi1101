\section{L’Analyse des liens}
\begin{itemize}
\item Concentrateurs ( Hubs )
\item Autorités ( Authorities )
\item Comment trouver la meilleure page ? (Voir ~\ref{fig-14-1}	)
\end{itemize}



\begin{figure}[!h]
\centering
\includegraphics[scale=0.5]{images/ref/fig-14-1.jpeg}
\caption{Counting in-links to pages for the query "newspapers.	"}
\label{fig-14-1}
\end{figure}

\subsection{Requête News Papers}

\paragraph{Pourquoi Facebook, Yahoo, Amazon, se retrouvent-ils dans la requette News Papers ?}
Car beaucoup d'utilisateurs ont des pages concentrées sur ces sites et comme dans cet exemple on utilise un algorithme qui n'est pas très sophistiqué, elles apparaissent. 

\paragraph{Comment trouver la meilleure page ?}
\begin{enumerate}
\item Liens entrants $ \rightarrow $ votes
\item Liens sortants  calcul du$ \rightarrow$  poids
\item Mise à jour des liens entrants
\item Mise à jour des liens sortants 
\end{enumerate}

\subsubsection{Algorithme}

\begin{itemize}
\item Pages autorité (liens entrants) $ \rightarrow $ auto (p)
\item Pages concentrateurs (liens sortants) $ \rightarrow $ conc (p)
\item Mises à jour des liens entrants:
\begin{align*}
 auto'(p) =   \sum_ {p'\rightarrow p}conc(p') &&
\text{\emph{avec $p'\rightarrow p $ les pages p' qui ont un lien vers P}}
\end{align*}

\item Mises à jour des liens sortants: 
\begin{align*}
conc'(p) = \sum_ {p\rightarrow p'}auto(p') &&
\text{\emph{avec $p \rightarrow p' $ les pages p' qui sont référencées à partir de p}} 
\end{align*}

\end{itemize}



\subsubsection*{Itération}}

\begin{equation}
\forall p(page) :
\left \{
\begin{array}{l}
auto(p) = 1 \\
conc (p) = 1 

\end{array}
\right.
\end{equation}

 Mise à jour, normalisation :
 
 \begin{align*}
 auto'(p) = \frac{auto (p)}{\sum auto(p')}
 \end{align*}
 \begin{align*}
 conc' (p) = \frac{conc (p)}{\sum conc(p)}
 \end{align*}
 Cet algorithme converge


En appliquant maintenant cet algorithme, la Requête News Papers nous donnerait les résultats suivants (~\ref{pageRankNews1} et ~\ref{pageRankNews2}):


\begin{figure}
\centering
\includegraphics[scale=0.5]{images/ref/fig-14-2.jpeg}
\includegraphics[scale=0.5]{images/ref/fig-14-4.jpeg}
\caption{Page Rank}
\label{pageRankNews1}
\end{figure}

\begin{figure}
\centering
\includegraphics[scale=0.5]{images/ref/fig-14-3.jpeg}
\includegraphics[scale=0.5]{images/ref/fig-14-5.jpeg}
\caption{Page Rank}
\label{pageRankNews2}
\end{figure}



\newpage

	
Comme nous pouvons voir, l'algorithme compte le nombre de liens entrants pour attribuer un poids à chaque place. Puis on normalise les valeurs trouvées, ce qui correspond au poids de la page.
Au plus grand est le poids d'une page au plus son autorité sera grande.\\
\section{PageRank}
\begin{itemize}
\item Consolider autorités et concentrateurs.
\item Une valeur par noeud $ \rightarrow $ son "PageRank" que nous allons calculer. 
\item Intuition: Un "fluide" qui circule dans le réseau. 
\end{itemize}
\textbf{ Algorithme PageRank :}
\begin{enumerate}
\item N noeuds (chaque noeud représentant une page) : Initialisation Pr(p) =  $\frac{1}{n}$
\item Choisir un nombre de pas k
\item K mises à jour:\\
Pr(p) = $ \sum_ {p'}\frac{Pr(p')}{n(p')} $ avec n(p') le nombre de liens sortant de p' et Pr(p') le poids (ou PageRank de p') à la $k^{ème}$ itération.
\end{enumerate}
\subsection*{Exemple de PageRank :}

\begin{figure}[h!]
\centering
 \includegraphics[scale=0.8]{images/24_imagePr.pdf}
\label{graphPageRank}
\end{figure}

 
 	\begin{tabular}{|c| c |c |c |}
		\hline
		Noeuds/Itérations & 0 & 1 & 2 \\
		\hline
		A & 1/8 & 1/2 & 5/16 \\
		B & 1/8 & 1/16 &  1/4   \\
		C & 1/8 & 1/16 & 1/4    \\
		D & 1/8 & 1/16 & 1/32  \\
		E & 1/8 & 1/16 &  1/32   \\
		F & 1/8 & 1/16 & 1/32    \\
		G & 1/8 & 1/16 & 1/32    \\
		H & 1/8 & 1/8 &  1/16   \\
		\hline
		$\sum $ & 1 & 1 & 1 \\
		\hline
	\end{tabular}


 Si nous continuons à laisser travailler l'algorithme, pour un nombre n fini d'itérations telles que k=n, il y aura convergence de l'algorithme. Dans cet exemple-ci,  nous aurons :

	\begin{align*}
 	Pr(A) = 4/13 ; Pr(B) = 2/13 ; Pr(C) = 2/13 ; Pr(Autres) = 1/13 \\
 	\text{Et la condition} \sum Pr(p) = 1 ~\text{est toujours vérifiée.}
	\end{align*}
 
 L'équilibre est vérifié si le graphe est connexe,par contre si il ne l'est pas un problème se pose: Le fluide peut arriver au mauvais noeud (analogie réseau d'eau )  :

\includegraphics[scale=0.8]{images/24_Nconnexe.pdf}

\subsection*{Solution du problème}
\begin{itemize}
\item Fermer la boucle
\item Créer des cycles
\item Analogie : Circulation d'eau dans l'atmosphère
\end{itemize}
 	La solution est de réinjecter un peu de fluide partout à chaque itération pour éviter que le fluide se concentre dans les noeuds qui n'ont que des liens entrants et pas de liens sortant.

	Ancienne règle de mise à jour :
	\begin{align*}
	Pr(p) =  \sum_ {p'}\frac{Pr(p')}{n(p')}
	\end{align*}
 	Nouvelle règle de mise à jour : \\
	\begin{align*}
         Pr(p) = Sx(Pr(p) + (1-S)  \times \frac{1}{n}) \\
	 \text{Ou S est un paramètre : $ 0 \le S \le 1 $} 
	\end{align*}
        
\subsection*{ Une autre manière de voir l'algorithme:}
        Marche aléatoire d'un utilisateur sur le web : 
	\begin{itemize}
        \item Probabilité de S : Suivre un lien dans la page web ou l'on se trouve \\
        \item (1-S) : Choisi un n\oe ud au hasard, par exemple, taper une adresse URL et accéder directement à un site \\
        \item $\rightarrow $ La même valeur pour Pr(p) \\
	\end{itemize}
        PageRank : (début 1990) \\
        - Abandon partiel en 2003/2004 à cause des SEO : Search engine optimisation (Tricheurs)

