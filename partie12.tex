\section{Opérations sur les théories}

Nous allons définir une opération, l'extension, qui permet de créer une nouvelle théorie par rapport à une théorie existante,
et deux opérations, l'inclusion et l'équivalence, qui permettent de comparer deux théories.

\subsection{Extension d'une théorie}
Lorsqu'on a une théorie déjà existante, on souhaite parfois l'étendre afin de la rendre plus complète.
On dit que $Th_2$ est une {\em extension} de $Th_1$ si:
\begin{itemize}
\item[$\bullet$] Le vocabulaire de $Th_1$ est inclus dans le vocabulaire de $Th_2$.
\item[$\bullet$] Tout axiome de $Th_1$ est axiome de $Th_2$.
\end{itemize}
On peut donc étendre le vocabulaire et ajouter des axiomes.
Voici deux exemples d'extension de théorie pour mieux comprendre comment effectuer cette opération. Ils étendent tous les deux
la théorie des liens familiaux (\bsc{fam}) décrite dans la section précédente.

\subsubsection{Exemple 1}

Considérons le nouvel axiome suivant, que nous noterons Ax:
$$ Ax \equiv (\forall x) \neg P(x,x) $$
La nouvelle théorie ainsi étendue que nous noterons \bsc{fam*} possède un axiome de plus : Ax. Cette théorie \bsc{fam*}
est {\em consistante}, c'est-à-dire qu'il existe au moins un modèle qui valide cette théorie.
Pour s'en convaincre, il suffit de considérer la première interprétation de la théorie \bsc{fam} de la section précédente,
qui utilise les liens familiaux: personne est parent de lui-même.\\

En revanche, si on considère la deuxième interprétation (deuxième modèle, noté $J$) de \bsc{fam}
qui associe les symboles $p$ et $m$ aux fonctions mathématiques $p_J : \mathbb{N} \rightarrow \mathbb{N} : d \rightarrow 2d$ et $m_J : \mathbb{N} \rightarrow \mathbb{N} : d \rightarrow 3d$, on observe une contradiction.
En effet, dans le modèle $J$ on a la définition suivante du prédicat $P$ :
$$ P_J(d_1, d_2) \textrm{ ssi } d_2 = 2d_1 \textrm{ ou } d_2 = 3d_1$$
Il suffit de choisir $x=0$ dans notre nouvel axiome Ax pour constater que le modèle $J$ ne valide pas la théorie étendue \bsc{fam*}.
De manière générale, l'extension d'une théorie peut donc {\em réduire} l'ensemble des modèles de celle-ci.

\subsubsection{Exemple 2}

Considérons le nouvel axiome suivant, que nous noterons Adam:
$$ (\forall y) \neg P(a,y) $$
où $a$ est une constante arbitraire.
Notons la théorie étendue \bsc{fam'} = \bsc{fam} + Adam. Dans cet exemple, on peut observer que \bsc{fam'}
est {\em inconsistante}, car aucun modèle ne peut valider cette théorie.
En effet, en partant du premier axiome de \bsc{fam} (appelé "père"), nous effectuons quelques étapes pour obtenir une contradiction:
\begin{align*}
& (\forall x) P(x,p(x)) && \textrm{Axiome père} \\
& P(a,p(a)) && \textrm{Elimination } \forall \\
& (\exists y) P(a,y) && \textrm{Introduction} \exists
\end{align*}
Ci-dessus, le premier axiome de \bsc{fam} reformulé (père),
qui est en contradiction avec le nouvel axiome (Adam), que nous reformulons ci-dessous.
\begin{align*}
& (\forall y) \neg P(a,y) \\
\Leftrightarrow & \ \neg (\exists y) P(a,y)
\end{align*}
Par la règle de preuve par contradiction, on démontre qu'aucun modèle n'est possible pour la théorie étendue \bsc{fam'}.
Nous avons vu que
l'extension d'une théorie peut réduire le nombre de modèles: ici on voit qu'il faut bien vérifier que ce nombre ne devient pas 0.

\subsection{Comparaison des théories}

Dans cette section, nous abordons la comparaison de différentes théories :
inclusion, équivalence et quelques corollaires.
Dans ce qui suit, on prend $Th_1$ et $Th_2$ deux théories.

\subsubsection{Inclusion}
On dit que $Th_1$ est {\em contenue} dans $Th_2$ si
\begin{itemize}
\item[$\bullet$] Le vocabulaire de $Th_1$ est inclus dans le vocabulaire de $Th_2$.
\item[$\bullet$] Toute formule valide dans $Th_1$ l'est aussi dans $Th_2$.
\end{itemize}
Tout modèle de $Th_2$ est donc aussi un modèle de $Th_1$.
Mais pas inversément: il est possible qu'il existe des modèles de $Th_1$ qui ne sont pas des modèles de $Th_2$.
C'est parce que $Th_2$ peut contenir des formules valides qui ne le sont pas dans $Th_1$.

\subsubsection{Équivalence}
On dit que $Th_1$ et $Th_2$ sont {\em équivalentes} si elles sont contenues l'une dans l'autre.
Cela signifie que les deux théories "disent la même chose" et que tout modèle d'une des théories est également modèle de l'autre.\\


Il est important de bien faire la différence entre
l'extension d'une théorie et l'inclusion d'une théorie dans une autre.
La confusion est possible parce que dans les deux cas, le nombre de modèles peut être réduit.
Mais il ne faut pas oublier que
l'extension est définie par rapport aux axiomes et l'inclusion est définie par rapport aux modèles.
Pour l'extension on ajoute des axiomes et on ne regarde pas les modèles.
Poir l'inclusion on compare les modèles et on ne regarde pas les axiomes.
On peut dire que l'extension est une manipulation {\em syntaxique} (on change les axiomes ce qui permette de nouvelles preuves)
et l'inclusion est une manipulation {\em sémantique} (on fait une comparaison entre ce que modélisent les deux théories).

\subsection{Corollaires}
On note respectivement $V_t$, $M_t$ et $Ax_t$ le vocabulaire, les modèles et les axiomes d'une théorie $t$. \\

\noindent \underline{Si} $V_{Th_1} \subseteq V_{Th_2}$ et $M_{Th_2} \subseteq M_{Th_1}$
(tout modèle de $Th_2$ est aussi modèle de $Th_1$) \underline{alors} $Th_1$ est contenue dans $Th_2$.
Attention à la direction: $M_{Th_2}$ est bien à gauche!
\\

\noindent \underline{Si} $V_{Th_1} \subseteq V_{Th_2}$ et $Ax_{Th_1} \subseteq Ax_{Th_2}$
(tout axiome de $Th_1$ est aussi axiome de $Th_2$) \underline{alors} $Th_1$ est contenue dans $Th_2$.
Il est donc vrai que $M_{Th_2} \subseteq M_{Th_1}$.
On peut donc dire qu'une extension donne lieu à une inclusion (mais pas inversément).\\

\noindent \underline{Si} $V_{Th_1} = V_{Th_2}$ 
et pour tout axiome $p \in Ax_{Th_1}$ nous avons $\models_{Th_2} p$
et pour tout axiome $q \in Ax_{Th_2}$ nous avons $\models_{Th_1} q$,
\underline{alors}
$Th_1$ et $Th_2$ sont équivalentes. \\

\noindent \underline{Si} $p$ une formule fermée telle que $\models_{Th_1} p$ et $Th_2 = Th_1 \bigcup \left\lbrace p \right\rbrace$ \underline{alors} $Th_1$ et $Th_2$ sont équivalentes. \\

\section{Théorie des ordres partiels stricts (OPS)}

Le concept d'ordre partiel strict est un des concepts de base des mathématiques.
Nous allons le définir par une théorie.
Ensuite nous donnons deux interprétations différentes de cette théorie.
\subsection*{Vocabulaire}
\begin{itemize}
\item[$\bullet$] Le symbole P/2
\end{itemize}
\subsection*{Axiomes}
\begin{itemize}
\item[$\bullet$] $ (\forall x) \neg P(x,x) $  (irréflexivité) (OPS1)
\item[$\bullet$] $ (\forall x) (\forall y) (\forall z) (P(x,y) \wedge P(y,z) \Rightarrow P(x,z))$ (transitivité) (OPS2)
\end{itemize}
\subsection*{Première interprétation}
Soit l'interprétation $I$ de cette théorie:
\begin{itemize}
\item[$\bullet$] $D_I = \mathbb{Z} $
\item[$\bullet$] $\mathrm{val}_I (P) = "<"$
\end{itemize}
Lorsque l'on écrit $\mathrm{val}_I(<) = "<"$, le premier symbole $<$ est un mot de vocabulaire dans la théorie
tandis que le deuxième est une fonction sur les entiers.
Cette fonction confère un sens au symbole. 
On remarque que cette interprétation est un modèle de notre théorie.
En effet, la fonction $<$ sur les entiers est irréflexive et transitive.
Si on a $x < y$ et $y <z$, on peut en déduire que $x<z$.

\subsection*{Deuxième interprétation}
Soit l'interprétation $J$ de cette théorie:
\begin{itemize}
\item[$\bullet$] $D_J = \mathbb{Z} $
\item[$\bullet$] $val_J (P) = "\ne"$
\end{itemize}
Ici on change le sens du symbole $<$ :
$$val_J(<) = "\ne"$$
Cette interprétation n'est pas un modèle. En effet, par exemple, pour $x=5, y=3, z=5$, on a :
$$x<y \wedge y<z \nRightarrow x<z$$
$$5 \neq 3 \wedge 3 \neq 5 \nRightarrow 5 \neq 5$$
Ceci est un contre-exemple, car $5$ n'est pas différent de $5$ (irréflexivité),
et donc cette interprétation ne vérifie pas l'axiome sur la transitivité!

