\documentclass[a4paper,11pt,final]{report}
% Pour une impression recto verso, utilisez plutôt ce documentclass :
%\documentclass[a4paper,11pt,twoside,final]{article}

\usepackage[english,francais]{babel}
\usepackage[normalem]{ulem}
\usepackage[utf8]{inputenc}
\usepackage[T1]{fontenc}
\usepackage[pdftex]{graphicx}
\usepackage{setspace}
\usepackage{hyperref}
% \usepackage[french]{varioref}
\usepackage{amsmath}
\usepackage{algorithm2e} % needed by: 15
\usepackage{mathtools}
\usepackage{systeme}
\usepackage{amssymb}
\usepackage[usenames,dvipsnames]{color}
\usepackage[usenames,dvipsnames,svgnames,table]{xcolor}
\usepackage{cancel} % needed by: 9
\usepackage{framed}
\usepackage{array}
%\usepackage{enumitem} % needed by: 17
\usepackage{lmodern} % needed by: 17
\usepackage{listings} % needed by: 17
\usepackage{color} % needed by: 17
\usepackage{tikz} % needed by: 17
\usepackage{array} % needed by: 17

\definecolor{mygreen}{rgb}{0,0.6,0} % needed by: 17
\definecolor{mygray}{rgb}{0.5,0.5,0.5} % needed by: 17
\definecolor{mymauve}{rgb}{0.58,0,0.82} % needed by: 17

\selectlanguage{french}

\DeclareUnicodeCharacter{22A8}{\tautologie}
\DeclareUnicodeCharacter{22AD}{\contradiction}

\newcommand{\reporttitle}{LINGI1101\\
\vspace{\baselineskip}
Logique et Structures Discrètes}     % Titre
\newcommand{\reportauthor}{Peter \textsc{Van Roy}} % Auteur
\newcommand{\reportsubject}{ } % Sujet
\newcommand{\HRule}{\rule{\linewidth}{0.5mm}}
\setlength{\parskip}{1ex} % Espace entre les paragraphes

\newcommand{\true}{\mathrm{true}}
\newcommand{\false}{\mathrm{false}}
\newcommand{\val}{\mathrm{val}}
\newcommand{\VAL}{\mathrm{VAL}}

\hypersetup{
    pdftitle={\reporttitle},%
    pdfauthor={\reportauthor},%
    pdfsubject={\reportsubject},%
    pdfkeywords={rapport} {vos} {mots} {clés}
}

\begin{document}
  % Inspiré de http://en.wikibooks.org/wiki/LaTeX/Title_Creation

\begin{titlepage}

\begin{center}

\begin{minipage}[t]{0.8\textwidth}
  \begin{center}
    \includegraphics [width=70mm]{images/logo_Ucl.jpg} \\[0.5cm]
    % \begin{spacing}{1.5}
    %   \textsc{\LARGE Université catholique de Louvain}
    % \end{spacing}
  \end{center}
\end{minipage}
% \begin{minipage}[t]{0.48\textwidth}
%   \begin{flushright}
%     \includegraphics [width=30mm]{images/logo-societe.jpg} \\[0.5cm]
%     \textsc{\LARGE Entreprise}
%   \end{flushright}
% \end{minipage} \\[1.5cm]

\textsc{\Large \reportsubject}\\[0.5cm]
\HRule \\[0.4cm]
{\huge \bfseries \reporttitle}\\[0.4cm]
\HRule \\[1.5cm]

\begin{minipage}[t]{0.5\textwidth}
  \begin{center} \large
    \emph{Titulaire:} \reportauthor
  \end{center}
\end{minipage}
% \begin{minipage}[t]{0.6\textwidth}
%   \begin{flushright} \large
%     \emph{Responsables :} \\
%     M.~Jean \textsc{Machin} \\
%     M.~Pierre \textsc{Bidon}
%   \end{flushright}
% \end{minipage}

\vfill

{\large 2015}

\end{center}

\end{titlepage}

  \cleardoublepage % Dans le cas du recto verso, ajoute une page blanche si besoin
  \tableofcontents % Table des matières
  \sloppy          % Justification moins stricte : des mots ne dépasseront pas des paragraphes
  \cleardoublepage
  \section*{Remerciements}
\addcontentsline{toc}{section}{Remerciements}

Je tiens à remercier les étudiants de LINGI1101 pour avoir pris des
notes pendant mon cours, ce qui faisait la base de ce syllabus.  Les
contributeurs sont:
Antoine Walsdorff,
Goeric Huybrechts,
Romane Schelkens,
Nicolas Van Wallendael, % Partie 1
Kilian Verhetsel,
Cyril de Vogelaere,
Jonathan Legat, % Partie 2
Siciliano Damiano-Joseph,
Aghakhani Ghazaleh,
Kühn Alexandre,
Maas Dylan, % Partie 3
Gerniers Alexander,
Ndizera Eddy,
El Jilali Solaiman,
Dhillon Sundeep, % Partie 18
Dagnely Vincent,
Dethise Arnaud,
Bellenger Jordan,
Schmitz Loic, % Partie 23
Pignolet Aurélien,


  \cleardoublepage
  % \section*{Introduction} % Pas de numérotation
\addcontentsline{toc}{section}{Introduction} % Ajout dans la table des matières

Ce document est le syllabus du cours LINGI1101 ``Logique et Structures Discrètes''
donné par Peter Van Roy.

% Ce document est un exemple de rapport. J'espère aider des étudiants à réaliser leur rapport en \LaTeX.
% Écrit par Bruno Voisin (Hiko Sejûrô) et publié sur \url{http://blog.hikoweb.net/}.

  % \cleardoublepage
  \chapter*{Introduction au cours LINGI1101}

Le cours "\textit{Logique et Structures discrètes"} a deux buts importants:

\begin{itemize}
\item Donner la motivation et l'intuition de la logique, pour que cette matière devienne véritablement utile pour les étudiants.
\item Donner les concepts et les formalismes mathématiques nécessaires pour utiliser la logique à bon escient.
\end{itemize}


L'intuition est donc importante pour ce cours, néanmoins, la connaissance des formalismes mathématiques reste essentielle.  Le cours sera coté sur les deux: intuitions (un tiers) et formalismes (deux tiers).\\

\section*{Déroulement du cours}

Le cours est composé de deux parties. La première partie, \textit{logique formelle}, représentera deux tiers du cours. La seconde partie, \textit{structures discrètes sur Internet}, comptera quant à elle pour un tiers du cours.\\

L'évaluation de ce cours se compose de trois parties. Il y aura tout d'abord une interrogation au milieu du quadrimestre portant sur 5 points. Il vous sera également demandé de prendre note pendant une heure de cours par groupe de trois, ceci afin de contribuer au syllabus. Ces notes prises au cours rapporteront au maximum 2 points de la note finale à chacun des participants. L'examen sera divisé en deux parties. La première partie sur 5 points portera sur la matière de l'interrogation. La note retenue sera le maximum entre la note de l'interrogation et celle obtenue à la question de l'examen. La seconde partie de l'examen sera donc cotée sur 13 points et portera sur le reste de la matière.\\

Afin de suivre ce cours, nous nous baserons sur deux livres de référence correspondants chacun à une partie du cours :

\begin{itemize}
\item Introductory Logic and Sets for Computer Scientists, by \textit{Nimal Nissanke}.
\item Networks, Crowds, and Markets: Reasoning About a Highly Connected World, by \textit{David Easley and Jon Kleinberg}. \footnote{Quelques chapitres.}
\end{itemize}

La première partie sera complétée par des sujets et exercices plus avancés qui approfondissent le traitement du livre.

\section*{Plan du cours}

Cette partie va parler du rôle des raisonnements et des différentes formes de raisonnement. Nous prendrons en exemple la méthode scientifique.

\subsection*{Logique des propositions}

La logique des propositions est un langage formel constitué d'une syntaxe et d'une sémantique. La syntaxe décrit l'ensemble des formules qui appartiennent au langage. La sémantique permet de donner un sens aux formules de langage. C'est une logique très ancienne qui vient de l'antiquité. 

\subsection*{Logique des prédicats}

C'est une logique beaucoup plus expressive et la plupart des travaux mathématiques peuvent être écrits dans ce langage\footnote{Elle est un effet un parfait compromis entre expressivité et efficacité.}. Elle est aussi définie comme la logique du premier ordre.\footnote{Il existe d'autres formes de logiques plus expressives, mais plus difficiles à utiliser. Exemple : la logique du deuxième ordre. } En logique des prédicats, les éléments de base du langage ne sont plus des propositions, mais des prédicats. 

\subsection*{Interprétations et modèles}

La logique a besoin d'un langage, de phrases pour la décrire. Cette section couvrira donc la sémantique à utiliser.

\subsection*{Théorie de la preuve}

Nous pouvons manipuler une phrase en logique pour obtenir un résultat. Par exemple, si A et B sont vrais, nous pouvons en déduire que A est vrai. Il y a des règles d'inférences à utiliser pour prendre une phrase en logique et en déduire une autre.
Une preuve mathématique est une séquence de phrases liées par des règles d'inférences.

\subsection*{Algorithme de preuve}
C'est l'algorithme le plus puissant qui existe en logique des prédicats. Néanmoins, il est inefficace seul. Afin de le rendre efficace, il faut poser des hypothèses. Nous approfondirons ce problème dans le cadre de cette section. 

\subsection*{Théorie logique}

Il est possible de formaliser tout objet mathématique avec une théorie logique qui lui est propre.  En exemple, citons la théorie des ensembles, des fonctions et des ordres partiels.

\subsection*{Programmation logique}

Le rêve serait de pouvoir exprimer toute chose logique en langage de programmation efficace. Il s'agira d'appliquer ce principe avec l'algorithme de preuve, sur base d'hypothèses.

\part{Logique formelle}
\chapter{Contexte: la méthode scientifique}

\section{Formalisation d'un système}

Comment pouvons-nous formaliser un système dans le monde réel tels que les champs magnétiques ou la gravitation ?

\begin{center}
\includegraphics[scale=0.65]{images/Abstraction.pdf}
\end{center}

Afin de formaliser un système dans le monde réel, nous devons faire une abstraction vers un modèle théorique. Ce modèle théorique, aussi appelé théorie, est un ensemble de phrases logiques dont il est possible de tirer des prédictions en utilisant le raisonnement déductif. Il n'est intéressant que s'il se comporte comme le vrai système.

Un exemple de cette formalisation pourrait être les équations de Maxwell qui sont le modèle théorique correspondant pour l'électromagnétisme.

\section{Boucle de raisonnement}

Il existe trois formes de raisonnement: la déduction, l'induction et l'abduction. Ces trois formes de raisonnement peuvent être liées dans une boucle de raisonnement de la façon suivante:

\begin{center}
\includegraphics[scale=0.50]{images/BoucleRaisonnement1.pdf}
\end{center}

\subsection{Déduction}

Il s'agit de faire des calculs et des raisonnements logiques par rapport à une théorie. 
Avec ces raisonnements, on déduit le résultat qu'une expérience donnerait selon la théorie.
Par exemple, en utilisant les équations de Maxwell on peut déduire le trajectoire d'un
objet avec une charge électrique dans un champ électromagnétique.  \\

\subsection{Induction}

L'induction est le fait de trouver une règle générale à partir des expériences répétées.
On choisit en général une règle moyenne qui deviendra la règle générale.  Il faut souligner que les résultats expérimentaux ne sont pas totalement fiables ou complets. Dès lors, la règle trouvée n'est pas nécessairement exacte.  Par exemple, si par induction nous avons trouvé la règle, "les oiseaux volent", cela est vrai tant que l'on n’a pas vu un pingouin. Autre exemple, nous pouvons supposer que demain le soleil va se lever comme depuis des milliers d'années, même si rien ne l'assure.\\

\subsection{Abduction}

On compare la règle générale trouvée lors de l'induction avec la théorie.
S'il y a une incohérence entre la règle générale et la théorie qui ne rentre pas dans la marge d'erreur expérimentale, on suppose qu'il y a une erreur dans la théorie.
Il faut alors corriger la théorie existante ou en inventer/deviner une nouvelle. 
Ce type de raisonnement s'appelle l'abduction:
trouver une {\em explication} (= la théorie corrigée) pour une règle ou un fait.
On applique l'abduction couramment dans la vie de tous les jours; par exemple, lorsqu'un élève entre trempé dans la classe, nous supposons qu'il pleut dehors.
La pluie est une explication possible pour l'état de l'élève. \\

\subsection{Conclusion}

Sur ces trois formes de raisonnement, seule la déduction est un raisonnement sûr.
Les deux autres, l'induction et l'abduction, peuvent donner des erreurs.
Malgré cela, les trois formes sont tout aussi importantes.
Par example, il faut les trois pour expliquer comment marche la méthode scientifique.
Dans l'état actuel de la science du raisonnement,
nous comprenons beaucoup mieux la déduction que l'induction et l'abduction.
Toute la logique mathématique est une approfondissement de la science de la déduction.
Nous nous focaliserons dans ce cours uniquement sur la déduction. \\

\section{Exemples}

\subsection{Loi de Maxwell}

Nous illustrons dès à présent le fonctionnement de la boucle de raisonnement à l'aide de l'exemple cité plus haut, c'est-à-dire les équations de Maxwell :

\begin{center}
\includegraphics[scale=0.50]{images/BoucleRaisonnement2.pdf}
\end{center}

Par déduction, grâce à la théorie et aux conditions initiales que nous fixons, nous calculons
la trajectoire d'un électron. Nous effectuons ensuite des mesures dans le monde réel. Nous allons, par exemple, mesurer la trajectoire plusieurs fois avec des méthodes différentes et, par induction, nous trouvons une règle qui est la loi de comportement de la particule.
Nous comparons ensuite cette règle à la théorie, et nous la corrigeons si besoin.
La création d'une théorie qui explique la règle trouvée est une abduction.
\\



\subsection{Sac de billes}

Afin d'illustrer les 3 formes de raisonnements de manière plus formelle, considérons un sac de billes pouvant contenir des billes noires ou blanches. Notons que sac(x) signifie "la bille x est dans le sac" et que blanc(x) signifie "la bille x est blanche". \\

\subsubsection{Déduction}

\begin{enumerate}
  \item Règle: $\forall$ $x$, $sac(x)$ $\Rightarrow$ $blanc(x)$
  \item Cas: $sac(a)$, $sac(b)$, $\cdots$\\
  \rule{5.5cm}{.1pt} 
  \item Résultat: $blanc(a)$, $blanc(b)$, $\cdots$
\end{enumerate}

Si toutes les billes se trouvant dans le sac sont blanches et que l'on pioche une bille de ce sac, cette bille sera blanche. Cette déduction est forcément correcte.

\subsubsection{Induction}

\begin{enumerate}
  \item Cas: $sac(a)$, $sac(b)$,$\cdots$
  \item Résultat: $blanc(a)$, $blanc(b)$, $\cdots$\\
  \rule{5.5cm}{.1pt}	
  \item Règle: $\forall$ $x$, $sac(x)$ $\Rightarrow$ $blanc(x)$
\end{enumerate}

Si toutes les billes que l'on pioche du sac sont blanches, alors nous pouvons établir comme règle que toutes les billes dans le sac sont blanches. Cette induction n'est pas forcément correcte.

\subsubsection{Abduction}

\begin{enumerate}
  \item Règle: $\forall$ $x$, $sac(x)$ $\Rightarrow$ $blanc(x)$
  \item Résultat: $blanc(a)$, $blanc(b)$, $\cdots$\\
  \rule{5.5cm}{.1pt}
  \item Cas: $sac(a)$, $sac(b)$, $\cdots$
\end{enumerate}

Si toutes les billes se trouvant dans le sac sont blanches et que nous trouvons des billes blanches à côté du sac, nous pouvons penser qu'elles viennent du sac.
L'explication pour la couleur des billes trouvées est qu'elles viennent du sac.
Cette abduction n'est pas forcément correcte.

  
\chapter{La logique propositionnelle}

\begin{quote}
« \textit{\foreignlanguage{english}{They don't even need to know what they're
talking about.}} » --- Richard Feynman à propos des mathématiciens.
\end{quote}

La logique propositionnelle est la plus simple des formes de logique. Elle
permet de formaliser des expressions telles que les suivantes :

\begin{enumerate}
\item « S’il fait beau, alors je vais dehors. »
\item « Cet homme est grand et fort. »
\item « Il fait jour mais pas nuit. »
\end{enumerate}

Plus précisément, nous partons d’un ensemble de propositions
premières. Par exemple :

\begin{enumerate}
\item il fait beau ;
\item je vais dehors ;
\item cet homme est grand ;
\item cet homme est beau ;
\item il fait jour ;
\item il fait nuit.
\end{enumerate}

Le contenu exact de ces propositions n’a en fait aucune importance. C’est
pourquoi elles seront remplacées par des lettres majuscules :

\begin{enumerate}
\item A = « il fait beau » ;
\item B = « je vais dehors » ;
\item C = « cet homme est grand » ;
\item D = « cet homme est beau » ;
\item E = « il fait jour » ;
\item F = « il fait nuit ».
\end{enumerate}

Une proposition logique est alors :

\begin{itemize}
\item soit une des propositions premières ;
\item soit une combinaison de propositions logiques connectées par des
  connecteurs logiques.
\end{itemize}

Ainsi, les exemples de propositions précédentes peuvent être réécrites
comme ceci (la signification précise des différents symboles sera décrite
plus loin) :

\begin{enumerate}
\item $A \Rightarrow B$ ;
\item $C \land D$ ;
\item $E \land \lnot F$.
\end{enumerate}

% TODO: Plus d’exemples des différents connecteurs pour tout illustrer ? % On ne trouve pas nécessaire dans la mesure ou les trois exemple sont représentatifs - Cyril et Jon

L’avantage de cette notation par rapport aux phrases en français est qu’elle
nous permet d’effectuer des raisonnements formels sur les propositions
logiques. En particulier, nous pouvons précisément définir :

\begin{enumerate}
\item une \textbf{grammaire} qui définit ce qui est une proposition logique et ce qui
  ne l’est pas ;
\item une \textbf{sémantique} qui donne un sens à chaque proposition logique ;
\item une \textbf{théorie de preuve} permettant, en sachant qu’une proposition est
  vraie, de trouver d’autres propositions vraies (par exemple à partir de $A
  \Rightarrow B$ on peut trouver $\lnot B \Rightarrow \lnot A$).
\end{enumerate}
\section{La grammaire}

La logique propositionnelle est un \textbf{langage} formel. Il peut être défini
à l’aide d’une grammaire sur un \textbf{alphabet}. L’alphabet est l’ensemble des
symboles qui composent une proposition logique, c’est-à-dire :

\begin{itemize}
\item les lettres majuscules représentant les différentes propositions
  premières : $A$, $B$, $C$, etc. ;
\item $\true$ et $\false$ qui représentent des propositions qui sont
  respectivement toujours vraies et toujours fausses ;
\item les différents connecteurs logiques :
	\begin{itemize}
	\item Conjonction : $\land$
	\item Disjonction : $\lor$
	\item Négation :  $\lnot$
	\item Implication :  $\Rightarrow$
	\item Equivalence :  $\Leftrightarrow$
	\end{itemize}
\item les caractères de ponctuation « $($ » et « $)$ ».
\end{itemize}

Cependant, toutes les séquences composées de ces caractères ne sont pas des
phrases propositionnelles. La grammaire suivante permet de donner les règles que
les phrases propositionnelles doivent respecter~:

\begin{tabular}{rl}
  $\textrm{<identificateur>}$ & = $A$ | $B$ | $C$ | $D$ | \dots \\
  $\textrm{<proposition>}$
  & = $\true$ \\
  & | $\false$ \\
  & | $\textrm{<identificateur>}$ \\
  & | $(\textrm{<proposition>})$ \\
  & | $\lnot \textrm{<proposition>}$ \\
  & | $\textrm{<proposition>} \land \textrm{<proposition>}$ \\
  & | $\textrm{<proposition>} \lor \textrm{<proposition>}$ \\
  & | $\textrm{<proposition>} \Rightarrow \textrm{<proposition>}$ \\
  & | $\textrm{<proposition>} \Leftrightarrow \textrm{<proposition>}$
\end{tabular}
\vspace{2 mm}

Remarquez que seules les séquences de symboles qui respectent cette
grammaire sont des phrases propositionnelles. Ainsi, $p \Leftrightarrow q$
n’est pas une phrase propositionnelle parce que les propositions
premières sont des lettres majuscules ; de même les phrases en français —
ou en klingon, ou encore dans d’autres formalismes mathématiques – comme «
  s’il fait beau alors je vais dehors » ne respectent pas la grammaire
précédente et ne sont donc pas des phrases propositionnelles.

Notez aussi que le discours à propos des phrases propositionnelles n’est
pas une phrase propositionnelle. La grammaire précédente, la description de
l’alphabet, et cette explication parlent de propositions logiques sans en
être, et font donc partie de ce qui est appelé le \textbf{métalangage}.

\section{Les tables de vérité}

La grammaire permet de définir précisément l’ensemble des phrases
propositionnelles, mais pas de leur donner un sens, c’est à dire de définir
quand une proposition est vraie ou fausse.

Pour cela, il faut décider lesquelles des propositions premières sont vraies
et lesquelles sont fausses. Rappelez-vous que la signification des
propositions premières n’a aucune importance, ce choix est donc complètement
arbitraire (le choix qui décrit au mieux le monde réel n’est qu’un des choix
possibles parmi tous les autres). On sait également que les propositions
$\true$ et $\false$ sont, respectivement, toujours vraies et toujours
fausses.

Les autres propositions sont construites à partir de propositions plus
simples. Leur véracité est fonction de celle des propositions qui les
composent. Prenons par exemple $p \land q$, où $p$ et $q$ sont d’autres
propositions. La véracité de $p \land q$ est une fonction de celle de $p$ et
de $q$ : $p \land q$ est vrai si et seulement si $p$ et $q$ sont vrais aussi
($\land$ est un « et » logique). Cette relation peut être exprimée à l’aide
de la table de vérité suivante :

\vspace{2 mm}
\begin{tabular}{ll|l}
  $p$ & $q$ & $p \land q$ \\
  \hline

  $\true$ & $\true$ & $\true$ \\
  $\true$ & $\false$ & $\false$ \\
  $\false$ & $\true$ & $\false$ \\
  $\false$ & $\false$ & $\false$
\end{tabular}

\vspace{2 mm}
Voici la table de vérité des autres connecteurs logiques :
\vspace{2 mm}

\begin{tabular}{ll|l}
  $p$ & $q$ & $p \lor q$ \\
  \hline

  $\true$ & $\true$ & $\true$ \\
  $\true$ & $\false$ & $\true$ \\
  $\false$ & $\true$ & $\true$ \\
  $\false$ & $\false$ & $\false$
\end{tabular}
\vspace{2 mm}

\begin{tabular}{ll|l}
  $p$ & $q$ & $p \Leftrightarrow q$ \\
  \hline

  $\true$ & $\true$ & $\true$ \\
  $\true$ & $\false$ & $\false$ \\
  $\false$ & $\true$ & $\false$ \\
  $\false$ & $\false$ & $\true$
\end{tabular}
\vspace{2 mm}

\begin{tabular}{l|l}
  $p$ & $\lnot p$ \\
  \hline

  $\true$ & $\false$ \\
  $\false$ & $\true$
\end{tabular}
\vspace{2 mm}

\begin{tabular}{ll|l}
  $p$ & $q$ & $p \Rightarrow q$ \\
  \hline

  $\true$ & $\true$ & $\true$ \\
  $\true$ & $\false$ & $\false$ \\
  $\false$ & $\true$ & $\true$ \\
  $\false$ & $\false$ & $\true$
\end{tabular}
\vspace{2 mm}

Remarquez que le dernier tableau est correct. Il n’est parfois pas intuitif
que la proposition $A \Rightarrow B$ (qui pourrait s’exprimer en français
par « si $A$, alors $B$ ») soit toujours vraie quand $A$ est faux, mais c’est
pourtant le cas : la proposition dit que quand $A$ est vrai, $B$ doit
l’être aussi, mais elle ne donne aucune information sur le cas où $A$ est
faux.

Les parenthèses, quant à elles, servent à distinguer des propositions telles
que $A \land (B \lor C)$ et $(A \land B) \lor C$, qui s’écriraient de la
même manière sans parenthèses alors qu’elles n’ont pas la même table de
vérité :

\begin{tabular}{lll|ll}
  $A$ & $B$ & $C$ & $A \land (B \lor C)$ & $(A \land B) \lor C$ \\
  \hline

  $\true$  & $\true$  & $\true$  & $\true$  & $\true$  \\
  $\true$  & $\true$  & $\false$ & $\true$  & $\true$  \\
  $\true$  & $\false$ & $\true$  & $\true$  & $\true$  \\
  $\true$  & $\false$ & $\false$ & $\false$ & $\false$ \\
  $\false$ & $\true$  & $\true$  & $\false$ & $\true$  \\
  $\false$ & $\true$  & $\false$ & $\false$ & $\false$ \\
  $\false$ & $\false$ & $\true$  & $\false$ & $\true$  \\
  $\false$ & $\false$ & $\false$ & $\false$ & $\false$
\end{tabular}

\section{Les interprétations}

Une autre façon plus puissante de définir si une proposition est vraie ou fausse est d’utiliser
une \textbf{interprétation}. Si $E_P$ est l’ensemble des propositions premières,
alors une interprétation $I$ définit la fonction $\val_I : E_P \rightarrow
\{\true, \false\}$ \footnote{La notation $f : A \rightarrow B$ signifie que $f$
  est une fonction depuis l’ensemble $A$ vers l’ensemble $B$.  } qui permet de
savoir si ces propositions premières sont vraies ou fausses. Par exemple, on
pourrait écrire ceci :

\[\val_I(A) = \true\]
\[\val_I(B) = \false\]
\[\val_I(C) = \true\]

Étant donné la fonction $\val_I$, il est possible de définir la fonction
$\VAL_I : P \rightarrow \{\true, \false\}$, qui est une extension de
$\val_I$ à $P$, l’ensemble de toutes les propositions. L’équivalent des
tables de vérité pourrait être des expressions telles que~:

\[\forall p \in E_P. \VAL_I(p) = \val_I(p)\]

% Les définitions du type « A \land B est vrai si A et B sont vrais » me
% paraissent être cycliques, donc je les ai plutôt écrites comme ça, mais je ne
% sais pas si cela paraît évident pour tout le monde.
\[\VAL_I(p \land q) = \begin{cases}
  \true \mbox{ si } \VAL_I(p) \mbox{ et } \VAL_I(q) = \true \\
  \false\mbox{ sinon}
\end{cases}\]

\[\VAL_I(p \lor q) = \begin{cases}
  \false \mbox{ si } \VAL_I(p) \mbox{ ou } \VAL_I(q) = \false \\
  \true\mbox{ sinon}
\end{cases}\]

\[\VAL_I(\lnot p) = \begin{cases}
  \false\mbox{ si }\VAL_I(p) = \true \\
  \true\mbox{ si }\VAL_I(p) = \false
\end{cases}\]

\[\VAL_I(p \Leftrightarrow q) = \begin{cases}
  \true\mbox{ si }\VAL_I(p) = \VAL_I(q) \\
  \false\mbox{ sinon}
\end{cases}\]

\[\VAL_I(p \Rightarrow q) = \begin{cases}
  \false \mbox{ si } \VAL_I(q) = \false \mbox{ alors que } \VAL_I(p) = \true \\
  \true\mbox{ sinon}
\end{cases}\]

Prenons un exemple concret, utilisons une interprétation pour étudier la phrase
suivante~: « S'il fait beau à midi, j'irai promener le chien ». Nous devons
d’abord traduire cette phrase en l’une des proposition du formalisme que nous
avons défini, en commençant par identifier les propositions premières~:

\begin{enumerate}
\item B = « Il fait beau »~;
\item M = « Il est midi »~;
\item P = « Je vais promener le chien »~;
\end{enumerate}

Identifions également les connecteurs à employer~: « s’il fait beau $\land$
  qu'il est midi $\Rightarrow$ j’irai promener le chien ». En combinant ces deux
résultats, nous obtenons la phrase propositionnelle $(B \land M) \Rightarrow P$.

Interprétons désormais notre proposition à l’aide de l’interprétation suivante~:
\begin{enumerate}
  \item Il fait beau~: $\val_I(B) = \true$~;
  \item Il est midi~: $\val_I(M) = \true$~;
  \item Je n’irai pas promener le chien~: $\val_I(P) = \false$.
\end{enumerate}

Nous pouvons alors effectuer le développement suivant~:

\begin{align*}
  \VAL_I((B \land M) \Rightarrow P) &= \begin{cases}
    \false \mbox{ si } \VAL_I(P) = \false \mbox{ alors que } \VAL_I(B \land M) = \true \\
    \true\mbox{ sinon}
  \end{cases} \\
  \mbox{ or }\VAL_I(B \land M) &= \begin{cases}
    \true \mbox{ si } \VAL_I(B) \mbox{ et } \VAL_I(M) = \true \\
    \false\mbox{ sinon}
  \end{cases} \\
  &= \true \\
  \mbox{ donc }  \VAL_I((B \land M) \Rightarrow P) &= \begin{cases}
    \false \mbox{ si } \VAL_I(P) = \false \\
    \true\mbox{ sinon}
  \end{cases} \\
  &= \false
\end{align*}

Nous pouvons donc conclure que, dans cette interprétation, la personne ayant
fait cette affirmation a menti. Notons néanmoins que notre homme n'aurait pas
mentit en partant promener le chien alors qu'il pleuvait à midi, rien n'ayant
été dit sur ce qu'il ferait dans le cas où il ne ferait pas beau.

% TODO: Rajouter un exemple simple pour vérifier que l’un des connecteurs est
% bien défini (\land, etc.) et aussi montrer la démarche :
%   - avoir une phrase en français avec 2-3 connecteurs logiques ;
%   - identifier les propositions primaires et les connecteurs ;
%   - réécrire la phrase mathématiquement ;
%   - définir une interprétation ;
%   - l’utiliser pour évaluer la fonction.

\section{Les modèles logiques}

À partir de la notion d’interprétation, nous pouvons définir ce qu’est
un \textbf{modèle}. Soit $B = \{b_1, b_2, \dots, b_n \}$ un ensemble de
propositions logiques. Une interprétation $I$ est un modèle de $B$ si et
seulement si $\forall b_i \in B. \VAL_I(b_i) = \true$. Autrement dit, $I$
décrit un univers qui respecte toutes les règles se trouvant dans l’ensemble
$B$.

Donc, dans notre exemple précédent, l’interprétation choisie n’est pas un modèle
de la proposition analysée, celle-ci n’étant pas validée, mais par exemple
l’interprétation telle que $\val_I(B) = \true$, $\val_I(M) = \true$ et
$\val_I(P) = \true$ est bien un modèle de la proposition utilisée comme
exemple. Remarquez aussi que cela ne change rien au fait que l’interprétation
choisie est le modèle d’autres propositions que celle étudiée (par exemple $B
\land M$).

% TODO: Exemple d’expression avec une interprétation qui est un modèle et une
% autre qui n’en est pas un.
\begin{description}\item[Les tautologies :] Pour certaines propositions, toute interprétation est un modèle, c’est à dire que ces propositions sont toujours
  vraies. Par exemple, $\true$ est évidemment une tautologie, de même que $A
  \Rightarrow A$ ou encore $A \lor \lnot A$. Le fait qu’une proposition $p$ est
  une tautologie se note $\vDash p$.
\item[Les contradictions :] Pour d’autres propositions, il n’existe aucun
  modèle, c’est à dire qu’elles sont toujours fausses. Par exemple $A \land
  \lnot A$ est une contradiction. Le fait qu’une proposition $p$ est une
  contradiction se note $\nvDash p$.
\item[Les contingences :] Toutes les autres propositions sont des contigences. Il
  existe des interprétations qui sont des modèles et d’autres qui n’en sont
  pas. Par exemple $A \land B$ est vrai pour l’interprétation $I$ telle que
  $\val_I(A) = \val_I(B) = \true$, mais faux dans tous les autres cas.
\end{description}

		\subsection{Conséquence logique}
			$p$ est conséquence logique de $q$ si et seulement si $p \Rightarrow q$ est une tautologie. En d'autres termes, si
			\begin{center}
			\begin{tabular}{ll}
			$p \models q$ & $q$ est valide dans tous les modèles de $p$ \\
			&\\
			alors & \\
			$\models (p \Rightarrow q)$ & $p \Rightarrow q$ est une tautologie.\\
			&\\
			On peut donc écrire & \\
			$p \Rrightarrow q$ & $p$ est conséquence logique de $q$.\\
			\end{tabular}
			\end{center}
			Cependant, la conséquence logique ($\Rrightarrow$) n'est pas une proposition logique (cf. syntaxe d'une proposition).
		
		\subsection{Équivalence logique}
			Par le raisonnement ci-dessus, on peut dire que $p$ est logiquement équivalent à $q$ si et seulement si
			\begin{center}
			\begin{tabular}{ll}
			$p \models q$ & $q$ est valide dans tous les modèles de $p$ \\
			$q \models p$ & $p$ est valide dans tous les modèles de $q$ \\
			&\\
			et donc & \\
			$\models (p \Rightarrow q)$ & $p \Rightarrow q$ est une tautologie et\\
			$\models (p \Rightarrow q)$ & $p \Rightarrow q$ est une tautologie.\\
			&\\
			On peut donc écrire & \\
			%impossible de trouver l'équivalence logique en symbole
			$p \Lleftarrow \Rrightarrow q$ & $p$ est conséquence logique de $q$.\\ 
			\end{tabular}
			\end{center}
			L'équivalence logique n'est pas non plus une proposition logique.\\
			
			Il ne faut pas non plus oublier la différence entre phrase propositionnelle ($p$, $q$, $s$,...) et propositions primaires ($P$, $Q$, $S$,...) (cf. syntaxe d'une proposition) :
			\begin{center}
			\begin{tabular}{ll}
				$p \Rightarrow q$ & n'est pas une proposition\\
				
				mais &\\
				$P \land Q \Rightarrow R \land \lnot S$ & en est bien une.\\
			\end{tabular}
			\end{center}
	

  % \documentclass[10pt,a4paper]{article}
% \usepackage[utf8]{inputenc}
% \usepackage{amsmath}
% \usepackage{amsfonts}
% \usepackage{amssymb}
% \usepackage{array}
% \begin{document}
%% 	\chapter{Rappel}
% 		\subsection{Conséquence logique}
% 			$p$ est conséquence logique de $q$ si et seulement si $p \Rightarrow q$ est une tautologie. En d'autres termes, si
% 			\begin{center}
% 			\begin{tabular}{ll}
% 			$p \models q$ & $q$ est valide dans tous les modèles de $p$ \\
% 			&\\
% 			alors & \\
% 			$\models (p \Rightarrow q)$ & $p \Rightarrow q$ est une tautologie.\\
% 			&\\
% 			On peut donc écrire & \\
% 			$p \Rrightarrow q$ & $p$ est conséquence logique de $q$.\\
% 			\end{tabular}
% 			\end{center}
% 			Cependant, la conséquence logique ($\Rrightarrow$) n'est pas une proposition logique (cf. syntaxe d'une proposition).
% 		
% 		\subsection{Équivalence logique}
% 			Par le raisonnement ci-dessus, on peut dire que $p$ est logiquement équivalent à $q$ si et seulement si
% 			\begin{center}
% 			\begin{tabular}{ll}
% 			$p \models q$ & $q$ est valide dans tous les modèles de $p$ \\
% 			$q \models p$ & $p$ est valide dans tous les modèles de $q$ \\
% 			&\\
% 			et donc & \\
% 			$\models (p \Rightarrow q)$ & $p \Rightarrow q$ est une tautologie et\\
% 			$\models (p \Rightarrow q)$ & $p \Rightarrow q$ est une tautologie.\\
% 			&\\
% 			On peut donc écrire & \\
% 			%impossible de trouver l'équivalence logique en symbole
% 			$p \Lleftarrow \Rrightarrow q$ & $p$ est conséquence logique de $q$.\\ 
% 			\end{tabular}
% 			\end{center}
% 			L'équivalence logique n'est pas non plus une proposition logique.\\
% 			
% 			Il ne faut pas non plus oublier la différence entre phrase propositionnelle ($p$, $q$, $s$,...) et propositions primaires ($P$, $Q$, $S$,...) (cf. syntaxe d'une proposition) :
% 			\begin{center}
% 			\begin{tabular}{ll}
% 				$p \Rightarrow q$ & n'est pas une proposition\\
% 				
% 				mais &\\
% 				$P \land Q \Rightarrow R \land \lnot S$ & en est bien une.\\
% 			\end{tabular}
% 			\end{center}
% 	
	\chapter{Preuves en logique propositionnelle}
		Une preuve est un raisonnement déductif qui démontre si une proposition
est vraie ou fausse.
On distingue des preuves informelles et des preuves formelles.
Une preuve informelle est un raisonnement en langage naturel, parfois augmenté avec des
notations mathématiques.
Une preuve formelle est un objet mathématique qui formalise le raisonnement déductif.
Un des buts de la logique mathématique est de prouver le plus possibles des résultats
mathématiques avec des preuves formelles.

Au 20ème siècle les mathématiciens sont arrivés à prouver la plupart des mathématiques
classiques (telles qu'utilisées par des ingénieurs) avec des preuves formelles.
Un des résultats les plus célèbres est la preuve formelle du théorème des quatre couleurs,
fait par Georges Gonthier et Benjamin Werner avec l'assistant de preuve Coq (un logiciel
qui automatise la plupart des manipulations formelles nécessaires).
Ce théorème dit que toute carte découpée en régions connexes peut être colorée avec seulement
quatre couleurs, de sorte que deux régions adjacentes ont toujours des couleurs distinctes.

Dans ce chapitre nous allons définir des preuves formelles pour la logique propositionnelle.
Nous présenterons trois approches:
		\begin{itemize}
			\item Table de vérité
			\item Preuve transformationnelle
			\item Preuve déductive (la plus générale)		
		\end{itemize}

		\section{Preuve avec table de vérité}
La preuve formelle la plus simple est une table de vérité.
			\newcolumntype{x}{>{\itshape\bfseries}c}
		Prouvons que $\lnot (P \land Q) \Leftrightarrow (\lnot P \lor \lnot Q)$ est vrai :
			\begin{center}
			\begin{tabular}{cc|ccxcx}
			$P$ & $Q$ & $\lnot P$ & $\lnot Q$ & $(\lnot P \lor \lnot Q)$ & $P \land Q$ & $ (\lnot P \lor \lnot Q)$\\
			\hline
			F&F&T&T&T&F&T\\
			T&F&F&T&T&F&T\\
			F&T&T&F&T&F&T\\
			T&T&F&F&F&T&F\\
			\end{tabular}
			\end{center}

On peut constater que le vecteur de vérité de $\lnot (P \land Q)$ est équivalent à celui de $(\lnot P \lor  \lnot Q)$. La preuve a donc vérifié la véracité de la proposition.
Notez qu'une table de vérité est un objet mathématique en métalangage
parce qu'elle n'est pas une proposition.

L'inconvénient de cette méthode de preuve est qu'elle devient rapidement très lourde quand le nombre de propositions premières augmente. Il faut en effet $2^n$ lignes dans la table pour $n$ propositions. 

\section{Preuve transformationnelle}

Une preuve transformationnelle est une séquence de transformations
$p_1 \Lleftarrow \Rrightarrow p_2 \Lleftarrow \Rrightarrow \cdots \Lleftarrow \Rrightarrow p_n$,
dans laquelle on a toujours $p_i \Lleftarrow \Rrightarrow p_{i+1}$
(équivalence logique entre éléments adjacents dans la séquence).
Une preuve transformationnelle est aussi un objet mathématique en métalangage.
Pour faciliter la création d'une preuve transformationnelle,
on utilise des "Lois", c'est-à-dire des équivalences connues.
			\begin{center}
			\begin{tabular}{|ll|}
			\hline
			$p \Lleftarrow \Rrightarrow p \lor p$ & Idempotence\\
			$p \lor q \Lleftarrow \Rrightarrow q \lor p$ & Commutativité\\
			$(p \lor q) \lor r \Lleftarrow \Rrightarrow p \lor (q \lor r)$ & Associativité\\
			$ \lnot \lnot p \Lleftarrow \Rrightarrow p$ & Double Négation\\
			$p \Rightarrow q \Lleftarrow \Rrightarrow \lnot p \lor q$ & Implication\\
			$\lnot (p \land q) \Lleftarrow \Rrightarrow \lnot p \lor \lnot q$ & $1^{ere}$ loi de De Morgan\\
			$p \Leftrightarrow q \Lleftarrow \Rrightarrow (p \Rightarrow q) \land (q \Rightarrow p)$ & Équivalence\\
			\hline
			\end{tabular}
			\end{center}
On ajoute deux règles supplémentaires : la transitivité et la substitution.
			\subsection*{Transitivité de l'équivalence}
			\indent Si $p \Lleftarrow \Rrightarrow q$ et $q \Lleftarrow \Rrightarrow r$, alors $p \Lleftarrow \Rrightarrow r$.
			\subsection*{Substitution}
			Il est autorisé de remplacer une formule par une formule équivalente à l’intérieur d’une autre formule. Autrement dit : \\
			\indent Soit p,q,r des formules propositionnelles.\\
			\indent Si $p \Leftrightarrow q$ et $r(p)$, alors $r(p) \Lleftarrow \Rrightarrow r(q)$.\\
			On peut remplacer $p$ par $q$ car elles sont équivalentes. 
			
			
			\subsection*{Exemple}
			On veut prouver : $p \land (q \land r) \Lleftarrow \Rrightarrow (p \land q) \land r$
			\begin{center}
			\begin{tabular}{ll}
			
			$p \land (q \land r)$ & $\Lleftarrow \Rrightarrow p \land \lnot \lnot (q \land r)$\\
			& $\Lleftarrow \Rrightarrow p \land \lnot (\lnot q \lor \lnot r)$\\
			& $\Lleftarrow \Rrightarrow \lnot \lnot (p \land \lnot (\lnot q \lor \lnot r))$\\
			& $\Lleftarrow \Rrightarrow \lnot (\lnot p \lor \lnot \lnot (\lnot q \lor \lnot r))$\\
			& $\Lleftarrow \Rrightarrow \lnot (\lnot p \lor (\lnot q \lor \lnot r))$\\
			& $\Lleftarrow \Rrightarrow \lnot ((\lnot p \lor \lnot q) \lor \lnot r)$\\
			&$\vdots$\\
			& effectuer les mêmes lois dans le sens contraire \\
			&$\vdots$\\
			& $\Lleftarrow \Rrightarrow (p \land q) \land r$\\
			\end{tabular}
			\end{center}
Le problème de cette méthode de preuve est qu'elle requiert de l'intuition, de la créativité. Elle n'est donc pas forcément plus efficace que les tables de vérité, surtout si "l'astuce" est difficile à trouver.
		
\section{Preuve déductive}

Une preuve déductive est un objet mathématique qui formalise une séquence
de pas de raisonnement simples.
Chaque pas doit être justifié avec le nom de la règle ou la loi qui est utilisée.
Les pas utilisent trois techniques de raisonnement différentes:
les équivalences logiques, les règles d'inférence et les schémas de preuve.
Avec ces techniques,
une preuve déductive est beaucoup plus expressive qu'une preuve transformationnelle.

\subsection{Equivalences logiques}
			\begin{center}
			\begin{tabular}{ll}
			$p \Lleftarrow \Rrightarrow p \lor p$ & Idempotence de $\lor$\\
			$p \lor q \Lleftarrow \Rrightarrow q \lor p$ & Commutativité de $\lor$\\
			$(p \lor q) \lor r \Lleftarrow \Rrightarrow p \lor (q \lor r)$ & Associativité de $\lor$\\
			$ \lnot \lnot p \Lleftarrow \Rrightarrow p$ & Double Négation\\
			$p \Rightarrow q \Lleftarrow \Rrightarrow \lnot p \lor q$ & Implication\\
			$\lnot (p \land q) \Lleftarrow \Rrightarrow \lnot p \lor \lnot q$ & $1^{ere}$ loi de De Morgan\\
			$\lnot (p \lor q) \Lleftarrow \Rrightarrow \lnot p \land \lnot q$ & $2^{eme}$ loi de De Morgan\\
			$(p \land q) \lor r \Lleftarrow \Rrightarrow (p \lor r) \land (q \lor r)$ & Distributivité de $\lor$\\
			\end{tabular}
			\end{center}
À ces équivalence nous ajoutons aussi l'idempotence, la commutativité, l'associativité et
la distributivité de $\land$.\\
			
\subsection{Règles d'inférence}
		
À la différence de la preuve transformationnelle, les règles d'inférences ont une direction : elles commencent par les prémisses et se terminent par la conclusion.
Pour chaque règle, si les prémisses sont vraies, alors la conclusion est vraie.
Nous utilisons un raisonnement informel pour justifier chaque règle.
		\begin{center}
		\begin{tabular}{llp{3cm}l}
		
   			Conjonction : &
   			\begin{tabular}{cl}
      		p & prémisse\\
      		q & prémisse\\
      		\line(1,0){25}&\\
      		$p\land q$ & Conclusion
   			\end{tabular} 
   			&
   			
   			Simplification : &
   			\begin{tabular}{c}
      		$p \land q$ \\
      		\hline
      		$p$\\
   			\end{tabular} \\
   			&\\
   			
   			Addition : &
   			\begin{tabular}{c}
      		$p $ \\
      		\hline
      		$p \lor q$\\
   			\end{tabular}
   			&
   			
   			Contradiction : &
   			\begin{tabular}{c}
      		$p$ \\
      		$\lnot p$\\
      		\hline
      		$q$\\
   			\end{tabular}\\
   			&\\
   			
   			Double Négation : &
   			\begin{tabular}{c}
      		$\lnot \lnot p $ \\
      		\hline
      		$p$\\
   			\end{tabular}
   			&
   			
   			\raggedright Transitivité de l'équivalence : &
   			\begin{tabular}{c}
      		$p \Leftrightarrow q$ \\
      		$q \Leftrightarrow r$ \\
      		\hline
      		$p \Leftrightarrow r$\\
   			\end{tabular}\\
   			&\\
   			
   			Modus Ponens : &
   			\begin{tabular}{c}
      		$p \Rightarrow q$ \\
      		$p$ \\
      		\hline
      		$q$\\
   			\end{tabular}
   			&
   			
   			Modus Tollens : &
   			\begin{tabular}{c}
      		$p \Rightarrow q$ \\
      		$\lnot q$ \\
      		\hline
      		$\lnot p$\\
   			\end{tabular}\\
   			&\\
   			
   			Loi d'équivalence : &
   			\begin{tabular}{c}
      		$p \Leftrightarrow q$ \\
      		\hline
      		$q \Leftrightarrow p$\\
   			\end{tabular}\\		
		\end{tabular}
		\end{center}
		
\subsection{Schémas de preuve}

En plus des équivalences logiques et des règles d'inférence,
nous ajoutons deux schémas de preuve qui formalisent des
techniques de raisonnement plus abstraites:
le théorème de déduction et la démonstration par l'absurde.
Ces schémas donnent à l'approche de preuve déductive une grande expressivité,
beaucoup plus qu'une preuve transformationnelle.

\subsubsection*{Théorème de déduction (preuve conditionnelle)}

Pour prouver la proposition $s\Rightarrow t$, on suppose $s$ vrai.
La proposition $s$ s'ajoute donc aux prémisses utilisées dans la preuve.  
Ensuite, on fait une preuve de $t$:
on peut construire une preuve (objet mathématique) de $t$ en commençant de $s$.
On note ce théorème $s \vdash t$.
On écrit ce schéma un peu comme une règle d'inférence:
			\begin{center}
			\begin{tabular}{c}
      		$p,...,r,s \vdash t$ \\
      		\hline
      		$p,...,r \vdash s\Rightarrow t$\\
   			\end{tabular}\\
			\end{center}
On déduit $t$ et donc on sait que l'hypothèse $s\Rightarrow t$ est vraie et on l'évacue.
			
			\textbf{Remarque :} Il ne faut pas confondre les deux notations $p\models t$ et $p\vdash t$.
			\begin{itemize}
			\item $p\models t$ est une notion de vérité (tout modèle de $p$ est un modèle de $t$), et donc de sémantique;
			\item $ p\vdash t$ est une notion syntaxique (en commençant de $p$ on peut construire une preuve de $t$),
car une preuve est une séquence de manipulations syntaxiques.
			\end{itemize}
			
\subsubsection*{Démonstration par l'absurde (preuve par contradiction)}
		On suppose que les prémisses $p, ..., q$ n'ont pas de problème,
c'est-à-dire qu'on ne peut pas prouver une contradiction à partir de ces propositions.
Ensuite, on ajoute $r$ aux prémisses.
S'il est possible de prouver $s$ et aussi de prouver $\lnot s$, cela signifie qu'il y a une erreur dans les prémisses.
On suppose que c'est l'ajout $r$ qui est fautif.
On justifie qu'il n'y a aucune contradiction dans $p, ... ,q$ car on part du principe qu'il existe un modèle de $p,...q$.
On écrit ce schéma ainsi:
		\begin{center}
			\begin{tabular}{c}
      		$p,...,q,r \vdash s$ \\
      		$p,...,q,r \vdash \lnot s$\\
      		\hline
      		$p,...,q \vdash \lnot r$\\
   			\end{tabular}\\
			\end{center}
			
\section{Exemple de preuve déductive}

Nous donnons un premier exemple de preuve déductive.
Voici les propositions premières :
\begin{enumerate}
\item[A =] "tu manges bien"
\item[B =] "ton système digestif est en bonne santé"
\item[C =] "tu pratiques une activité physique régulière"
\item[D =] "tu es en bonne forme physique"
\item[E =] "tu vis longtemps"
\end{enumerate}
On peut maintenant établir une théorie, c'est-à-dire, un ensemble de propositions,
dont on espère qu'elle aura un modèle.

\paragraph{Théorie} 
\begin{enumerate}
\item $A \implies B$
\item $C \implies D$
\item $B \lor D \implies E$
\item $\lnot E$
\end{enumerate}

\paragraph{À prouver} $\lnot A \land \lnot C$

\paragraph{Preuve}
Voici la preuve déductive.  Nous la mettons dans un cadre pour souligner qu'elle est un objet mathématique.
Chaque ligne est où une prémisse, où un pas de raisonnement (une équivalence ou une règle d'inférence).
Pour chaque ligne il faut donner le nom de la règle qui est appliquée, cela s'appelle la {\em justification}
et c'est une partie importante de la preuve.
Les deux schémas se présentent avec des indentations; la partie indentée d'une preuve contient une prémisse
en plus ($s$ pour la preuve conditionnelle, $r$ pour la preuve indirecte).

\begin{tabular}{|l|l|}
\hline
1. A$\Rightarrow$B & prémisse \\
2. C$\Rightarrow$D & prémisse \\
3. B$\lor$D $\Rightarrow$E & prémisse \\
4. $\lnot$E & prémisse \\ 
\indent 5. A & hypothèse \\
\indent 6. B & modus ponens (1) \\
\indent 7. B$\lor$D & addition (6) \\
\indent 8. E & modus ponens (7) \\
9. $\lnot$A & preuve indirecte \\
\indent 10. C & hypothèse \\
\indent 11. D & modus ponens (2) \\
\indent 12. D$\lor$B & addition (11) \\
\indent 13. B$\lor$D & commutativité (12)\\
\indent 14. E & modus ponens (9) \\
15. $\lnot$C & preuve indirecte \\
16. $\lnot$A $\land$ $\lnot$C & conjonction (9,15) \\
\hline
\end{tabular}\\

Les quatre premières lignes introduisent les prémisses (les propositions de la théorie).
La ligne 5 commence une première preuve indirecte: on fait l'hypothèse A
et ensuite on déduit E (sur la ligne 8).  C'est une contradiction avec la prémisse $\lnot$E
et donc on vient de prouver $\lnot$A (sur la ligne 9).
La ligne 10 commence une deuxième preuve indirecte: on fait l'hypothèse C
et ensuite on déduit E (sur la ligne 14).  De nouveau, c'est une contradiction (avec la prémisse $\lnot$E)
et donc on vient de prouver $\lnot$C (sur la ligne 15).
Les justifications pour les lignes 9 et 15 sont {\em preuve indirecte}.

Avec cette preuve déductive,
nous avons pu prouver que tu ne manges pas bien et que tu ne pratiques pas d'activité physique régulière. 
% Maintenant nous allons automatiser les preuves, quand elles existent.
% Mais il faut savoir s'il peut tout résoudre ou pas. 
% Dans la section suivante
% on va donc construire un algorithme nous permettant de trouver automatiquement les preuves en logique propositionnelle.
% \end{document}

%% Partie 4 commence ici

% \section{Deux règles plus sophistiquées}
% \subsection{Théorème de déduction}
% 
% \begin{itemize}
% \item  Pour prouver s $\Rightarrow$ t
% \item  On suppose s vrai
% \item  On déduit t
% \item  Ensuite, on évacue l'hypothèse
% \end{itemize}
% 
% 
% \textit{Notation: p $\vdash$ t (on peut prouver t à partir de p) }
% 
% \subsubsection{Prémisse:}
% 
% \begin{equation}
% \frac{p,..., r, s \vdash t} 
% {p,..., r \vdash (s \Rightarrow t)}
% \end{equation}
% 
% \subsubsection{Conclusion:}
% 
% Déduire une implication
% 
% \subsection{Preuve par contradiction (ou preuve indirecte)}
% 
% On prend une hypothèse, et on peut prouver qu'elle est vraie ou fausse, d'où l'hypothèse n'est pas bonne.
% 
% \subsubsection{Prémisse:} 
% on suppose que p...q n'a pas de problème
% 
% \begin{equation}
% \begin{split}
% p,...,q, r, s \vdash s \\
% \frac{p,...,q, r, s \vdash \lnot s}
% {p,...,q \vdash \lnot r}
% \end{split}
% \end{equation}
% 
% \subsubsection{Conclusion:}
% 
% si p...q n'a pas de problème, on se focalise alors sur r

\section{Exemples de l'utilisation des schémas}

Pour illustrer l'utilisation des deux schémas, la preuve conditionnelle et la preuve par contradiction,
nous allons prouver la même conclusion en trois manières, avec chaque schéma et sans schéma.
\begin{itemize}
\item Prémisse: $(p \land q) \lor r$
\item Conclusion: $\lnot p \Rightarrow r$
\end{itemize}

\subsection{Exemple sans schéma}

\begin{tabular}{|l|l|}
\hline
1. $(p \land q) \lor r$ & \textit{Prémisse} \\
2. $r \lor (p \land q)$ & \textit{Commutativité en 1} \\
3. $(r \lor p) \land (r \lor q)$ & \textit{Associativité en 2}\\
4. $(r \lor p)$ & \textit{Simplification en 3}\\
5. $(p \lor r)$ & \textit{Commutativité en 4}\\
6. $\lnot \lnot p \lor r $ & \textit{Loi de la négation en 5}\\
7. $\lnot p \Rightarrow r $ & \textit{Implication en 6}\\
\hline
\end{tabular}

\subsection{Exemple de preuve conditionnelle}

\begin{tabular}{|l|l|}
\hline
1. $(p \land q) \lor r $ & \textit{Prémisse} \\
2. $\lnot \lnot(p \land q) \lor r $ & \textit{Double négation en 1} \\
3. $\lnot ( \lnot p \lor \lnot q) \lor r $ & \textit{Loi De Morgan en 2} \\
4. $\lnot p \lor \lnot q \Rightarrow r $ & \textit{Implication en 3}\\
\indent 5.  $\lnot p $ & \textit{Hypothèse}\\
\indent 6.  $\lnot p \lor \lnot q $& \textit{ Addition sur 5}\\
\indent 7.  $r$ & \textit{ Modus Ponens sur 4 et 6}\\
8.  $\lnot p \Rightarrow r $& \textit{Evacuation de l'hypothèse}\\
\hline
\end{tabular}

\subsection{Exemple de preuve par contradiction}

\begin{tabular}{|l|l|}
\hline
1. $(p \land q) \lor r $ & \textit{Prémisse}\\
2. $ (p  \lor r) \land (q \lor r)$ & \textit{Distributivité sur 1}\\
3. $(p \lor r)$ & \textit{Simplification en 2}\\

 \indent 4. $\lnot ( \lnot p \Rightarrow r)$ & \textit{Hypothèse}\\
 \indent 5. $\lnot ( \lnot \lnot p \lor r)$ &\textit{Implication en 4}\\
 \indent 6. $\lnot (p \lor r)$ & \textit{ Négation en 5}\\


7. $\lnot \lnot (\lnot p \Rightarrow r) $ & \textit{ Preuve par contradiction}\\
8. $\lnot p \Rightarrow r $ & \textit{Négation en 7}\\
\hline
\end{tabular}


\section{Quelques concepts supplémentaires}
\subsection{Principe de dualité }

\subsubsection{Dans les formules sans $\rightarrow$ :}

\begin{align*}
\land \leftrightarrow \lor \\ 
\true \leftrightarrow \false 
\end{align*}
\begin{align*}
\models \lnot ( p \land q)  \Leftrightarrow \lnot p \lor \lnot q \\
\models \lnot ( p \lor q)  \Leftrightarrow \lnot p \land \lnot q 
\end{align*}

\subsubsection{Formule quelconque:}

\begin{align*}
\land \leftrightarrow \lor \\ 
\true \leftrightarrow \false \\ 
p \leftrightarrow \lnot p 
\end{align*}

 Justification en raisonnant sur les modèles:
 \begin{align*}
	 (p_1,...,p_n) \models q \hspace{1cm} ssi \models (p_1 \land ... \land p_n \land q) \leftrightarrow false \\
	 ssi \models ( \lnot p_1 \lor ... \lor \lnot p_n \lor q) \leftrightarrow \true
 \end{align*}



% \subsection{Algorithme de normalisation}
% 
% Toute formule peut être transformée en une formule équivalente, la forme normale,
% qui a toujours la même forme.
% La forme normale facilite les manipulations des formules par des algorithmes.
% Nous allons introduire une forme normale que nous allons utiliser pour l'algorithme de preuve.
% 
% \subsubsection{Forme normale}
% 
% Il y a deux formes normales qui sont souvent utilisées:
% la forme normale conjonctive (FNC) et la forme normale disjonctive (FND).
% Pour l'algorithme de preuve, nous allons utiliser la FNC, mais comme la FND est parfois importante,
% nous les définissons toutes les deux.
% Dans la FNC, la formule est écrite comme une conjonction de disjonctions.
% Dans la FND, la formule est écrite comme une disjonction de conjonctions.
% À l'intérieur de chaque forme normale on trouve des propositions premières ou des négations des propositions premières.
% Voici un exemple de chaque forme normale:
% \begin{itemize}
%   \item FNC: $( P \lor \lnot Q ) \land ( Q \lor A ) \land ( \lnot S \lor R )$  
%   \item FND: $( P \land \lnot Q ) \lor ( Q \land A ) \lor ( \lnot S \land R )$  
% \end{itemize}
% Pour faciliter la discussion autour des formes normales, nous introduisons une terminologie:
% \begin{itemize}
% \item Un {\em littéral}, écrit $L$, est où une proposition première où la négation d'une proposition première.
% Pour une proposition première $P$ on peut faire deux littéraux,
% $P$ et $\lnot P$.
% \item Une {\em clause}, écrite $C$, est (pour la forme normale conjonctive) une disjonction de littéraux.
% On écrit $\lor L_i$ ou $( L_1 \lor L_2 \lor L_3 \lor ... \lor L_i )$.
% \end{itemize}
% 
% \subsubsection{Algorithme de normalisation}
% 
% Nous donnons une explication brève de l'algorithme de normalisation,
% qui peut transformer toute formule en forme normale conjonctive.
% L'algorithme a quatre phases:
% \begin{enumerate}
% \item Eliminer les $\rightarrow$ et $\leftrightarrow$ en les remplaçant par des formules équivalentes.  Par exemple, $p \rightarrow q$ sera remplacée par $\lnot p \lor q$.
% \item Déplacer les négations vers l'intérieur (jusqu'à dans les propositions premières) en utilisant les formules de De Morgan.
% \item Déplacer les disjonctions ($\lor$) vers l'intérieur en utlisant les lois distributives.
% \item Simplifier en éliminant les formes $(P \lor \lnot P)$ dans chaque disjonction.
% \end{enumerate}
% 
% \subsubsection{Exemple de normalisation}
% 
% \begin{align*}
% & (P \rightarrow (Q \rightarrow R)) \rightarrow ((P \land S) \rightarrow R) \\
% & \lnot (\lnot P \lor (\lnot Q \lor R)) \lor (\lnot (P \land S) \lor R) \\
% & ( \lnot \lnot P \land \lnot (\lnot Q \lor R)) \lor ((\lnot P \lor \lnot S) \lor R) \\
% & (P \land (Q \land \lnot R)) \lor ( \lnot P \lor \lnot S \lor R) \\
% & (P \lor \lnot P \lor \lnot S \lor R) \land ( Q \lor \lnot P \lor \lnot S \lor R) \land (\lnot R \lor \lnot P \lor \lnot S \lor R) \\
% & (Q \lor \lnot P \lor \lnot S \lor R) 
% \end{align*}
% 

  \section{Deux règles plus sophistiquées}
\subsection{Théorème de déduction}

\begin{itemize}
\item  Pour prouver s $\Rightarrow$ t
\item  On suppose s vrai
\item  On déduit t
\item  Ensuite, on évacue l'hypothèse
\end{itemize}


\textit{Notation: p $\vdash$ t (on peut prouver t à partir de p) }

\subsubsection{Prémisse:}

\begin{equation}
\frac{p,..., r, s \vdash t} 
{p,..., r \vdash (s \Rightarrow t)}
\end{equation}

\subsubsection{Conclusion:}

Déduire une implication

\subsection{Preuve par contradiction (ou preuve indirecte)}

On prend une hypothèse, et on peut prouver qu'elle est vraie ou fausse, d'où l'hypothèse n'est pas bonne.

\subsubsection{Prémisse:} 
on suppose que p...q n'a pas de problème

\begin{equation}
\begin{split}
p,...,q, r, s \vdash s \\
\frac{p,...,q, r, s \vdash \lnot s}
{p,...,q \vdash \lnot r}
\end{split}
\end{equation}

\subsubsection{Conclusion:}

si p...q n'a pas de problème, on se focalise alors sur r
 

\section{Exemples de preuves}

Une preuve est une séquence de pas où chaque pas est une application de règles d'inférences et de lois logiques
Il faut justifier à chaque étape le nom de la règle / loi, et indenter les éléments de la preuve en preuve conditionnelle indirecte.

\subsection{Prémisse:} 

$(p \land q) \lor r$

\subsection{Conclusion:}

$\lnot$ p $\Rightarrow$ r

\subsection{Exemple sans preuve conditionnelle}

\begin{enumerate}
\item   (p $\land$ q) $\lor$ r  \textit{Prémisse}
\item   r $\lor$ (p $\land$ q)  \textit{Commutativité en 1}
\item   (r $\lor$ p) $\land$ (r$\lor$q) \textit{Associativité en 2}
\item   (r $\lor$ p) \textit{Simplification en 3}
\item   (p $\lor$ r) \textit{Commutativité en 4}
\item   $\lnot$$\lnot$ p $\lor$ r \textit{Loi de la négation en 5}
\item   $\lnot$ p $\Rightarrow$ r \textit{Implication en 6}
\end{enumerate}

\subsection{Exemple avec preuve conditionnelle}

\begin{enumerate}
\item (p $\land$ q) $\lor$ r  \textit{Prémisse}
\item $\lnot$ $\lnot$(p $\land$ q) $\lor$ r \textit{Double négation en 1}
\item $\lnot$ ( $\lnot$ p $\lor$ $\lnot$ q) $\lor$ r \textit{Loi De Morgan en 2}
\item $\lnot$ p $\lor$ $\lnot$ q $\Rightarrow$ r \textit{Implication en 3}

\begin{enumerate}
 \item  $\lnot$ p \textit{Hypothèse}
 \item  $\lnot$ p $\lor$ $\lnot$ q \textit{ Addition sur 5}
 \item  r \textit{ Modus Ponens sur 4 et 6}
\end{enumerate}

\item  $\lnot$ p $\Rightarrow$ r \textit{Evacuation de l'hypothèse}
\end{enumerate}

\subsection{Exemple de preuve par contradiction}

\begin{enumerate}
\item  (p $\land$ q) $\lor$ r  \textit{Prémisse}
\item  (p  $\lor$ r) $\land$ (q $\lor$ r) \textit{Distributivité sur 1}
\item  (p $\lor$ r)  \textit{Simplification en 2}

\begin{enumerate}
 \item $\lnot$ ( $\lnot$ p $\Rightarrow$ r)  \textit{Hypothèse}
 \item $\lnot$ ( $\lnot$ $\lnot$ p $\lor$ r)\textit{Implication en 4}
 \item $\lnot$ (p $\lor$ r) \textit{ Négation en 5}
\end{enumerate}

\item $\lnot$ $\lnot$ ($\lnot$ p $\Rightarrow$ r)  \textit{ Preuve par contradiction}
\item $\lnot$ p $\Rightarrow$ r \textit{Négation en 7}
\end{enumerate}


\section{Quelques concepts supplémentaires}
\subsection{Principe de dualité }

\subsubsection{Dans les formules sans $\rightarrow$ :}

\begin{align*}
\land \leftrightarrow \lor \\ 
true \leftrightarrow false 
\end{align*}
\begin{align*}
\models \lnot ( p \land q)  \Leftrightarrow \lnot p \lor \lnot q \\
\models \lnot ( p \lor q)  \Leftrightarrow \lnot p \land \lnot q 
\end{align*}

\subsubsection{Formule quelconque:}

\begin{align*}
\land \leftrightarrow \lor \\ 
true \leftrightarrow false \\ 
p \leftrightarrow \lnot p 
\end{align*}

 Justification en raisonnant sur les modèles:
 \begin{align*}
	 (p_1,...,p_n) \models q \hspace{1cm} ssi \models (p_1 \land ... \land p_n \land q) \leftrightarrow false \\
	 ssi \models ( \lnot p_1 \lor ... \lor \lnot p_n \lor q) \leftrightarrow true
 \end{align*}



\subsection{Algorithme de normalisation}

Une formule quelconque peut être transformée en une formule équivalente de forme normale.

\subsubsection{Forme Normale}

\begin{itemize}
  \item Conjonctive: $( p \lor q ) \land ( q \lor a ) \land ( s \lor r )$  
  \item Disjonctive: $( p \land q ) \lor ( q \land a ) \lor ( s \land r )$  
\end{itemize}

\subsubsection{Terminologie}

\begin{itemize}
	\item Littéral : $P \lor \lnot P \approxeq L$
	\item Clause : $\lor L_i = ( L_1 \lor L_2 \lor L_3 \lor ... \lor L_i )$
\end{itemize}

\subsubsection{Algorithme de normalisation}

\begin{enumerate}
\item Eliminer les $\rightarrow$ et $\leftrightarrow$
\item Déplacer les négations vers l'intérieur (dans les propositions premières) De Morgan
\item Déplacer les disjonctions ($\lor$) vers l'intérieur
\item Simplifier $(P \lor \lnot P)$
\end{enumerate}

\subsubsection{Exemple de normalisation}

\begin{align*}
& (p \rightarrow (Q \rightarrow R)) \rightarrow ((P \land S) \rightarrow R) \\
& \lnot ( ... ) \lor ( ... ) \\
& \lnot (\lnot P \lor (\lnot Q \lor R)) \lor (\lnot (P \land S) \lor R) \\
& ( \lnot \lnot P \land \lnot (\lnot Q \lor R)) \lor ((\lnot P \lor \lnot S) \lor R) \\
& (P \land (Q \land \lnot R)) \lor ( \lnot P \lor \lnot S \lor R) \\
& (P \lor \lnot P \lor \lnot S \lor R) \land ( Q \lor \lnot P \lor \lnot S \lor R) \land (\lnot R \lor \lnot P \lor \lnot S \lor R) \\
& (Q \lor \lnot P \lor \lnot S \lor R) 
\end{align*}


  \chapter{Algorithme de preuve}

% \section{Exemple de preuve propositionnelle}
% 
% \noindent Voici les propositions premières : \\
% A : tu manges bien \\
% B : ton système digestif est en bonne santé \\
% C : tu pratiques une activité physique régulière \\
% D : tu es en bonne forme physique \\
% E : tu vis longtemps \\
% 
% \noindent On peut maintenant faire une théorie et on espère qu'elle aura un modèle. \\
% A$\Rightarrow$B, C$\Rightarrow$D, B$\lor$D $\Rightarrow$ E, $\lnot$E
% \\
% \noindent On aimerait prouver que $\lnot$A $\land$ $\lnot$C est vrai.\\
% 
% \noindent Preuve : \\
% \\
% \begin{tabular}{|l|l|}
% \hline
% 1. A$\Rightarrow$B & prémisse \\
% 2. C$\Rightarrow$D & prémisse \\
% 3. B$\lor$D $\Rightarrow$E & prémisse \\
% 4. $\lnot$E & prémisse \\ 
% \indent 5. A & hypothèse \\
% \indent 6. B & modus ponens (1) \\
% \indent 7. B$\lor$D & addition (6) \\
% \indent 8. E & modus ponens (7) \\
% 9. $\lnot$A & preuve indirecte \\
% \indent 10. C & hypothèse \\
% \indent 11. D & modus ponens (2) \\
% \indent 12. D$\lor$B & addition (11) \\
% \indent 13. B$\lor$D & commutativité (12)\\
% \indent 14. E & modus ponens (9) \\
% 15. $\lnot$C & preuve indirecte \\
% 16. $\lnot$A $\land$ $\lnot$C & conjonction (9,15) \\
% \hline
% \end{tabular}\\
% 
% Grâce à la déduction on a donc pu prouver que tu ne manges pas bien et que tu ne pratiques pas d'activité physique régulière. 
% On aimerait maintenant pouvoir automatiser les preuves quand elles existent. Mais il faut savoir s'il peut tout résoudre ou pas. 
% On va donc construire un algorithme nous permettant de résoudre automatiquement les preuves en logique propositionnelle.

Nous allons maintenant introduire un algorithme qui permet de trouver
une preuve en logique des prédicats.
Cet algorithme est une automatisation de la {\em démonstration par l'absurde}
qui est basé sur une seule règle d'inférence, la {\em résolution}.

\section{La résolution}

On veut quelque chose de simple, sans toutes les règles que nous avons vues auparavant, mais le plus puissant possible. Nous n'utiliserons qu'une seule règle : la \textbf{résolution}. On peut faire des résolutions de preuves propositionnelles rien qu'en ayant cette règle. Cette règle utilise la forme normale conjonctive. 
On utilise les preuves indirectes (preuves par l'absurde), car c'est le plus simple.
\\
Commençons par un exemple de résolution.

\subsection{Exemple de résolution}

\noindent Prenons comme propositions premières :\\

\noindent P$_{1}$ : il neige \\
P$_{2}$ : la route est dangereuse \\
P$_{3}$ : on prend des risques \\
P$_{4}$ : on va vite \\
P$_{5}$ : on va lentement \\
P$_{6}$ : on prend le train \\


\noindent $\left.
\begin{array}{l}
$1. P$_{1}$ $\Rightarrow$ P$_{2}$ $ \\
$2. P$_{2}$ $\Rightarrow$ $\lnot$P$_{3}$ $ \\
$3. P$_{4}$ $\Rightarrow$ P$_{3}$ $\lor$   P$_{6}$ $ \\
$4. P$_{4}$ $\lor$ P$_{5}$ $ \\
$5. P$_{1}$ $ \\
\end{array}
\right\rbrace$ B : notre théorie \\
\\
On va utiliser B + modus ponens + résolution.. \\
\\
\begin{tabular}{|l|l|}
\hline
1. P$_{1}$ $\Rightarrow$ P$_{2}$&  \\
2. P$_{2}$ $\Rightarrow$ $\lnot$P$_{3}$ & \\
3. P$_{4}$ $\Rightarrow$ P$_{3}$ $\lor$ P$_{6}$ & 1-5 : B : notre théorie que l'on utilise comme prémisse\\
4. P$_{4}$ $\lor$ P$_{5}$ & \\
5. P$_{1}$ & \\
3'. $\lnot$P$_{3}$ $\Rightarrow$ $\lnot$P$_{4}$ $\lor$ P$_{6}$ & réécriture de 3 \\
6. P$_{2}$ & modus ponens (1,5) \\
7. $\lnot$P$_{3}$ & modus ponens (6,2) \\
8. $\lnot$P$_{4}$ $\lor$ P$_{6}$ & modus ponens (7,3') \\
9. P$_{5}$ $\lor$ P$_{6}$ & résolution (4,8) \\
\hline
\end{tabular}\\
\\

La ligne 3 n'étant pas symétrique, nous pouvons la transformer pour obtenir une proposition symétrique et donc choisir le membre qui est à gauche de l'implication. Pour rappel, P$_{4}$ $\Rightarrow$ P$_{3}$ $\lor$ P$_{6}$ peut être réécrit  : $\lnot$P$_{4}$ $\lor$ P$_{3}$ $\lor$ P$_{6}$ (loi de l'implication), qui est logiquement équivalent à $\lnot$P$_{3}$ $\Rightarrow$ $\lnot$ P$_{4}$ $\lor$ P$_{6}$ (loi de l'implication). C'est de cette manière que nous avons obtenu la ligne 3'.\\

On peut fusionner les lignes 4 et 8 grâce à la résolution. La \textbf{résolution} est une règle qui prend deux disjonctions avec une proposition première et sa négation, et qui les fusionne en retirant cette proposition première. On peut prouver que cela fonctionne de plusieurs manières.\\ 
Par exemple : si P$_{4}$ est vrai, P$_{6}$ doit être vrai. Si P$_{4}$ est faux, P$_{5}$ doit être vrai. Donc on sait que P$_{5}$ ou P$_{6}$ doit être vrai car on sait que dans tous les cas de figure, c'est soit l'un soit l'autre qui doit être vrai. 


\subsection{Principe de résolution}
\noindent p$_{1}$ $\lor$ q \\
\noindent p$_{2}$ $\lor$ $\lnot$q \\
\rule{3cm}{0.4pt} \\
p$_{1}$ $\lor$ p$_{2}$
\\

Cette règle représente la base de l'algorithme de résolution. On peut la vérifier en utilisant le métalangage. De plus, cette règle est aussi utilisée dans la logique des prédicats. 

\subsection{La résolution préserve les modèles}
\noindent Tout ce qui est modèle des deux premières disjonctions sera aussi modèle de la résultante. \\

\noindent $p$ : $\bigwedge\limits_{1 \leq i \leq n}$ C$_{i}$ \indent   \indent \indent \indent C$_{i}$ : disjonction : $\bigvee\limits_{1 \leq j \leq n}$ L$_{j}$ \indent  \{C$_{1}$, ..., C$_{n}$\}\\

\noindent C$_{1}$, C$_{2}$ = deux disjonctions \\

\noindent On doit prouver : \{C$_{1}$, ..., C$_{n}$\} $\models$ $r$ avec  $r$ = C$_{1}$ - \{P\} $\lor$ C$_{2}$ - \{$\lnot$P\}. $r$ est une nouvelle disjonction à partir de deux autres disjonctions. On doit prouver que $r$ est toujours vrai.\\

\noindent On considère que P est dans C$_{1}$ et que $\lnot$P est dans C$_{2}$.\\
\\
Pour prouver cela, on utilise la sémantique. On fait une preuve en métalangage, ce n'est pas formalisé. \\
Val$_{I}$(P) =
$\left\lbrace
\begin{array}{l}
T \\
F \\
\end{array}
\right.$ 

Dans les deux cas de figure, on doit démontrer que quand on a un modèle, une interprétation qui rend vrai $p$, le $r$ sera vrai aussi. Si P est vrai alors $\lnot$P est faux, donc C$_{2}$ sera vrai et donc $r$ sera vrai. Quand P est faux, le C$_{1}$ doit être vrai, donc $r$ est vrai. $r$ est donc vrai dans les deux cas. 

\section{Algorithme}

\noindent C$_{i}$ : clause $\bigvee\limits_{i}$ L$_{i}$ \newline
L$_{i}$ : P ou $\lnot$P \newline
\{C$_{i}$,...,C$_{n}$\} $\models$ C \newline
s.s.i \newline
\{C$_{i}$,...,C$_{n}$,$\lnot$C\} $\models$ false \newline
C$_{i}$ : axiomes \newline
C: candidat théorème \newline
Ce que nous voulons prouver : \{C$_{i}$,...,C$_{n}$\} $\vdash$ C \newline
\noindent \emph{Il existe une preuve avec les règles d'inférence \{C$_{i}$,...,C$_{n}$\} tel qu'on obtient C.} \newline
S = \{C$_{i}$,...,C$_{n}$,$\lnot$C\} \newline
But : Déterminer si S est inconsistant. On veut faire des déductions jusqu'à arriver sur false. 

\subsection{Pseudocode}

\noindent
\begin{tabbing}
\hspace{2cm}\=\kill
\underline{while} \> false $\not\in$ S et $\exists$  ? clauses résolvables non-résolues \\
\newline
\indent \underline{do} \> choisir C$_{1}$, C$_{2}$ $\in$ S tel que $\exists$ P $\in$ C$_{1}$, $\lnot$P $\in$ C$_{2}$ \\
\> calculer r := C$_{1}$ - \{P\} $\lor$ C$_{2}$ - \{$\lnot$P\} \\
\> calcul S := S $\cup$ \{r\} \\
\underline{end}
\end{tabbing}
\underline{if} false $\in$ S \underline{then} \emph{C est prouvé} \underline{else} \emph{C n'est pas prouvé} \newline

La subtilité dans cet algorithme est de choisir correctement les clauses C$_{1}$ et C$_{2}$ car l'efficacité de l'algorithme en dépend. 

\section{Exemples}

\subsection{Exemple 1}
\begin{tabbing}
\hspace{3cm}\=\hspace{2cm}\=\kill
C$_{1}$ : P $\lor$ Q \\
C$_{2}$ : P $\lor$ R \\
C$_{3}$ : $\lnot$Q $\lor$ $\lnot$R \\
C : P \> \> \{C$_{1}$,C$_{2}$,C$_{3}$,$\lnot$C\} \\
\end{tabbing}

\noindent \emph{Quelques pas de résolution :}

\noindent C$_{1}$ + $\lnot$C $\rightarrow$  Q (C$_{5}$) \newline
C$_{2}$ + $\lnot$C $\rightarrow$ R  (C$_{6}$) \newline
C$_{3}$ + C$_{5}$ $\rightarrow$ $\lnot$R (C$_{7}$) \newline 
C$_{6}$ + C$_{7}$ $\rightarrow$ \underline{false} ($\in$ S donc C est prouvé) \newline

\subsection{Exemple 2}
\noindent p$_{1}$ : Mal de tête $\land$ Fièvre $\Rightarrow$ Grippe \newline
p$_{2}$ : Gorge blanche $\land$ Fièvre $\Rightarrow$ Angine \newline
p$_{3}$ : Mal de tête \newline
p$_{4}$ : Fièvre\newline

\noindent \underline{Algorithme}
\begin{itemize}
\item{Normalisation en forme normale}
\item{Pseudocode avec résolution}
\end{itemize}
Question : Grippe ?

\section{Conclusion}
Nous pouvons tirer des conclusions sur la logique des propositions et sur notre algorithme.

Pour toute théorie $B = \left\{c_1, \dots, c_n \right\}$ et $p$,
\begin{itemize}
\item si $B \vdash p$ alors $B \models p$ (Adéquat - \textit{Soundness}) ;
\item si $B \models p$ alors $B \vdash p$ (Complet - \textit{Completeness}) ;
\item $\forall$ $B, p$, l'exécution de l'algorithme se termine après un nombre fini d'étapes. (Décidable - \textit{Decidable})
\end{itemize}
Cet algorithme est donc très puissant, même s'il n'est peut être pas très efficace. Mais au moins on est sûr qu'il s'arrêtera toujours à un moment.

Malheureusement, la logique des propositions n'est pas très expressive. On va donc essayer de faire la même chose mais avec une logique plus puissante : la logique des prédicats. Mais nous n'arriverons pas à avoir un algorithme aussi puissant pour la logique des prédicats, car c'est une logique trop forte. 

  %Packages à ajouter pour la compilation
%\usepackage{amsmath}
%\usepackage{lmodern}
%\usepackage{vmargin}
%\usepackage{tabularx}
%\usepackage[usenames,dvipsnames]{color}

\chapter{La logique des prédicats}
 
\section{Introduction}

Nous allons maintenant étudier une logique beaucoup plus expressive que la
logique propositionnelle, la logique des prédicats, qui est aussi appelée
la logique de premier ordre.\footnote{Il existe des logiques d'ordres supérieures,
mais elles ne feront pas l'objet de ce cours.}
Voici un premier tableau qui montre les différences entre la logique propositionnelle vue jusqu'à présent et la logique des prédicats que nous allons étudier.
\begin{center}
\begin{tabular}{|c|c|}
\hline 
Logique Propositionnelle & Logique des prédicats \\ 
\hline
Propositions premières & Prédicats P(x,y) \\ 
P, Q, R & Quantifieurs: $\exists$x, $\forall$y \\ 
$\hookrightarrow$ Pas de Relations & $\hookrightarrow$ Relation \\ 
\hline 
\end{tabular} 
\end{center}

On note P(x,y) dans la logique des prédicats avec x,y, les arguments du prédicat P qui sont des variables.
Dans la logique propositionnelle, chaque proposition est isolée/indépendante alors que dans les prédicats on peut lier plusieurs prédicats ensemble.


\begin{center}
\begin{tabular}{|c|c|c|}
\hline 
Exemple & Logique propositionnelle & Logique des prédicats \\ 
\hline 
Socrate est un philosophe & P & Phil(Socrate) \\ 
Platon est un philosophe & Q & Phil(Platon) \\ 
\hline 
\end{tabular} 
\end{center}

En logique propositionnelle il n'y a aucunes relations entre P et Q, alors qu'en logique des prédicats on peut lier Socrate et Platon avec le prédicat Philosophe qui prend en argument le nom du philosophe (Socrate ou Platon dans ce cas). Phil(Socrate) est donc vrai. On peut donc dire grâce aux prédicats que Socrate et Platon sont "la même chose", des philosophes.\\

Un autre exemple de prédicat:

\begin{center}
$\forall \alpha$ Phil($\alpha$) $\Rightarrow$ Savant($\alpha$)\\
\vspace{3mm}
$\hookrightarrow$ ... \textit{c'est un résumé d'un très grand nombre de faits. Ça marche pour tous arguments, ça peut être un ensemble infini!}
\end{center}
Comme Socrate est un philosophe, on peut déduire que Socrate est un savant aussi!

Dire la même chose en logique propositionnelle serait beaucoup plus compliqué: \\

"Socrate est un savant" Proposition "R"\\
\indent "Platon est un savant" Proposition "S"\\

On va donc noter en logique propositionnelle
\begin{center}
(P$\Rightarrow$R) $\cup$ (Q$\Rightarrow$S) $\cup$ ...(\textit{potentiellement infini})
\end{center}

On doit tout énumérer et il n'y a aucune relation entre les différentes propositions. S'il y a un nombre infini, ça ne marche pas. Il y a donc de grandes limitations dans la logique propositionnelle.

Néanmoins parfois la logique propositionnelle peut être utile. 
Il existe des outils informatiques qui utilisent la logique propositionnelle. On peut prendre l'exemple de "SAT solver" à qui on donne des équations booléennes très compliquées et qui va trouver les valeurs des propositions primitives qui rendent vraie cette proposition. C'est donc assez utilisé! La logique propositionnelle est utile, mais si l'on veut faire du raisonnement sur plus que  "vrai" et "faux" avec des relations entre des propositions,  la logique propositionnelle ne marche pas. Si on veut faire un logiciel qui montre une certaine intelligence, il faut utiliser la logique des prédicats.\\

Autre exemple:

\begin{tabular}{|ccc|} 
\hline
Exemple & Logique propositionnelle & Logique des prédicats \\ 
\hline
Tout adulte peut voter & P & $\forall$x adulte(x) $\Rightarrow$ voter(x) \\ 
John est un adulte & Q & adulte(\textcolor{OliveGreen}{John}) \\ 
\line(1,0){50} & \line(1,0){10} & \line(1,0){45} \\ 

John peut voter & \textcolor{Red}{?R?}& voter(\textcolor{OliveGreen}{John}) \\ 
\hline
\end{tabular}\\

Ce genre de raisonnement est très difficile à faire en logique propositionnelle alors qu'en logique des prédicats c'est beaucoup plus simple! Le John en ligne 3 et en ligne 4 correspond à la même personne, ou de manière plus général à la même variable! 
Ceci montre donc bien qu'il nous faut la logique des prédicats pour faire des relations de ce type.

\section{Quantificateurs}

Les expressions "pour tout $x$" ($\forall x$) et "il existe $x$ tel que" ($\exists x$) sont appelés des quantificateurs en logique des prédicats. Les quantificateurs permettent d'instancier les variables dans une formule. La notion de portée d'un quantificateur est un concept très important auquel il faut faire très attention, car il peut changer complètement le sens d'une formulation. \\

$\forall x$ (enfants($x$) $\wedge$ intelligents($x$) $\Rightarrow$ $\exists y$ aime($x$,$y$)) \\

$\forall x$ (enfants($x$) $\wedge$ intelligents($x$)) $\Rightarrow$ $\exists y$ aime($x$,$y$) \\

Ces deux formules peuvent paraître équivalentes, mais en réalité elles ont un sens tout à fait différent.
En effet, dans le deuxième cas on remarque que le quantificateur $\forall x$ ne porte pas sur la dernière variable $x$ qui est en argument du prédicat aime($x$,$y$).

Il faut donc faire bien attention à quel quantificateur une variable s'identifie lorsqu'on manipule des formules. \\

\begin{itemize}
\item[$\bullet$] $\forall x$ P($x$) $\wedge$ $\exists x$ Q($x$) : contient deux variables différentes\\
 
\item[$\bullet$] $\forall x$ $\exists x$  P($x$) $\wedge$ Q($x$) : est une forme incorrecte, conflit des noms de variables \\
\end{itemize}

Pour résoudre ces conflits, on fait appel à une nouvelle opération, le renommage. Cette opération permet de changer le nom des variables tout en conservant le sens de la formule. Ainsi on obtient : \\

 $\forall x$ $\exists z$  P($x$) $\wedge$ Q($z$)  \textit{renommage (2)} \\
 
 Le concept de variables, de leurs portées ainsi que d'opérateurs en logique des prédicats fait fortement penser au langage de programmation
 
Une comparaison entre un code et une formule est tout à fait envisageable. Prenons un code tout à fait banal comprenant des variables différentes avec des portées différentes qui ont le même identificateur ainsi qu'une formule correspondante. 

\begin{verbatim}
1.  begin {
2.      var x,y: int;        
3.      x := 4;
4.      y := 2;
5.    
6.      begin {
7.          var x: int;
8.          x := 5;
9.          x := x*y;
10.     end }
11.     x := x*y;
12. end }
\end{verbatim}


\textcolor{Green}{$\forall x$} \textcolor{Red}{$\forall y$} p(\textcolor{Green}{$x$}) $\wedge$ (\textcolor{Blue}{$\exists x$}  q(\textcolor{Blue}{$x$},\textcolor{Red}{$y$}) $\vee$ r(\textcolor{Green}{$x$},\textcolor{Red}{$y$}))  \\ 

En analysant morceau par morceau de la formule : \\

\begin{itemize}

\item[$\bullet$] " \textcolor{Green}{$\forall x$} \textcolor{Red}{$\forall y$} p(\textcolor{Green}{$x$}) $\wedge$ " correspond aux points $\lbrace 2,3,4 \rbrace$ du code \\

\item[$\bullet$]" \textcolor{Blue}{$\exists x$}  q(\textcolor{Blue}{$x$},\textcolor{Red}{$y$}) $\vee$ " correspond aux points $\lbrace 7,8,9 \rbrace$ \\
 
\item[$\bullet$]" r(\textcolor{Green}{$x$},\textcolor{Red}{$y$})) "  correspond au point $\lbrace 11 \rbrace$ \\

\end{itemize}

Cet exemple illustre parfaitement la ressemblance et le lien entre le monde de la programmation et celui de la logique des prédicats

\section{Syntaxe}

\begin{tabular}{|c|c|c|}
	\hline
	Symboles logiques & quantificateurs & $\forall$ $\exists$ \\
	                  & connecteurs logiques & $\wedge$ $\vee$ $\neg$ $\Rightarrow$ $\Leftrightarrow$ \\
	                  & parenthèses & ( ) \\
	                  & variables & $x, y, z$ \\
	                  & true, false & \\
	\hline
	Symboles non logiques & symboles de prédicats & $P$ $\varphi$ $R$ + arguments $\geq 0$ \\
						  & symboles de fonction & + $arguments \geq 0 $\\
	\hline
\end{tabular}

\section{Grammaire}
\subsection{Règles de formation}
\begin{tabular}{rl}
$<formule>::=$ 	  &	$<formule$ $atomique>$ \\
				  & $\vert$ $\neg$ $<formule>$ \\
				  & $\vert$ $<formule>$ $<connecteur>$ $<formule>$ \\
				  & $\vert$ $\forall <var>.<formule>$ \\
				  & $\vert$ $\exists <var>.<formule>$ \\
$<formule$ $atomique>::=$ 
				  & true, false \\
				  & $\vert$ $<predicat>(<terme>*)$ \\
$<terme>::=$	  & $<constante>$ \\
				  & $\vert <var>$ \\
				  & $\vert <fonction>(<terme>*)$ \\
$<connecteur$ $binaire>::=$ 
				  & $\wedge \vert \vee \vert \Rightarrow \vert \Leftrightarrow$ \\

\end{tabular}

\section{Sémantique}
Dans la logique des prédicats, nous gardons les notions de modèle et d'interprétation qui sont déjà définis dans la logique propositionnelle. Même si la logique des prédicats est beaucoup plus puissante, nous gardons une sémantique assez similaire à la logique propositionnelle. \\
L'idée dans l'interprétation c'est de dire si la formule est vraie ou fausse. Tout comme pour la logique propositionnelle, on va faire la même chose pour les prédicats avec les relations.\\

Illustrons par un exemple : 

\begin{center}
$p : P(b,f(b)) \Rightarrow \exists y   P(a,y)$  \\
\vspace{3mm}
\end{center}
On suppose que le $a$ et le $b$ sont des constantes et que le $f$ est une fonction. Il faut donner un sens à cette formule et pour cela il faut donner un sens à : 
\begin{itemize}
\item[$\bullet$]P : $val_{I}(P) = $ $ \geq $ \hspace{3mm} (\textit{considéré comme un vrai prédicat})
\item[$\bullet$] a : $val_{I}(a) = $ $ 2 $ 
\item[$\bullet$] b : $val_{I}(b) = $ $ \pi $ 
\item[$\bullet$] $f$ : $val_{I}(f) = $ $ f_{i} $ \hspace{3mm} $f_{i}= \Re \rightarrow \Re : d \rightarrow \dfrac{d}{2} $ 
\end{itemize}
Avec ces éléments, on peut donc trouver l'interprétation : 
\begin{center}
\textit{Si $\pi \geq  \dfrac{\pi}{2}$, alors $\exists$ $ d \in \Re$ tel que $\sqrt2 \geq d$ }
\end{center}
Cette phrase n'est pas très utile, mais avec cette interprétation, la phrase logique donne ce sens. Une autre interprétation donnerait un sens totalement différent à la phrase logique. Voici une autre interprétation totalement différente :
\begin{itemize}
\item[$\bullet$] a : $val_{I}(a) = $ "Barack Obama"
\item[$\bullet$] b : $val_{I}(b) = $ "Vladimir Putin"
\item[$\bullet$] $f$ : $val_{I}(f) = $ $ f_{i} $ \hspace{3mm} $f_{i} \rightarrow$ père(d) 
\item[$\bullet$] P :  $val_{I}(P) = $ $P_{I}$ \hspace{3mm} $d_{1}$ est enfant de $d_{2}$\\
\end{itemize}

Avec ce nouveau sens, on trouve l'interprétation suivante : 
\begin{center}
\textit{Si Vladimir Putin est l'enfant du père de Vladimir Putin alors $\exists$ une personne telle que Barack Obama est l'enfant de cette personne.}
\end{center}
La seconde interprétation est très différente de la première malgré le fait que ce soit la même formule à l'origine ! La connexion entre une formule et son sens permet de garder une certaine souplesse dans le sens où l'on peut choisir ça. C'est un peu comme dans la logique propositionnelle, mais avec encore plus de souplesse. Si l'on fait ce raisonnement-ci, ce dernier sera toujours vrai pour toutes les interprétations qui sont vraies. \\

On peut se demander si ces deux interprétations sont des modèles de la formule ? \\
La première interprétation est un modèle de la formule, car le sens de la formule est vrai dans l'interprétation. En effet, $\pi \geq \dfrac{\pi}{2}$ et $\exists$ $ d \in \Re$ tel que $\sqrt2 \geq d$. On voit donc que le modèle est vrai.

La deuxième interprétation est aussi un modèle de la formule, car l'interprétation trouvée est vraie aussi. Cela peut paraître bizarre, mais c'est correct. 

  \include{partie7}
  
\section{Élimination de $\forall$}
\begin{flushleft}

$\forall$x$\bullet$P(x) $\rightarrow$ P(a) $\>$ a est une constante\\
$\>$ $\>$ $\>$ $\>$ $\>$ $\>$ $\>$ $\rightarrow$ P(y) $\>$ y est une variable $\>$ ($P_{I}$($y_{I}$) est vrai $\>$ $y_{I}$ $\in$ $P_{I}$)$\linebreak$ 
$\forall$ = pour tout $x_{I}$ $\in$ $P_{I}$ : $P_{I}$($x_{I}$) est vrai$\linebreak$

\underline{Règle :}$\linebreak$\\
\begin{center}
{\LARGE $\frac{\forall x : p}{p[x/t]}$}
\begin{flushright}$\>$ Substitution : t remplace x\end{flushright}
\end{center}
\textcolor{red}{\danger Il est parfois nécessaire d'effectuer un renommmge }

\underline{Exemple :}\\
\begin{enumerate}
\item $\forall$x $\bullet$ $\forall$y $\bullet$ P(x,y) $\>$ Pr\'emisse
\item $\forall$y $\bullet$ P(x,y) $\>$ $\>$ $\>$ $\>$ $\>$ Élimination de $\forall$
\item P(x,x) $\>$ $\>$ $\>$ $\>$ $\>$ $\>$ $\>$ $\>$ $\>$ Élimination de $\forall$ $\rightarrow$ Pas de renommage, car pas de capture
\item $\forall$x $\bullet$ P(x,x) $\>$ $\>$ $\>$ $\>$ $\>$ Introduction de $\forall$
\end{enumerate}

\section{Élimination de $\exists$}
$\exists$x $\bullet$ P(x) $\rightarrow$ P(x) $\>$ x = nouvelle constante qui apparaît nulle part ailleurs ($val_{I}(a) = x_{I}$)\\
Il existe un $x_{I}$ $\in$ $D_{I}$ avec $P_{I}$($x_{I}$) est vrai\\
$\>$ $\>$ $\>$ $\>$ $\>$ $\>$ $\>$ $\>$ $\nrightarrow$ P(y) $\>$ y = variable qui existe d\'ej\`a dans la preuve \\
$\>$ $\>$ $\>$ $\>$ $\>$ $\>$ $\>$ $\>$ $\rightarrow$ P(z) $\>$ z = nouvelle variable dans la preuve $\>$ $val_{I}(z) = x_{I}$\\
\underline{Exemple 1 :}\\
\begin{enumerate}
\item $\exists$x $\bullet$ chef(x) $\>$ $\>$ $\>$ $\>$ $\>$ $\>$ $\>$ $\>$ $\>$ $\>\>$Pr\'emisse
\item $\exists$x $\bullet$ voleur(x) $\>$ $\>$ $\>$ $\>$ $\>$ $\>$ $\>$ $\>$  $\>$Pr\'emisse
\item chef(y) $\>$ $\>$ $\>$ $\>$ $\>$ $\>$ $\>$ $\>$ $\>$ $\>$ $\>$ $\>$ $\>$ $\>$ $\>$Élimination de $\exists$
\item \sout{voleur(y)} $\>$ $\>$ $\>$ $\>$ $\>$ $\>$ $\>$ $\>$ $\>$ $\>$ $\>$ $\>$ $\>$ Élimination de $\exists$ $\>$ \textcolor{red}{y n'est pas une nouvelle variable dans la preuve}
\item chef(y) $\wedge$ voleur(y) $\>$ $\>$ $\>$ $\>$ $\>$ Conjonction
\item $\exists$y $\bullet$ chef(y) $\wedge$ voleur(y) $\>$ Introduction de $\exists$
\end{enumerate}

\underline{Exemple 2 :}\\
\begin{enumerate}
\item $\exists$x $\bullet$ chef(x) $\>$ $\>$ $\>$ $\>$ $\>$ $\>$ $\>$ $\>$ $\>$ $\>$ $\>$Pr\'emisse
\item $\exists$x $\bullet$ voleur (x) $\>$ $\>$ $\>$ $\>$ $\>$ $\>$ $\>$ $\>$ $\>$Pr\'emisse
\item chef(y) $\>$ $\>$ $\>$ $\>$ $\>$ $\>$ $\>$ $\>$ $\>$ $\>$ $\>$ $\>$ $\>$ $\>$ $\>$ Élimination de $\exists$
\item voleur (z) $\>$ $\>$ $\>$ $\>$ $\>$ $\>$ $\>$ $\>$ $\>$ $\>$ $\>$ $\>$ $\>$ Élimination de $\exists$
\item chef(y) $\wedge$ voleur(z) $\>$ $\>$ $\>$ $\>$ $\>$ $\>$ Conjonction
\item $\exists$y $\exists$z chef(y) $\wedge$ voleur(z) $\>$ Introduction de $\exists$
\end{enumerate}

\section{Introduction de $\exists$}
\underline{R\`egle :}\\
\begin{center}
{\LARGE $\frac{p[t]}{\exists x \bullet p[x]}$}
\end{center}
\begin{flushright}
Il y a une substitution p[x/t]
\end{flushright}

\underline{Exemple :}\\
\begin{center}
 P(y,y)\\
$\exists$x $\bullet$ P(x,x)\\[2\baselineskip]
\sout{P(y,x)}\\
\sout{$\exists$x $\bullet$ P(x,x)}
\begin{flushright}
\textcolor{red}{Ceci n'est pas correct !}
\end{flushright}
\end{center}
$\Rightarrow$ Il doit être possible de retrouver la formule originale en remplaçant.\\

\section{Introduction de $\forall$}
\underline{R\`egle :}\\
\begin{center}
{\LARGE $\frac{p}{\forall x \bullet p}$}
\end{center}
\begin{itemize}
\item Si p n'a pas d'occurrence libre de x alors c'est OK
\item Si p contient une occurrence libre de x : on doit s'assurer que la preuve jusqu'à cet endroit marchera pour toutes valeurs affectées à x\\
$\> \> \> \hookrightarrow$ Aucune formule dans la preuve jusqu'à cet endroit ne doit mettre une contrainte sur x !
\end{itemize}
\underline{Deux conditions :}\\
\begin{itemize}
\item x n'est pas libre dans une formule dans la preuve jusqu'à cet endroit obtenu par élimination de $\exists$
\item x n'est pas libre dans une prémisse (x est déjà connu au début donc il possède déjà une valeur)
\end{itemize}

\underline{Exemple :}\\
\begin{enumerate}
\item $\forall$x $\exists$y parent(y,x) $\>$ Prémisse
\item $\exists$y parent(y,x) $\>$ $\>$ $\>$ $\>$Élimination de $\forall$
\item parent(y,x) $\>$ $\>$ $\>$ $\>$ $\>$ $\>$ Élimination de $\exists$ \textcolor{blue}{$\rightarrow$ N'est valable que pour ce y et ce x, pas pour tous}
\item \sout{$\forall$x parent(y,x)} $\>$ $\>$ $\>$ $\>$Introduction de $\forall$ \textcolor{red}{$\rightarrow$ On ne peut pas faire ça, car il y a une contrainte sur x. Là on dit que ce y est parent de tous !}
\end{enumerate}




\end{flushleft}

  Terminons par un exemple un peu plus conséquent d'une preuve manuelle en logique des prédicats avant d'introduire l'algorithme permettant d'effectuer des preuves de manière automatisée.
\subsection{Exemple de preuve manuelle}
Il est important de pouvoir faire des preuves manuellement, car cela permet de bien comprendre toutes les étapes de raisonnement d'une preuve, même si par la suite on utilise un algorithme plutôt que de faire les preuves à la main. 

L'exemple suivant est inspiré de l'Empire romain: 
\subsubsection{Prémisses}
\begin{itemize}
    \item Les maîtres et esclaves sont tous des hommes adultes
    \item toutes les personnes ne sont pas des hommes adultes
\end{itemize}
\textit{Note: on voit dans les prémisses qu'il y a des quantificateurs: tous, toutes. 
}
\subsubsection{A prouver} 
\begin{itemize}
    \item il existe des personnes qui ne sont pas des maîtres
\end{itemize}
\subsubsection{Preuve}
\begin{enumerate}
    \item $\forall{}x\ (maitre(x)\lor{}esclave(x)\implies{}adulte(x)\land{}homme(x))$ \hfill prémisse
    \item $\lnot{}\forall{}x\ (adulte(x)\land{}homme(x))$ \hfill prémisse
    \item $\exists{}x\ \lnot{}(adulte(x)\land{}homme(x))$ \hfill théorème négation\\
    \textit{S' il n'est pas vrai que toutes les personnes sont des hommes adultes alors il existe une personne qui n'est pas un homme adulte}
    \item $\lnot{}(adulte(x)\land{}homme(x))$ \hfill $\exists{}$ elim\\
\textit{    On élimine le quantificateur existentiel:  on peut le faire, car on introduit une variable x qu'on choisit comme étant une personne rendant vraie la proposition.}
    \item $(maitre(x)\lor{}esclave(x)\implies{}adulte(x)\land{}homme(x))$ \hfill $\forall{}$ elim\\
    \textit{On peut retirer le $\forall$  en réduisant le champ de x aux x rendant vraie la proposition.}
    \item $\lnot{}(maitre(x)\lor{}esclave(x))$ \hfill modus tollens
    \item $\lnot{}maitre(x)\land{}\lnot{}esclave(x)$ \hfill De Morgan
    \item $\lnot{}maitre(x)$ \hfill simplification
    \item $\exists{}y\ \lnot{}maitre(y)$ \hfill $\exists{}$ intro\\
    \textit{Comme dans l'interprétation, x est une personne qui rend valable cette proposition, on peut dire qu'il existe une personne rendant valable cette proposition et réintroduire le quantificateur $\exists$}
\end{enumerate}
C'était un exemple très simple ne faisant que quelques pas, mais la logique est assez expressive pour permettre des preuves plus complexes(par exemple, formaliser les mathématiques), le nombre de pas serait alors beaucoup plus important. 

\hfill {\begin{minipage}{0.90\textwidth}
\begin{small}
{\large Instant Histoire:}\\
A la fin du 19ème siècle, début du 20ème: 
\begin{itemize}
\item création de la logique de 1er ordre (Gottlob Frege)
\item Deux personnes ont essayé de formaliser toutes les mathématiques. Principa Mathématica (Alfred Whitehead, Bertrand Russell)
\end{itemize}
Lors que l'arrivée des ordinateurs, fin du 20ème siècle(années 50-60 et fin du siècle) on a essayé de formaliser la logique via des algorithmes:
\begin{itemize}
\item Création de l'Algorithme de Preuves (1965):
\begin{itemize}
\item Alan Robinson crée La Règle de Résolution (qui va être expliquée au chapitre suivant)
\item Création de prouveurs (assistants de preuve) par exemple Coq et Isabelle en 1972
\item Création de la logique de programmation qui aide à l'élaboration de la programmation par contraintes: Prolog (1972) 
\end{itemize}
\item Création de la sémantique Web: OWL (Web Ontology Langage)
\end{itemize}
\end{small}
\end{minipage}
\chapter{Algorithme de preuve pour la logique des prédicats}
\begin{itemize}
    \item Cet algorithme s'inspire de l'algorithme de réfutation de la logique des propositions [résolution forme clausale]
    \item Pour la logique des prédicats, c'est un peu plus complique, mais ça marche!
\end{itemize}
On peut le faire marcher malgré la complexité des variables et des quantificateurs ce qui est assez étonnant, car c'est une logique très expressive. 
Arriver à trouver un algorithme permettant de traiter la logique des prédicats était une sorte de Graal au 20ème siècle. 
\section{3 transformations} 
On va commencer par faire les transformations de normalisation. Il y a 3 transformations à effectuer: 
\begin{enumerate}
    \item formule $\to$ forme prénexe: $$(\ldots{}\forall{}\ldots{}\exists{}\ldots{}\forall{} )\implies \forall{}\exists{}\forall{}(\ldots{})$$
    Tous les quantificateurs sont mis en tête de la formule. Les quantificateurs étant très compliqués à gérer, on transforme la formule pour les extraire de celle-ci. \\Les modèles sont conservés durant cette transformation.
    \item forme prénexe $\to$ forme Skolem (élimination des $\exists{}$): $$ \forall{}\exists{}\forall{}(\ldots{})  \implies \forall{}\forall{}\forall{}(\ldots{}) $$
    Les quantificateurs existentiels sont très embêtants, car ils sont restrictifs. Ils disent qu'il existe des éléments, mais ne précisent pas lesquels, on va donc les éliminer. \\
    Cette transformation préserve l'existence des modèles, mais pas les modèles eux-mêmes. Ils doivent être modifiés pour conserver la même signification. 
    \item forme Skolem $\to$ forme normale conjonctive: 
    \[\forall \ldots \forall \land_i(\lor_j L_{ij})\]

Cette transformation est la même que celle effectuée dans la logique des propositions. \\
Les modèles sont préservés lors de cette transformation. 
\end{enumerate}

\section{Résolution}
En logique des propositions: $$\frac{L\lor C_1, \neg L \lor C_2}{C_1\lor C_2}$$
Cette technique fonctionne en logique des propositions, car il n'y a pas de variables, mais ici, on peut avoir $L_1 \lor C_1$ \hspace{10pt} $\neg L_2 \lor C_2$  avec $L_1$ et $L_2$ qui ont des variables différentes. Par exemple P(x,a) et P(y,z).\\
Pour pouvoir faire la résolution, il va falloir en quelque sorte les rendre identiques. \\
On va donc dire: ce ne sont peut-être pas toujours les mêmes, mais, pour certaines valeurs, ils sont identiques. Si x=y et a=z alors on peut faire la résolution. 
\subsection{Unification}
\begin{minipage}{0.25\textwidth}
		L1:  P(x,a)\\
		L2:  P(y,z)
\end{minipage}
\begin{minipage}{0.75\textwidth}
		Pour que $L_1$ et $L_2$ soient identiques, on va restreindre les variables et faire une substitution.
\end{minipage}

$$(a: constante,\ x,yz:variables)$$
$$\sigma=\{(x,y),(z,a)\}$$
$$P(x,a) \to P(y,a)$$
$$P(y,z) \to P(y,a)$$

Cette résolution marche pour toutes les valeurs qui sont limitées par la substitution. Le résultat ne sera donc pas général.
Cette opération s'appelle l'unification et utilise la substitution $\sigma$ (sigma). 
On peut maintenant faire la résolution en appliquant  le même algorithme de réfutation que pour la logique des prédicats: 

$$\frac{L_1C_1, \neg L_2 \lor C_2}{(C_1 \lor C_2)\sigma}$$

\section{Propriétés de cet algorithme}
\begin{itemize}
\item Cet algorithme est moins fort que pour la logique des propositions, car la logique des prédicats est beaucoup plus expressive.
\item  adéquat: Si $B\vdash T$ $\to$ $B \models T$\\
\textit{si l'on trouve une preuve de T avec les axiomes B alors T sera vrai dans tous les modèles de B}
\item complet: Si $B \models T$ $\to$ $B\vdash T$\\
\textit{Si quelque chose est vrai dans tous les modèles alors on va trouver une preuve}
\item L'algorithme possède les propriétés d'un algorithme semi-décidable:
\begin{itemize}
\item Si $B \vdash T$ $\to$ l'algorithme trouve une preuve. 
\item Si $B  \nvDash T$ $\to$ il peut tourner en rond indéfiniment.
\end{itemize}
\textit{Si ce qu'on tente de prouver est vrai dans tous les modèles, l'algorithme va finir par trouver une preuve, mais si ce n'est pas vrai, l'algorithme va tourner en rond et ne jamais se terminer. Le problème est donc que quand l'algorithme prend trop de temps à trouver une preuve, on doit l'arrêter et l'on n'est jamais certain du résultat. On ne peut jamais être sûr que l'algorithme n'aurait pas trouvé une preuve si on l'avait laissé tourner plus longtemps. Il est donc semi-décidable, car ses résultats ne sont totalement fiables que dans le cas ou une preuve est trouvée. }
\end{itemize}
\section{Transformation de la formule de base vers la forme prénexe}

Etapes de la transformation en formule logiquement équivalente:
\begin{enumerate}
    \item Éliminer $\Leftrightarrow$ et $\implies{}$
    \item Renommer les variables. 
    \begin{itemize}
    \item Chaque quantificateur ne porte que sur une variable, il faudra en créer de nouvelles si besoin en prenant soin de conserver l'équivalence de la formule . 
    \item Attention: ne jamais garder le même nom de variable pour une variable libre et une variable liée. 
    \item Supprimer les quantificateurs si possible. \\
    \end{itemize}
    \item Migrer les négations ($\neg$) vers l'intérieur, vers les prédicats. On peut faire cela, car $\neg\exists$ peut être transformé en $\forall\neg$ et vice versa.  
    \item On peut mettre tous les quantificateurs de la logique des prédicats à l'avant de la formule. 
\end{enumerate}
\subsection{Exemple d'une transformation en forme prénexe}
\begin{enumerate}
\item $\forall x  [p(x) \land \neg (\exists y) \forall x ( \neg q(x,y)) = > \forall z \exists v \bullet p(a,x,y,v))]$\\ \textit{Expression de base}
\item $\forall x [p(x) \land \neg(\exists y) (\forall x) ($\colorbox{lightgray}{$\neg$} $\neg q(x,y) $\colorbox{lightgray}{$\lor$} $\forall z \exists v \bullet r(a,x,y,v)]$\\\textit{Suppression des $\implies{}$} 
\item $\forall x [p(x) \land \neg(\exists y) (\forall $\colorbox{lightgray}{$u$}$)(\neg\neg q($\colorbox{lightgray}{$u$}$,y) \lor $\colorbox{lightgray}{$\cancel{\forall z}$}$\exists v \bullet r(a,$\colorbox{lightgray}{$u$}$,y,v)]$\\\textit{Renommage des variables et suppression des quantificateurs inutiles}
\item $\forall x [p(x) \land$ \colorbox{lightgray}{$\forall y \neg$} $ (\forall u)($ \colorbox{lightgray}{$\cancel{\neg \neg}$} $q(u,y) \lor \exists v \bullet r(a,u,y,v)] $\\$\neg\exists y\ devient\  \forall \ y\ \neg \ et\ simplification\ des\ \neg$
\item $\forall x [p(x) \land \forall y $ \colorbox{lightgray}{$\exists u \neg$}$(q(u,y) \lor \exists v \bullet r(a,u,y,v)] $\\$\neg\forall u\ devient\  \exists \ u\ \neg$
\item $\forall x [p(x) \land \forall y \exists u $\colorbox{lightgray}{$(\neg q(u,y) \land \neg (\exists v) \bullet r(a,u,y,v))]$}\\ \textit{Distribution des $\neg$}\textit{(De Morgan)}
\item $\forall x [p(x) \land \forall y \exists u (\neg q(u,y) \land $\colorbox{lightgray}{$(\forall v) \bullet \neg$}$ r(a,u,y,v))]$\\$\neg\exists y\ devient\  \forall \ y\ \neg$
\item $\forall x$\colorbox{lightgray}{$ \forall y \exists u \forall v  \bullet$}$ [p(x) \land (\neg q(u,y) \land  \neg r(a,u,y,v))]$\\\textit{Extraction des quantificateurs.}
\end{enumerate}

  \chapter*{Algorithme de preuve pour la logique des prédicats}

\chapter{Introduction}
\section{3 transformations}
\section{Résolution}
\section{Propriétés de l'algorithme}

\chapter{Transformation en forme prénexe}

\chapter{Transformation en forme Skolen}
\section{Intuition}

Cette transformation consiste à éliminer toutes les occurences de quantificateurs existentiels.
\smallskip

$(\forall x)(\forall y)(\exists u)(\forall v) \big[ P(x) \wedge \neg Q(u,y) \wedge \neg R(a,u,y,v) \big]$
\smallskip 


Dans ce cas-ci, la valeur de $u$ de dépend des valeurs de $x$ et $y$. Lorsqu'on a choisi $x$ et $y$, on est alors libres de choisir $u$.
On peut donc supposer qu'une fonction $g(x,y)$ fournit cet élément de façon à conserver la satisfaisabilité de la formule tout en supprimant $(\exists u)$.
\smallskip

$(\forall x)(\forall y)(\forall v) \big[ P(x) \wedge \neg Q(\textbf{g(x,y)},y) \wedge \neg R(a,\textbf{g(x,y)},y,v) \big]$
\smallskip

Après la transformation, l'existence des modèles est préservée.

\section{Règle}

Pour chaque élimination d'un quantificateur existentiel $(\exists x)$, on remplace sa variable quantifiée par une fonction $f(x_1,...,x_n)$ dont les arguments sont les variables des quantificateurs universels dont $x$ est dans la portée.
\smallskip

\underline{Justification par un exemple:}

$p: \forall x \forall y \exists z \big[ \neg P(x,y) \vee Q(x,z) \big]$
\smallskip

$p_s: \forall x \forall y \big[ \neg P(x,y) \vee Q(x,\textbf{f(x,y)}) \big]$
\smallskip

Les modèles de p $\neq$ modèles $p_s$.

\underline{Pour $p$}
\begin{itemize}
  \item Interprétation $I$
  \item $D_I =$ Professeur $\cup$ Université
  \item $Val_I(P) = P_i = $ ``a enseigné à l'université''
  \item $Val_I(Q) = Q_i = $ ``est diplômé de l'université''
  \item $Val_I(f)$ n'existe pas.
\end{itemize}

\vspace{5 mm}

\underline{Pour $p_s$}
\begin{itemize}
  \item Etendre $I$
  \item $I' = \big\{ F \vdash F_i \big\} \circ I$
  \item $f(a,b) = $ ``l'université ayant dû diplômer $a$ pour que $a$ puisse enseigner à $b$''
\end{itemize}

\vspace{5 mm}
$p$ admet un modèle $(I)$ SSI $p_s$ admet un modèle $(I')$.

Que ça ne soit exactement le même modèle ne pose pas de problème pour notre algorithme. L'algorithme par réfutation continue à itérer jusqu'à trouver une contradiction ($false$). S'il n'y a pas de modèle pour $p_s$, il n'y a pas de modèle pour $p$ et ça suffit.

\chapter{Transformation en forme normale conjonctive}

Même manipulations qu'en logique des propositions.

\chapter{La règle de résolution}

$$ \mbox{\huge $\frac{L_1 \vee C_1, \neg L_2 \vee C_2}{(C_1 \vee C_2)\sigma}$ } $$

Cette règle de résolution ne fonctionne que si $L_1$ et $L_2$ sont identiques.

\begin{itemize}
  \item $L_1 = P_1(a, y, z)$
  \item $L_2 = P_1(x, b, z)$
\end{itemize}

Dans un modèle il y a un prédicat qui correspond au symbole $P_1$ et il y a un ensemble de triplets qui rendent vrai $P_1$.

L'unification de $L_1$ et $L_2$ donne L qui représente l'intersection des deux ensembles. On écrit:
\begin{itemize}
  \item $L_1 \sigma = L$
  \item $L_2 \sigma = L$
\end{itemize}

\vspace{5 mm}
$L_1$ et $L_2$ sont unifiables s'il existe une substitution $\sigma$ telle que $L_1 \sigma = L_2 \sigma$.
\begin{itemize}
  \item $\sigma = \big\{(x,a),(y,b)\big\}$
  \item $L_1 \sigma = P(a,b,z)$
  \item $L_2 \sigma = P(a,b,z)$
\end{itemize}

\vspace{5 mm}
\underline{Exemple:}
\begin{itemize}
  \item $L_1 = P_1(a,x)$
  \item $L_2 = P_2(b,x)$
\end{itemize}

Il n'y pas de substitution qui existe car on a deux constantes différentes, l'intersection des deux ensembles est vide.
\smallskip

\underline{Exemple 2:}

\begin{itemize}
  \item $L_1 = P(f(x), z)$ 
  \item $L_2 = P(y, a)$
\end{itemize}

Dans ce cas-ci, il y a beaucoup de subtitutions possibles telles que:
\begin{itemize}
  \item $\sigma_1 : \big\{ (y, f(a)), (x,a), (z,a) \big\}$ => $L_1 \sigma_1 = L_2 \sigma_1 = p(f(x), a)$
  \item $\sigma_2 : \big\{ (y, f(x)), (z,a) \big\}$ \hspace{11 mm}=> $L_1 \sigma_2 = L_2 \sigma_2 = p(f(x), a)$
\end{itemize}
Dans ce cas-ci, $\sigma_2$ est plus général.\\

\smallskip
\begin{itemize}
  \item On préfère alors la substitution $\sigma$ \underline{la plus générale}.
  \item On peut démontrer qu'il existe \underline{un} unificateur plus général U.P.G.
  \item U.P.G. est calculable.
\end{itemize}

\vspace{5 mm}
\textbf{\underline{Règle de résolution:}}
\begin{itemize}
  \item $p_1, p_2$ clauses
  \item $p_1 = L^{+} \vee C_1$
  \item $p_2 = \neg L^{-} \vee C_2$
  \item $ L^{+}$ et $L^{-}$ ont les mêmes symboles de prédicat.
  \item $ \big\{L^{+}, L^{-}\big\}$ unifiable par $\sigma$ U.P.G.
\end{itemize}

\underline{Alors} $$ \mbox{\huge $\frac{L^{+} \vee C_1, \neg L^{-} \vee C_2}{(C_1 \vee C_2)\sigma}$ } $$

\chapter{Algorithme}

$S := \big\{ \Delta x_1, ..., \Delta x_i, \neg Th \big\}$ dont chaque formule est en forme normale conjonctive (FNC). 


\underline{while}: \textbf{false} $\ni S$ et il existe une paire de clauses résolvables et non-résolues.

\underline{do}: choisir $p_i, p_j$ dans S et L tel que:
\begin{itemize}
  \item $L^{+}$ dans $p_i$ 
  \item $L^{-}$ dans $p_i$ 
  \item $\big\{ L^{+}, L^{-} \big\}$ unifiable par $\sigma$ U.P.G.
\end{itemize}

Calculer:
\begin{itemize}
  \item $r:=(p_i - [L^{+}] \vee p_j - [\neg L^{-}]) \sigma$
  \item $S_i = S \cup  \big\{ r \big\}$
\end{itemize}

\underline{end}\\

\underline{If} \textbf{false} $\ni S$ \underline{then} Th prouvé. \underline{else} Th non prouvé. \underline{end}

\chapter{Exemple}
\begin{itemize}
  \item $(\forall x) homme(x) \wedge fume(x) \Rightarrow mortel(x)$
  \item $(\forall x) animal(x) \Rightarrow mortel(x)$
  \item $homme(Ginzburg)$
  \item $fume(Ginzburg)$
  \item \textbf{Candidat-théorème:} $mortel(Ginzburg)$
\end{itemize}

\section{Initialisation de S}
\begin{itemize}
  \item P1: $(\forall x) homme(x) \wedge fume(x) \Rightarrow mortel(x)$
  \item P2: $(\forall x) animal(x) \Rightarrow mortel(x)$
  \item P3: $homme(Ginzburg)$
  \item P4: $fume(Ginzburg)$
  \item P5: $\neg mortel(Ginzburg)$
\end{itemize}

\section{Itérations}

\textbf{\underline{A}}
\begin{itemize}
  \item P1 + P5
  \item $\sigma = \big\{ (x, Ginzburg) \big\}$
  \item $r = P6 = \neg homme(Ginzburg) \vee \neg fume(Ginzburg)$
\end{itemize}

\textbf{\underline{B}}
\begin{itemize}
  \item P3 + P6
  \item $\sigma = \big\{ \big\}$
  \item $r = P7 = \neg fume(Ginzburg)$
\end{itemize}

\textbf{\underline{C}}
\begin{itemize}
  \item P4 + P7
  \item $\sigma = \big\{ \big\}$
  \item $r = false.$ Inconsistence donc le candidat théorème est prouvé.
\end{itemize}

\section{Non-déterminisme}

Importance des \underline{choix} qu'on fait.

\textbf{\underline{A}}
\begin{itemize}
  \item P2 + P5
  \item $\sigma = \big\{ (x, Ginzburg) \big\}$
  \item $r = \neg animal(Ginzburg)$
\end{itemize}

Si on avait fait ce choix-ci pour la première itération, l'algorithme ne peut plus continuer et on doit faire marche-arrière.

\chapter{Stratégies}

\begin{itemize}
  \item Quelles paires $p_i, p_j$ choisir?
  \item Quelles $L^{+}, L^{-}$ choisir?
\end{itemize}

Les assistants de preuves utilisent des stratégies existantes, avec l'input de l'humain.

Le language \textbf{Prolog}, inventé en 1972 par Alain Colmerauer et Robert Kowalski, utilise \underline{volontairement} des stratégies naives qui permettent de rendre l'algorithme prévisible. Les axiomes deviennent un \underline{programme}, c'est la programmation logique. Un exemple de stratégie naïve est la stratégie LUSH qui choisi de haut vers le bas les paires $p_i, p_j$, et de gauche à droite dans $p_i$.

  \chapter{Théorie logique}
\section{Étude des structures discrètes}
\noindent
\textbf{Exemples de structures discrètes} : entiers positifs, chaînes, arbres, ensembles, relations, fonctions ...\\

On peut définir ces structures avec la logique des prédicats et faire des raisonnements sur celles-ci en utilisant des règles d'inférence.\\

\noindent
\textbf{Utilisation} :\begin{itemize}
\item On peut programmer avec ces structures. \textit{Prolog} permet d'écrire les axiomes directement. Il faut cependant faire attention de bien les choisir ;
\item On peut les utiliser dans les assistants de preuve (exemples d'assistants de preuve : \textit{Coq, Isabelle})\\
\end{itemize}

\section{Théorie du premier ordre}
\noindent
\textbf{Intuition} : définition logique d'une structure mathématique

\noindent
\subsection{Définition d'une théorie}
\begin{itemize}
\item Sous-langage de la logique du premier ordre
\begin{itemize}
\item \textbf{vocabulaire} : constantes, fonctions, prédicats ;
\item règles syntaxiques et sémantiques sur ce vocabulaire ;
\end{itemize}
\item Ensemble d'axiomes (formules fermées, c'est-à-dire formules ne contenant pas de variables libres);
\item Ensemble de règles d'inférence.
\end{itemize}

\subsection{Exemple : théorie des liens familiaux (\bsc{fam})}
\begin{enumerate}
\item Vocabulaire : 
\begin{itemize}
\item 2 symboles de fonctions : $p/1$, $m/1$
\item 3 symboles de prédicats : $P/2$, $GM/2$, $GP/2$\\
\end{itemize}

On peut interpréter les fonctions $p$ et $m$ comme "père de" et "mère de", et les prédicats $P$, $GM$ et $GP$ comme "parent de", "grand-mère de" et "grand-père de". On donnera plus de précisions sur cette interprétation par la suite.\\

\item Axiomes : \\
\begin{center}
\begin{tabular}{lcc}
$(\forall x) \left(P(x,p(x))\right)$ & \hspace*{2cm}& (père)\\
$(\forall x) \left(P(x,m(x))\right)$ & \hspace*{2cm}& (mère)\\
$(\forall x)(\forall y) \left(P(x,y)\Rightarrow GP(x,p(y)) \right)$&&\\
$(\forall x)(\forall y) \left(P(x,y)\Rightarrow GM(x,m(y)) \right)$&&\\
\end{tabular}
\end{center}

\item Règles : \\
Les règles sont uniquement celles de la logique des prédicats.
\end{enumerate}
\subsubsection{Première interprétation}
\begin{tabular}{lll}
$D_I$ : personnes&&\\
$\text{val}_I(p)=\text{"père de"}$ &\hspace*{1cm} &$\text{père de} : \text{Pers}\rightarrow\text{Pers} : d \rightarrow \text{"père de" }d$\\
$\text{val}_I(m)=\text{"mère de"}$ &\hspace*{1cm} &$\text{mère de} : \text{Pers}\rightarrow\text{Pers} : d \rightarrow \text{"mère de" }d$\\
$\text{val}_I(P)=\text{"Parent"}$ &\hspace*{1cm} &$\text{Parent}(d_1,d_2)=T$ ssi $d_2$ est un parent de $d_1$\\
$\text{val}_I(GP)=\text{"Grand-père"}$ &\hspace*{1cm} &$\text{Grand-père}(d_1,d_2)=T$ ssi $d_2$ est un grand-père de $d_1$\\
$\text{val}_I(GM)=\text{"Grand-mère"}$ &\hspace*{1cm} &$\text{Grand-mère}(d_1,d_2)=T$ ssi $d_2$ est une grand-mère de $d_1$\\
\end{tabular}\\

Cette interprétation est un modèle de \bsc{fam} car les axiomes sont tous vérifiés.\\
On remarque la ressemblance avec une théorie scientique, où les axiomes correspondent à la théorie et l'interprétation à ce que celle-ci signifie dans le monde réel.\\
Une théorie peut avoir plusieurs modèles.\\ 
\subsubsection{Seconde interprétation}
Cette interprétation est également un modèle de \bsc{fam}.\\

\begin{tabular}{lll}
$D_J$ : $\mathbb{N}$&&\\
$\text{val}_J(p)="p_J"$ &\hspace*{1cm} &$p_J : \mathbb{N}\rightarrow\mathbb{N} : d \rightarrow 2d$\\
$\text{val}_J(m)="m_J"$ &\hspace*{1cm} &$m_J : \mathbb{N}\rightarrow\mathbb{N} : d \rightarrow 3d$\\
$\text{val}_J(P)="P_J"$ &\hspace*{1cm} &$P_J(d_1,d_2)$ ssi $d_2=2d_1$ ou $d_2=3d_1$\\
$\text{val}_J(GP)="GP_J"$ &\hspace*{1cm} &$GP_J(d_1,d_2)$ ssi $d_2=4d_1$ ou $d_2=6d_1$\\
$\text{val}_J(GM)="GM_J"$ &\hspace*{1cm} &$GM_J(d_1,d_2)$ ssi $d_2=6d_1$ ou $d_2=9d_1$\\
\end{tabular}\\

\section{Propriétés des théories}
\begin{itemize}
\item Une formule fermée $p$ est \textbf{valide} dans la théorie $Th$ si elle est vraie dans chaque modèle de $Th$. On écrit : 
$$\vDash_{Th} p$$
Soit l'ensemble des axiomes $Ax=\{Ax_1, \hdots, Ax_n\}$. On a bien que $\vDash_{Th} Ax_i$.\\
\item $q$ est une \textbf{conséquence logique} de $p$ dans la théorie $Th$ si $q$ est vraie dans tous les modèles de $Th$ qui rendent $p$ vraie. On écrit :  
$$p \vDash_{Th} q$$
\item Une théorie est \textbf{consistante} si elle a au moins un modèle ($>0$ modèles).
\item Une théorie est \textbf{inconsistante} si elle n'a pas de modèle ($0$ modèles).\\
\end{itemize}

Comment peut-on faire pour établir $\vDash_{Th} p$? Il y a deux approches différentes :
\begin{enumerate}
\item \textbf{l'approche sémantique} : on prend un modèle quelconque de $Th$ et on évalue $\text{VAL}_I (p)$ en utilisant le fait que $\text{VAL}_I (Ax_i)=T$ ;
\item \textbf{l'approche syntaxique} : théorie de preuve : on essaye de construire une preuve de $p$ à partir des axiomes, en appliquant les règles de $Th$.\\
\end{enumerate}
En pratique, la deuxième approche est beaucoup plus souvent utilisée.\\


\subsubsection{Preuve (esquisse)}
La preuve qui suit est une esquisse. Il manque plusieurs étapes.\\
On veut montrer  : 
$$\vDash_{\text{\bsc{fam}}} (\forall x)(\exists z) GM(x,z)$$

\begin{tabular}{lll}
$(\forall x)(\forall y) \left(P(x,y)\Rightarrow GM(x, m(y)) \right)$&\hspace*{1cm}&($Ax$)\\
$(\forall x) \left(P(x,p(x))\Rightarrow GM(x, m(p(x))) \right)$&\hspace*{1cm}&(Elimination de $\forall y$ et substitution $y/p(x)$)\\
$\left( \forall x \hspace*{0.1cm} P(x,p(x)) \right)\Rightarrow \forall x \hspace*{0.1cm} GM(x,m(p(x)))$&\hspace*{1cm}&(Distribution $\forall/\Rightarrow$)\\
$\forall x \hspace*{0.1cm} GM(x,m(p(x)))$&\hspace*{1cm}&(Modus ponens)\\
$\forall x \hspace*{0.1cm} \exists y \hspace*{0.1cm} GM(x,y)$&\hspace*{1cm}&(Introduction de $\exists$)\\
\end{tabular}\\

\section{Qualité d'une théorie}
Certaines qualités sont directement issues de la logique:
\begin {enumerate}
\item {\textbf{consistante}} : il est impossible de déduire \(p\) et \(\neg p\)  de la même théorie.
\item {\textbf{minimale}} : les axiomes sont indépendants: 
$\{Ax_1, \hdots, Ax_n\}$   $\nvDash Ax_k $\\
\textbf{Vérification} : Construction d'interprétations.\\
$\text{VAL}_J (Ax_k)= False$\\
$\text{VAL}_J (Ax_i)= True$  
pour $i \neq k$
\item {\textbf{complète}} : les axiomes suffisent pour prouver la propriété d'intérêt. Sinon il faut en ajouter.
\end {enumerate}
\subsection{Exemple : qualité de deux théories}
\begin {enumerate}
\item {\textbf{Système de Copernic}} : Chaque axiome de chaque planète est indépendant, car elles tournent toutes autour du soleil.
\item{\textbf{Système de Ptolémée}} : Les axiomes de chaque planète dépendent de ceux de la Terre, car elle représente le centre de l'univers, mais elle est également une planète et dispose donc d'axiomes.
\end {enumerate}

  \section{Extension d'une théorie}
Lorsqu'on a une théorie déjà existante, on souhaite parfois l'étendre afin de la rendre plus complète. Pour ce faire, on ajoute des axiomes et on étend le vocabulaire. Voici deux exemples d'extension de théorie pour mieux comprendre comment effectuer cette opération. Ils étendent tous les deux la théorie des liens familiaux (FAM) décrite dans la section précédente.

\subsection*{Exemple 1}
Considérons le nouvel axiome suivant, que nous noterons Ax.
$$ (\forall x) \neg P(x,x) $$
La nouvelle théorie ainsi étendue que nous noterons FAM* possède un axiome de plus : Ax. Cette théorie FAM* est \textbf{consistante}, c'est-à-dire qu'il existe au moins un modèle qui valide cette théorie. Pour s'en convaincre, il suffit de considérer la première interprétation de la théorie FAM de la section précédente, qui utilise les liens familiaux. \\

En revanche, si on considère la deuxième interprétation (deuxième modèle, noté $J$) de FAM qui associe les symboles $p$ et $m$ aux fonctions mathématiques $p_J : \mathbb{N} \rightarrow \mathbb{N} : d \rightarrow 2d$ et $m_J : \mathbb{N} \rightarrow \mathbb{N} : d \rightarrow 3d$, on observe une contradiction. En effet, dans le modèle $J$ on a la définition suivante du prédicat $P$ :
$$ P_J(d_1, d_2) \textrm{ ssi } d_2 = 2d_1 \textrm{ ou } d_2 = 3d_1$$
Il suffit de choisir $x=0$ dans notre nouvel axiome Ax pour constater que le modèle $J$ ne valide pas la théorie étendue FAM*. De manière générale, l'extension d'une théorie peut donc réduire l'ensemble des modèles de celle-ci.

\subsection*{Exemple 2}
Considérons à présent le nouvel axiome suivant, que nous noterons Adam.
$$ (\forall y) \neg P(a,y) $$
où $a$ est une constante arbitraire. Notons la théorie étendue FAM' = FAM + Adam. Dans cet exemple, on peut observer que FAM' est \textbf{inconsistante,} car aucun modèle ne peut valider cette théorie. En effet, en partant du premier axiome de FAM (appelé "père"), nous effectuons quelques étapes pour obtenir une contradiction.
\begin{align*}
& (\forall x) P(x,p(x)) \\
\iff & P(a,p(a)) && \textrm{Elimination } \forall \\
\iff & (\exists y) P(a,y) && \textrm{Intro } \exists
\end{align*}
Ci-dessus, le premier axiome de FAM reformulé (père), qui est en contradiction avec le nouvel axiome (Adam), que nous reformulons ci-dessous.
\begin{align*}
& (\forall y) \neg P(a,y) \\
\iff & \neg (\exists y) P(a,y)
\end{align*}
Par la règle de preuve par contradiction, on démontre qu'aucun modèle n'est possible pour la théorie étendue FAM'. \'{E}tendre une théorie équivaut à faire de la manipulation syntaxique. Il est important de bien vérifier ce que notre modification a comme conséquence sur le nombre de modèles que la théorie accepte.

\section{Liens entre théories}
Dans cette section, nous abordons la comparaison de différentes théories : inclusion, équivalence et quelques corollaires ainsi que la théorie des ordres partiels stricts. Dans ce qui suit, on note $Th_1$ et $Th_2$ deux théories.

\subsection*{Inclusion}
On dit que $Th_1$ est \textbf{contenue} dans $Th_2$ si
\begin{itemize}
\item[$\bullet$] Le vocabulaire de $Th_1$ est inclus dans le vocabulaire de $Th_2$.
\item[$\bullet$] Toute formule valide dans $Th_1$ l'est aussi dans $Th_2$.
\end{itemize}
Attention, on peut donc avoir deux théories qui "parlent de la même chose" mais qui possèdent des axiomes totalement différents.
TODO : expliquer pourquoi / Ajouter un exemple

\subsection*{Équivalence}
On dit que $Th_1$ et $Th_2$ sont \textbf{équivalentes} si elles sont contenues l'une dans l'autre. Cela signifie que les deux théories "disent la même chose" et que tout modèle d'une des théories est également modèle de l'autre.\\

Il est important de bien faire la différence entre les \textbf{liens} entre les théories et l'\textbf{extension} d'une théorie. Le premier concept exprime ce que modélisent les théories tandis que le deuxième n'est que de la manipulation syntaxique.

\subsection*{Corolaires}
On note respectivement $V_i$, $M_i$ et $Ax$ le vocabulaire, les modèles et les axiomes d'une théorie $i$. \\

\noindent \underline{Si} $V_{Th_1} \subseteq V_{Th_2}$ et $M_{Th_2} \subseteq M_{Th_1}$ \underline{alors} $Th_1$ est contenue dans $Th_2$. \\

\noindent \underline{Si} $V_{Th_1} \subseteq V_{Th_2}$ et tout axiome de $Th_1$ est aussi axiome de $Th_2$ \underline{alors} $Th_1$ est contenue dans $Th_2$. \\

\noindent \underline{Si} $V_{Th_1} = V_{Th_2}$ et $\forall i,j \hspace{0.3cm} \vDash_{Th_2} Ax_{i,1} \textrm{ et } \vDash_{Th_1} Ax_{j,2} $ \underline{alors} $Th_1$ et $Th_2$ sont équivalentes. \\

\noindent \underline{Si} $p$ une formule fermée telle que $\vDash_{Th_1} p$ et $Th_2 = Th_1 \bigcup \left\lbrace p \right\rbrace$ \underline{alors} $Th_1$ et $Th_2$ sont équivalentes. \\

\section{Théorie des ordres partiels stricts}

Nous allons donner un autre exemple de théorie ainsi que deux interprétations différentes de cette théorie.
\subsection*{Vocabulaire}
\begin{itemize}
\item[$\bullet$] Le symbole P
\end{itemize}
\subsection*{Axiomes}
\begin{itemize}
\item[$\bullet$] $ (\forall x) \neg P(x,x) $  (irréflexivité) (OPS1)
\item[$\bullet$] $ (\forall x) (\forall y) (\forall z) (P(x,y) \wedge P(y,z) \Rightarrow P(x,z))$ (transitivité) (OPS2)
\end{itemize}
\subsection*{Première interprétation}
Soit l'interprétation I de cette théorie. On a :
\begin{itemize}
\item[$\bullet$] $D_I = \mathbb{Z} $
\item[$\bullet$] $val_I (P) = "<"$
\end{itemize}
Lorsque l'on écrit $val_I(<) = "<"$, le premier symbole $<$ est un mot de vocabulaire dans la théorie tandis que le deuxième est une fonction. Cette fonction confère un sens au symbole. 
On remarque que cette interprétation est un modèle de notre théorie. En effet, la fonction $<$ sur les entiers est irréflexive et transitive. Si on a $x < y$ et $y <z$, on peut en déduire que $x<z$.
\subsection*{Deuxième interprétation}
Soit l'interprétation J de cette théorie. On a : 
\begin{itemize}
\item[$\bullet$] $D_J = \mathbb{Z} $
\item[$\bullet$] $val_J (P) = "\ne"$
\end{itemize}
Ici on change le sens du symbole $<$ :
$$val_J(<) = "\ne"$$
Cette interprétation n'est pas un modèle. En effet, par exemple, pour $x=5, y=3, z=5$, on a :
$$5<3 \wedge 3<5 \nRightarrow 5<5$$
Ceci est un contre-exemple, car $5$ n'est pas différent de $5$ (irréflexivité), et donc cette interprétation ne vérifie pas l'axiome sur la transitivité!


  \section{Introduction à la programmation logique}
\label{p2}
\subsection{Introduction à la programmation logique}

Prolog est l’un des principaux langages de programmation logique. Il est à la base de nombreux fondements. 

La programmation logique fait de la déduction sur les axiomes.
On utilise la logique comme un langage de programmation : on va adapter l’algorithme de réfutation vu précédemment

Le programme (ressemble à une théorie):
\begin{itemize}
	\item Axiomes en logique des prédicats
	\item Une requête, un but (=goal)$\rightarrow$ le but du système est d'apporter une preuve
	\item Un prouveur de théorème $\rightarrow$ attention: il faut des conditions sur le prouveur car il faut être capable de prévoir le temps et l’espace utilisé par le programme.
\end{itemize}

\paragraph{}
Exécuter un programme = faire des déductions en essayant de prouver le but. Mais est-ce que cette idée peut donner un système de programmation pratique?

\paragraph{}
Il y a une compromis entre expressivité et efficacité: Si c’est trop expressif, ça devient moins efficace, par contre si c’est trop peu expressif, on ne peut rien programmer, ça ne sert à rien non plus. Il faut donc être expressif tout en restant efficace. Le Prolog offre un bon équilibre entre expressivité et efficacité.

\paragraph{}
Mais pour arriver à cela, il y a quelques problèmes à surmonter:
\begin{enumerate}\renewcommand {\theenumi }{\alph {enumi}}
	\item un prouveur est limité :\\
		vérité = $ p \models q $ (= $ q $ est vrai dans tous les modèles de $ p $)\\
		preuve = $ p  \vdash q $ \\
		$ p \models q \Rightarrow p \vdash q $ (= Si c’est vrai dans tous les modèles, on peut trouver une preuve)\\
		Si $ p \models q $ alors l’algorithme se terminera. Cependant, on ne peut pas trouver les preuves que pour des choses vraies dans tous les modèles. (Comme c’est impossible on ne prend qu’une partie des modèles, ce qui limite le programme).
		
		\item Même si on peut trouver une preuve, le prouveur est peut-être inefficace (utilise trop de temps ou de mémoire) ou imprévisible. $ \to $ On ne peut pas raisonner sur l’efficacité du prouveur.
		
		\item La déduction faite par le prouveur doit être constructive  
\end{enumerate}

		Si le prouveur affirme : $ (\exists X) P(X) $  alors le prouveur doit donner une valeur de x (c’est quoi $ x $).



Il faut construire un résultat.

\underline{Pour résoudre ces problèmes…}
\begin{enumerate}
	\item Restrictions sur la forme des axiomes.\\
	typiquement : 
	\begin{equation}
		(\forall X_{1}) ... (\forall X_{n}) A_{1} \wedge A_{2} \wedge ... \wedge A_{n} \Rightarrow A
	\end{equation}
	\begin{equation}
		C_{i} \neg A_{1} \vee \neg A_{2} \vee ... \vee \neg A_{n} \vee A
	\end{equation}
	Il n’y a qu’un seul littéral sans négation. (Pour prouver A, il faut prouver $A_{1}$; $A_{2}$, ..., $A_{n}$).\\
	$C_{i}$ est la clause. Le programme tout entier est une série de clauses :
	\begin{equation}
		C_{1} \wedge C_{2} \wedge ... \wedge C_{k}
	\end{equation}
	\item Le programmeur va aider le prouveur.
	Par exemple : il faut commencer par prouver $A_{1}$ puis $A_{2}$... dans cet ordre là.\\
	$\rightarrow$ Le programmeur donnera des heuristiques. Attention : ces heuristiques ne changent pas la sémantique \underline{logique du programme}. Elles ont seulement un effet sur l’efficacité !
	\begin{equation}
		C_{1} = \neg A_{1} \vee \neg A_{2} \vee ... \vee A_{n} \vee A
	\end{equation}
	\begin{equation}
		C_{2} = \neg B_{1} \vee \neg B_{2} \vee ... \vee B_{k} \vee A
	\end{equation}
\end{enumerate}

Le langage Prolog utilise ces 2 ordres.

\underline{Bref historique :}
\begin{enumerate}
	\item 1965 : 	La règle de résolution a été inventée par A. Robinson
	\item 1972 :	Invention du langage Prolog / premier interprète (de Prolog) par A. Calmerauer, R. Kowalski et Ph. Roussel. 
Ils voulaient faire un langage de programmation logique, et connaissaient les différentes formes de logique ainsi que la résolution. Ils ont donc inventé un langage très simple qu'ils ont appelé Prolog (pour Programmation Logique).  Il s'avère que ce langage a un compromis très intéressant par rapport à la tension entre efficacité et expressivité.
Aujourd'hui, on peut faire une implémentation extrêmement efficace de Prolog. Il est extrêmement expressif, ce qui permet de faire des programmes complexes. C’est un langage à part entière.
\end{enumerate}

\subsection{Introduction à Prolog}

En Prolog, on a des clauses (règles):
	\begin{equation}
		 A1 \leftarrow B1, … , Bn
	\end{equation}
	 (On peut prouver $ A $ en prouvant $ B1 $ jusqu’à $ Bn $. Remarque :$ \leftarrow $ou :- )
	 
	 \begin{equation}
	 	\neg (B_{1} \wedge ... \wedge B_{n}) \vee A_{1}
	 \end{equation}
	 \begin{equation}
	 	 (\neg B_{1} \vee \neg B_{2} \vee ... \vee \neg B_{n} \vee A_{1})
	 \end{equation}
	 
	 Programme = ensemble de clauses.
	 
	 \underline{Exemple d’un petit programme en Prolog:} (extrait du livre « The Art of Prolog » par L. Sterling et E. Shapine)
	 
	 Règle : $ grandpere(x, z) \leftarrow pere(x,y), pere(y,z) $. ($x$, $y$,… sont des variables. En Prolog, elles sont souvent en majuscule)
	 
	 \begin{table}[h]
	 	\begin{tabular}{ll}
	 		Faits: & $pere(terach, abraham)$ (terach, abraham,… sont des constantes) \\
	 		& $pere(abraham, isaac)$\\
	 		& $pere(haram, lot)$ \\
	 		& $…$ \\\\
	 		$ \to $ & Syntaxe clausale: $pere(terach,abraham) \wedge pere(abraham,isaac) \wedge  $ \\
	 		& $ pere(haram,lot) \wedge (\neg pere(x,y) \vee \neg pere(y,z) \vee grandpere(x,z))      $                                                         
	 	\end{tabular}
	 \end{table}
	 
	Il existe une corrélation évidente entre Prolog et les bases de données. Elles ont été inventées quasi au même moment et aujourd'hui Prolog est utilisé comme une sémantique pour les bases de données déductibles. Ici, on peut avoir une relation avec deux colonnes qui auraient l'argument père. le grand-père serait une combinaison de ces deux relations.
	 
	 Prolog peut être vu comme une sorte de base de données relationnelle mais beaucoup plus puissante : on peut faire des programmes qui sont plus que des simples requêtes, avec des calculs beaucoup plus complexes.
	 
\subsection{Algorithme d'exécution de Prolog}

Dans la version de l’algorithme de preuve par résolution, l’ensemble S grandit (ce qui n’est pas très efficace).

L’idée :
\begin{itemize}
	\item On commence par mettre le but (G) que l’on veut prouver dans r (sans négation).
	\item Ensuite, jusqu’à ce que r soit vide, on prend le premier littéral dans $r(A_{1})$.
	\item Puis, on parcourt un à un les axiomes de P (P est le programme, la base de faits) pour trouver une clause $Ax_{i}$ unifiable avec $A_{1}$ au moyen de $\sigma(u.p.g)$
	\begin{itemize}
		\item Si on trouve une telle clause, on ajoute à r les littéraux de $Ax_{i}$ après unification avec $A_{1}$ (et on recommence au début).
		\item Si on ne trouve pas de clause unifiable, on revient sur le dernier choix (Par exemple, pour un $A_{1}$ qui aurait plusieurs clauses unifiables, on a dû en choisir une. Et bien, on retourne en arrière pour en choisir une autre, sans oublier de modifier r. (En effet, il faut éliminer les résultats de toutes les unifications qui ont été réalisées entre le moment où le point de choix a été mémorisé et le moment du retour en arrière.))
		\item Si on épuise tous les choix sans que r soit vide alors nous sommes en cas d’échec.
	\end{itemize}
	\item Lorsque le programme s’arrête, si r est vide, on a un résultat.
\end{itemize}

Programme  : $P={Ax_{1}, ..., Ax_{n}}$

Un « but » (un goal, une « requête ») G ($\simeq$ théorie)


\begin{algorithm}[H]
$r := <G>$ … résolvante (une séquence de littéraux $\rightarrow$ $r= <A_{1},A_{2},...,A_{m}>$  Il n’y a qu’un seul r.)\\
\While{r est non vide}{
	\begin{itemize}
		\item Choisir un littéral $A_{1}$ dans r (on prend le premier littéral)
		\item Choisir une clause $Ax_{1}=(A \leftarrow B_{1},...,B_{k})$ dans P. (D’abord on prend la première clause, puis la suivantes jusqu’à ce qu’on trouve une clause unifiable avec $A_{1}$. Si aucune clause n’est unifiable on revient sur le dernier choix (backtrack))
		\item Nouvelle résolvante $:= <B_{1},..., B_{k}, A_{2},...,A_{m}> \sigma$
		\item $G^{'} = G \sigma$
	\end{itemize}
}
\If{r est vide}{le résultat est le dernier $G^{'}$}
\If{on épuise les choix sans que r soit vide}{le résultat est NON. ( On n’a pas prouvé G). (Attention : G est peut-être vrai, mais les heuristiques ne suffisent pas pour le prouver.)}
\end{algorithm}

On peut également avoir une boucle infinie (l’algorithme est non déterministe).

  
\author{Baptiste Degryse (27641200)}
\author{Charles Jacquet (27811200)}
\author{Jérôme Lemaire (6960100)}

\subsection{Exemples de programmes Prolog}

\paragraph{Programme (code Prolog)}
Ce petit programme permet de calculer la factorielle d'un nombre. Il s'agit de clauses exprimant des faits, des règles et des questions. Le code commence par un fait: 0!==1. Ensuite, il défini une clause pour les factorielles de façon généralisée. Attention, les virgules et les points sont importants. Les virgules sont les séparateurs entre les littéraux dits négatifs tandis que le point marque la fermeture de la clause.
\begin{verbatim}
fact(0,1)
fact(N,F) :- N>0, 
			 N1 is N-1 
			 fact(N1,F1), 
			 F is N* F1 .
\end{verbatim}
\paragraph{Requête} Nous allons maintenant exécuter le programme fact afin de trouver la factorielle de 5 et stocker la réponse dans la variable F. Le prompt Prolog est représenté par " |?- " et la sortie standard par "->". Les commentaire sont après les "\%" ou entre "/* */". Il ne faut pas oublier le point à la fin de la requête.
\begin{verbatim} 
|?- fact (5,F). %requête
-> F=120 % réponse
\end{verbatim}

\paragraph{Forme clausale}
Le programme réécrit sous forme clausale. 
\begin{align*}
fact(0,1) \\
 \wedge & \\
&( \neg n > 0 \\
&	\vee \neg minus( n1, n, 1) \\
&	\vee \neg fact(n1, f1) \\
&	\vee \neg times(f, n, f1) \\
&	\vee fact(n, f) ) \\
\end{align*}



\paragraph{Exécution} 
Le but de l'exécution est de prouver que $G \equiv fact(5, r)$. Pour y arriver, Prolog va faire une suite de substitutions $\sigma$ afin de vérifier les affirmations contenues dans r. La substitution finale contiendra le résultat final.

\begin{align*}
& G \equiv fact(5,r) \\
& r= < fact(5, r) > \\
& \% (1) \\
& \sigma = \{ (n, 5), (f, r) \} \\
& r= < ( 5>0 ), minus(n1, 5, 1), fact(n1, f1), times(r, 5, f1) > \sigma \\
& \% (2) \\
& r= < minus(n1, 5, 1), fact(n1, f1), times(r, 5, f1) > \\
& \sigma'= \{ (n1, 4) \cup \sigma\} \\
& r= < fact (4, f1), times(r, 5, f1) > \\
& [...]\\
& \sigma_{res} = \{(r,120),...\} 
\end{align*}
(1) avec clause 2\\
(2) a>b existe dans le système prédéfini -> true\\

\subsubsection{Exemple 2}
Le but de cet exemple est de faire un "append" de deux listes:\\
append(L1, L2, L2)
\paragraph{Prolog:}
\begin{itemize}
\item append ( [] , L , L)
\item append ([X|L1], L2, [X|L3]) :- append (L1,L2,L3)
\end{itemize}
\paragraph{Clausal}
Le programme réécrit sous forme clausale.\\
append(nil,l',l')\\
$\land$\\
($\neg$append($l_1,l_2,l_3$) $\land$ append (cons(x, $l_1$), $l_2$, cons(x,$l_3$))
\paragraph{Execution}
Attention, l'exécution du programme est une preuve en soit.\\
G = append (cons(1,nil), cons(2, nil), l)
\begin{itemize}
\item[1.] 
		$r$= < append (cons(1,nil), cons(2,nil, l)> et
		$\sigma_1$ = {(x,1), ($l_1$,nil), ($l_2$, cons(2,nil)), (cons(x,$l_3$),l)}
\item[2.]
		r=< append(nil,cons(2,nil),$l_3$)> et
		$\sigma_2$ = { (l', cons(2, nil), ($l_3$,l') }
		
\item[3.]
		r=<>
		
\item[Résultat]
l = con (1, $l_3$)
$l_3$ = l'
l' = cons(2,nil)
l=con(1,cons(1,nil))


\end{itemize}

\subsubsection{Exemple 3}

Intéressons-nous à la manière dont prolog exécute la requête \textbf{|?- append (L1,L2,[1])} en nous basant sur la syntaxe définie lors de l'exemple 2.\\
Pour information, la fonction cons prend comme premier argument une valeur et comme second argument une liste. Elle place le premier argument en tête du second argument.\\

\paragraph{Requête}
\begin{verbatim} 
|?- append (L_{1},L_{2},[1]). %requête
-> r = <append (l_{1}',l_{2}',cons(1,nil))> % réponse

		L_{1} = [],
		L_{2} = [1]. %clause 1
ou		
		L_{1} = [1],
		L_{2} = []. %clause 2
\end{verbatim}

\paragraph{Forme clausale}
Le programme réécrit sous forme clausale.\\
append(nil,l',l')\\
$\land$\\
($\neg$append($l_{1},l_{2},l_{3}$)
$\vee$ append (cons(x, $l_{1}$), $l_{2}$, cons(x,$l_{3}$))

\paragraph{Exécution}
Dans cette exemple, il y a plusieurs chemins pour arriver au résultat
\begin{itemize}
\item[] Si on choisi la clause 1: 
\begin{align*}
&\sigma = {(l_{1}',nil), (l_{2}',l'), (l',cons(1,nil))} \\
&r= < > \\
& l_{1}'=nil \\
& l_{2}'=cons(1,nil) \\
\end{align*}

\item[] Si on choisi la clause 2: 
\begin{align*}
&\sigma = {(l_{1}',cons(x,l_{1})), (l_{2}',l_{2}), (x,1), (l_{3},nil)} \\
&r= < append(l_{1},l_{2}',nil) > \\
&\sigma = {(l_{1},nil), (l_{2},l_{2}'), (nil,l_{3})} \\
&r= < > \\
& l_{1}'=nil \\
& l_{2}'=cons(1,nil) \\
\end{align*}

\end{itemize}


Il existe deux manières de percevoir une exécution Prolog.\\
\begin{itemize}
\item[1] Approche "impérative" : l'exécution est vue comme une séquence de calculs
\item[2] Approche logique : l'exécution est vue comme une preuve. Cette approche n'existe pas dans les langages classiques, c'est ce qui fait la force de Prolog.
\end{itemize}
Prolog est présent dans beaucoup de domaine de programmation :\\
\begin{itemize}
\item Dans la gestion de base de données avec Datalog
\item Dans la semantic web
\item Dans la programmation par contrainte. Les substitutions sont remplacées par des relations quelconques.
\end{itemize}


  \part{Structures discrètes sur Internet}
  \chapter{Structures discrètes sur l'internet}

\section{Ressources}
Livre : sur iCampus 
\begin{itemize}
\item Chapitre 1-5 (Graphe = modèle des réseaux) définitions, concepts (liens forts et faibles), contexte des réseaux, relations positives et négatives
\item Chapitre 13 (Structure du Web)
\item Chapitre 14 (Analyse des liens sur le web)
\item Chapitre 18 (Lois de puissances)
\item Chapitre 20 (Les phénomènes de petit monde)
\end{itemize}
\section{Exemple et analyse de graphes}
Voici quelques commentaires réalisés sur les figures du $1^{er}$ chapitre :
\begin{itemize}

	\item \textbf{Figure 1.1} (ci dessous : Figure \ref{karate} Illustration des relations amicales entre 34 personnes dans un club de karaté.
Chaque nœud représente une personne et chaque lien, un lien d'amitié entre ces personnes.
On constate que tout le monde n'est pas ami avec tout le monde. Grâce à cette structure, on peut déduire certaines choses, par exemple : deux personnes ont beaucoup de liens d'amitié avec d'autres personnes mais pas entre eux.\\
\begin{figure}[!h]
\centering
\includegraphics[scale=1]{images/17_karate_club.jpg}
\label{karate}
\caption{Relations amicales dans un club de karaté}
\end{figure}

	\item \textbf{Figure 1.2} Des employés dans un laboratoire de recherche (nœud) ont des liens entre eux :
Lignes claires = communication e-mail.
Lignes foncées = hiérarchie, organisation du laboratoire.
On voit que la communication entre les gens suit relativement bien la structure hiérarchique, mais pas complètement. On peut voir comment les gens collaborent, leurs degrés de collaboration, etc... \\

	\item \textbf{Figure 1.3} On constate que dans cette illustration, il y a beaucoup plus de nœuds. Chaque nœud est une institution financière (banque par exemple). Chaque nœud a un chemin vers un autre nœud (il y a des liens entre toutes les banques). Le centre est très dense, ça montre une faiblesse du système financier : si une banque dans le centre fait faillite par exemple, toutes les autres banques liées à elle sont également mises en danger . Donc si le noyau central est trop grand, c'est une faiblesse. En regardant cette structure, on peut trouver les faiblesses et les comprendre. Ca peut être très important.\\

	\item \textbf{Figure 1.4} Un nœud représente un blog politique et un lien, une référence vers un autre blog. Nous avons deux partis qui représentent chacun un noyau : les démocrates et les républicains. On constate qu'il y a moins de connexions entre les deux noyaux qu'à l'intérieur de ceux-ci. On peut visualiser cette structure et se poser des questions : est-ce que ce monde bipolaire est un problème ?\\\
\end{itemize}

Ce sont divers exemples que nous allons essayer d'analyser. Sur Internet, il y a beaucoup de nœuds avec de grandes capacités de calcul et de stockage. On peut maintenant regarder ces structures (pas avant). 
\section{Introduction}
\begin{itemize}
\item Structure des réseaux (Facebook, Twitter, réseau économique,...).
\item Comportement des participants (interactions : chaque nœud sera un participant et va interagir).En principe chaque nœud ne voit que son voisinage et interagit en conséquence.
\begin{itemize}


	\item Interactions LOCALES avec conséquence GLOBALES.
	Il faut faire le lien entre ces deux choses.
	\item Effets non attendus. Ex: réseaux routiers (nœud = automobilistes, lien = routes) : \\
	S'il y a des bouchons, on augmente la capacité du réseau (ajouter une voie par exemple)
	Le résultat peut être non intuitif, ça peut être :
	\begin{itemize} 
		\item Une réduction des transferts.
		\item Une réduction du trafic. (résultat contre celui attendu)
	\end{itemize}	
	\item Le \textbf{Paradoxe de Braess} nous dit que l'ajout d'une nouvelle capacité à un réseau peut 	réduire la performance globale (effet non attendu).
Il faut donc comprendre comment le réseau fonctionne au lieu de faire n'importe quoi et avoir des effets non attendus.
\end{itemize}
\end{itemize}
\section{Nouvelle discipline}
Les graphes et leurs propriétés évoluent avec le temps, ce n'est pas statique.\\
$\Rightarrow$Nouvelles disciplines pour analyser des graphes Youtube, Flicker, etc...\\
Synthèse de 3 disciplines :
\begin{enumerate}

	\item La théorie des graphes => mathématique
	\item La théorie des jeux => mathématique
		Exemple : Youtube impose ses règles et ceux qui utilisent Youtube sont des joueurs.
	\item La sociologie (étude des groupes sociaux) Les participants sont humains ou guidés par  un 	    humain. Ce n'est pas simplement des maths, il faut aussi comprendre les humains.\\
\end{enumerate}
Dans ce cours, nous nous concentrerons principalement sur la théorie des graphes. La théorie des jeux sera très intuitif, et nous parlerons un peu de la sociologie.
\subsection{Théorie des jeux}
On a un ensemble de participants qui jouent à "un jeu" (un ensemble de règles suivies par tous les participants). Chaque participant doit agir : 
\begin{itemize}

\item Simple à spécifier (comme les échecs : 2 participants et 1 action en alternance). 
\item Compliqué : pas d'alternance, tout le monde agit en même temps. C'est un système concurrent.
\end{itemize}
Exemple d'action simple : la vente aux enchères : 
	n participants,
	règles simples (différentes techniques) \\
	
On va rester intuitif sur ce sujet, mais si on veut être plus précis, il y a des mathématiques pour ça.
\begin{itemize}

\item \textbf{Figure 1.8} Réseau d'interaction économique entre pays. Structure de l'économie mondial : Honkong a un gros avantage, il a une porte d'entrée vers la Chine (à l'époque). Certains pays sont des partenaires privilégie des Etats-Unis,... 
\item \textbf{Figure 1.9} Chemins de commerces médiévaux en Europe. L'endroit comporte des avantages : la position dans le graphe. On a toute une série d'avantages qui viennent de la structure du réseau (le comportement d'un participant peut dépendre de la structure). Si on est malin et qu'on comprend le réseau dans lequel on est, on peut essayer de se mettre dans une structure ou on a plus de pouvoir. 
\end{itemize}
Si on connaît la structure, une petite action peut suffire pour arrêter une épidémie.

\subsection{Théorie des graphes}
\subsubsection{Définitions\\}
Graphe G = ensemble de liens et de nœuds. Un lien est une paire de deux nœuds.
\begin{center}
\textbf{
G = (N,E)\\
N = nœud\\
E = edge (lien, arête)\\
}
\end{center}
Deux types de graphes :
\begin{itemize}

\item \textbf{Les graphes orientés} (avec flèches et principe de direction). Une arête/lien est une paire de nœuds. L'ordre a de l'importance. Dans cet exemple, les paires de nœuds sont : (A,B), (A,C), (D,A) et (C,D). 
\begin{figure}[!h]
\centering
\includegraphics[scale=1]{images/17_oriente.png}
\caption{Graphe orienté}
\end{figure}
\item \textbf{Les graphes non orientés} (sans flèche, sans direction). Une arête est un ensemble de deux nœuds. On ne parle pas de "paire" vu que l'ordre des nœuds n'a pas d'importance. Ici, parler de l'arête (A,B) ou (B,A) revient à la même chose.
\begin{figure}[!h]
\centering
\includegraphics[scale=1]{images/17_non-oriente.png}
\caption{Graphe non orienté}
\end{figure}
\\
\end{itemize}
Dans le cours, on aura surtout affaire à des graphes non orientés.

\subsubsection{Notions de chemins et de connectivité}
\begin{itemize}

\item \textbf{Chemin} : séquence (un ordre) de nœud. Chaque paire consécutive est une arête. Un chemin peut passer plusieurs fois par la même arête (boucle), ce qui n'est pas toujours ce qu'on veut.
\item \textbf{Chemin simple} : chaque nœud est au maximum une fois dans la séquence. On ne peut plus faire le tour plusieurs fois.
\item \textbf{Cycle} : chemin qui arrive au même endroit d'où il est parti. 
	\begin{itemize}
			
	\item Le premier et le dernier nœud sont les mêmes. 
	\item Le chemin a au moins 3 liens (puisqu'on ne peut pas passer plusieurs fois par la même 	arête). 

\end{itemize}
\end{itemize}
  \section{Théorie des Graphes}


\subsection{Chemins et connectivité}
Nous allons commencer par rappeler la notion de chemin:
	\begin{description}
	\item[Chemin]: c'est une séquence de n\oe uds dont chaque paire consécutive est une arrête.
    \item[Chemin simple]: c'est un chemin dont chaque n\oe ud se trouve au maximum une fois dans la séquence de n\oe uds.
    \item[Cycle]: c'est un chemin dont le premier et le dernier n\oe ud sont le même. Un cycle possède au moins 3 liens (arrêtes).\\
	\end{description}

Maintenant, nous allons définir la notion de connectivité d'un graphe.
	\begin{description}
    \item[Connectivité]: un graphe est dit connexe si pour toutes paires de n\oe uds A et B, il existe au moins un chemin de A vers B.\\
	\end{description}

Tous les graphes ne sont en effet pas connexe. La figure \ref{graphe_non_connexe} montre un exemple de graphe non connexe.
	\begin{figure}
	\center
	\includegraphics[scale=0.3]{images/18_graphe_non_connexe.png}
	\caption{\label{graphe_non_connexe} Graphe non connexe}
	\end{figure}
Ce graphe possède deux composants, $AB$ et $CDE$. Ces composants (qui sont maximals) sont les parties connexes du graphe. On a les définitions suivantes:
	\begin{description}
    \item[Composant d'un graphe]: c'est un sous-graphe (partie de graphe) qui est connexe. Il est maximal s'il ne fait pas partie d'un composant plus grand.
    \item[Composant géant] : il s'agit d'un composant qui contient une fraction significative de l'ensemble des n\oe uds contenus dans le graphe.
    \end{description}
    
    Par exemple, le graphe des amitiés mondiales n'est surement pas connexe. En effet, il peut y avoir des habitants d'une île reculée qui se connaissent entre eux, mais qui n'ont pas de connection avec le reste du monde. Cependant, la majorité du monde est connecté; on a donc un énorme composant géant.\\
    
    C'est une propriété générale des grands graphes complexes: il est rare qu'ils soient connexe, mais ils ont très souvent un composant géant. Avoir plusieurs composants géants est instable: il arrive vite qu'un lien se forme entre les deux composants, formant ainsi un seul composant géant. Si avant le \textsc{xv}\textsuperscript{ème} siècle il y avait un composant géant eurasien et un autre américain, il n'a fallu qu'un lien (la découverte de l'Amérique par Christophe Colomb) pour assembler les deux composants, avec toutes les conséquences que cela a entrainé (maladies, exploitation, etc.).\\
    
Avec les éléments que nous venons de définir, on est en mesure d'analyser tout un graphe. On peut le partitionner en composants et regarder la structure interne de chacun d'entre eux. Par exemple, dans la figure \ref{gr_connexe}, on peut partitionner en deux composantes, $ABC$ et $DEF$. On constate que le lien $CD$ est un lien spécial car si on l'enlève, il déconnecte de graphe.\\
	\begin{figure}
	\center
	\includegraphics[scale=0.3]{images/18_gr_connexe.png}
	\caption{\label{gr_connexe} Graphe connexe}
	\end{figure}

\subsection{Distance entre n\oe uds}
Nous allons maintenant définir la longueur d'un chemin ainsi que la distance entre deux n\oe uds :
\begin{description}
\item[Longueur d'un chemin] : le nombre d'arêtes consécutives sur ce chemin.
\item [Distance entre 2 n\oe uds] : le chemin le plus court entre ces deux n\oe uds.
\end{description}

La méthode de calcul de distance entre deux n\oe ud est la traversée en largeur d'abord (\emph{breadth-first traversal}). Cela consiste à débuter par un n\oe ud, puis à regarder tous les n\oe uds qui sont à une distance de 1 du n\oe ud de départ. À partir de tous ceux-là, on regarder les n\oe uds qui sont à une distance 1, et donc à une distance 2 du n\oe ud de départ. On continue jusqu'à arriver au n\oe ud d'arrivée. On a donc à chaque étape constitué des couches de n\oe uds se trouvant à une certaine distance. Cet algorithme sera expliqué plus en détail par après.\\

\subsection{Phénomène du petit monde}
Le « phénomène du petit monde » aussi connu sous le vocable « paradoxe de Milgram » (du nom du psycho-sociologue Stanley Milgram) est l'hypothèse que chacun puisse être relié à n'importe quel autre individu par une courte chaîne de relations sociales. La figure \ref{petit_monde} montre la probabilité qu'ont deux personnes d'être relié par $x$ intermédiaires. 
	\begin{figure}
	\center
	\includegraphics[scale=1]{images/18_fig.png}
	\caption{\label{petit_monde} Statistiques du phénomène du petit monde}
	\end{figure}
Cette  courte chaîne de relations sociales a été approfondie par la théorie des « six degrés de séparation » affirmant qu'en prenant deux personnes, il est possible de trouver une chaîne d'amis entre eux de taille maximum 6.
Différents types de distances entre personnes existent comme:
	\begin{itemize}
	\item Le nombre d'Erdos est la distance de collaboration avec le célèbre mathématicien Paul Erdös qui a réalisé de très nombreuses co-publications. Avoir réalisé une publication en collaboration avec Erdös correspond au nombre d'Erdös 1. Avoir écrit une publication avec quelqu'un qui a co-publié avec Erdös équivaut au nombre 2, etc.
	\item Le nombre de Bacon est la distance de collaboration dans un film avec l'acteur Kevin Bacon.
	\end{itemize}
Ce phénomène de petit monde est particulièrement vrai pour les réseaux créés dynamiquement. Nous expliquerons plus en détail pourquoi cette affirmation est vraie dans la suite du cours.

\subsection{Liens forts et faibles}
Un réseau peut évoluer de différentes manières et selon différents mécanismes.
Considérons un exemple ou chaque n\oe ud correspond à une personne et les arcs correspondent à un lien d'amitié (figure \ref{fermeture_triadique}).
	\begin{figure}
	\center
	\includegraphics[scale=0.5]{images/18_fermeture_triadique.png}
	\caption{\label{fermeture_triadique} Exemple de fermeture triadique}
	\end{figure}
Dans cet exemple, on voit aue $A$ est amis à la fois avec $B$ et avec $C$. Dans cette condition, il est port probable que $B$ et $C$ deviennent eux-mêmes amis. C'est ce qu'on appelle la fermeture triadique.\\
\\
Cette situation nous amène à nous intéresser à la notion de coéficient de regroupement qui reflète la probabilité qu'un arc se crée entre deux n\oe uds dans un graphe dynamique. 
Dans l'exemple ci-dessus, cela correspondrait au fait que C et B deviennent amis. On peut démontrer qu'au plus on fait de fermeture triadique, au plus le coeficient de regroupement est elevé.

\subsubsection*{Exemple}
Une étude a été faite dans les années 1960 par le sociologue américain Mark Granovetter dans laquelle il s'est interessé aux personnes qui changent de travail, plus particulièrement à la manière dont ils trouvent un nouveau travail. Il a remarqué que les personnes trouvent du travail plutôt via des connaissances que via des amis. Cela s'explique par la structure des graphe des amis et nous amène à définir les notions de \textbf{liens forts} et de \textbf{liens faibles} (figure \ref{liens_forts_et_faibles}). La suite de cet exemple sera expliqué plus tard.\\
	\begin{figure}[!h]
	\center
	\includegraphics[scale=0.4]{images/18_liens_forts_et_faibles.png}
	\caption{\label{liens_forts_et_faibles} Liens forts et faibles}
	\end{figure}

    
\subsection{Ponts}
Pour expliquer l'exemple précédent, nous avons besoin de la notion de pont (figures \ref{Pont} et \ref{Pontlocal}).
	\begin{description}
	\item[Pont] Un lien entre $A$ et $B$ est un pont si l'enlèvement de ce lien aboutit à la séparation du graphe en deux composants disjoints.
    \item[Pont local] Un lien entre $A$ et $B$ est un pont local si l'enlèvement de ce lien aboutit au fait que deux composantes sont reliées par un chemin significativement plus long.
    \end{description}
    
    \begin{figure}
    \center
    \includegraphics[scale=0.4]{images/18_Pont.png}
    \caption{\label{Pont} Pont}
    \end{figure}
    
    \begin{figure}
    \center
    \includegraphics[scale=0.5]{images/18_Pontlocal.png}
    \caption{\label{Pontlocal} Pont local}
    \end{figure}
  % \section*{Conclusion}
\addcontentsline{toc}{section}{Conclusion}

Pour conclure, avec \LaTeX{} on obtient un rendu impeccable mais il faut s'investir pour le prendre en main.

  % \newpage
  \phantomsection\addcontentsline{toc}{section}{Références}
\begin{thebibliography}{ABC}	
    \bibitem[Nis]{nis} Nimal Nissanke. \emph{Introductory Logic and Sets for Computer Scientists}.
    \bibitem[LPP]{eas} David Easley and Jon Kleinberg. \emph{Networks, Crowds, and Markets: Reasoning About a Highly Connected World}.
    \bibitem[Pro]{pro} Leon Sterling and Ehud Shapiro. \emph{The Art of Prolog}, MIT Press.
\end{thebibliography}

\end{document}

